\section{Test von Batterie und Fahrzeug}
Da am Ende die Batterie in das Fahrzeug eingebaut werden soll, muss auch diese Kombination getestet werden. Um in jedem Fall die Sicherheit zu garantieren, werden dabei zuerst die sicherheitsrelevanten Funktionen im Fehlerfall getestet: \begin{itemize}
	\item Test des Hauptschalters - der Antrieb wird bei Betätigung ausgeschaltet
	\item Ausschalten bei überhitzter Batterie, wobei ein Sensor extern erwärmt wird
	\item Ausschalten bei Überlast, welche mit zusätzlichen Widerständen simuliert wird
	\item Funktion der Bremse auch bei Gegenmoment durch den Motor
\end{itemize}

Für die Tests des Antriebs im Normalbetrieb werden, nach Absprache mit dem Besitzer, folgende Versuche durchgeführt: \begin{itemize}
	\item Antriebsprüfung des aufgebockten Fahrzeuges
	\item Antriebsprüfung auf dem Parkplatz der Fachhochschule
	\item Antriebsprüfung auf einer längeren Strasse, um das Verhalten bei höheren Geschwindigkeiten zu Prüfen
\end{itemize}

Nach dem Abschluss dieser Tests werden die Zusatzfunktionen der Hilfsspannung überprüft. Diese umfassen die korrekte Funktion des Blinkers sowie der Hupe.