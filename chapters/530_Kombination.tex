\section{Test von Batterie und Fahrzeug}
Da am Ende die Batterie in das Fahrzeug eingebaut werden soll, muss auch diese Kombination getestet werden. Um in jedem Fall die Sicherheit zu garantieren, werden dabei zuerst die sicherheitsrelevanten Funktionen im Fehlerfall getestet: \begin{itemize}
	\item Test des Hauptschalters - der Antrieb wird bei Betätigung ausgeschaltet
	\item Ausschalten bei überhitzter Batterie, wobei ein Sensor extern erwärmt wird
	\item Ausschalten bei Überlast, welche mit zusätzlichen Widerständen simuliert wird
	\item Funktion der Bremse auch bei Gegenmoment durch den Motor
\end{itemize}

Für die Tests des Antriebs im Normalbetrieb werden nach Absprache mit dem Besitzer, folgende Versuche durchgeführt: \begin{itemize}
	\item Antriebsprüfung des aufgebockten Fahrzeuges
	\item Antriebsprüfung auf dem Parkplatz der Fachhochschule
	\item Antriebsprüfung auf einer längeren Strasse, um das Verhalten bei höheren Geschwindigkeiten zu Prüfen
\end{itemize}

Nach dem Abschluss dieser Tests wurden die Zusatzfunktionen der Hilfsspannung überprüft. Diese umfassen die korrekte Funktion des Blinkers sowie der Hupe.

\color{blue} Einige Tests, welche bereits für die Batterie durchgeführt werden, wurden im Fahrzeug wiederholt. Das liegt darin begründet, dass die Verkabelung neu aufgebaut wurde, womit sich auch hier potentiell Fehler einschleichen konnten. Dabei wurden folgende Tests am aufgebockten Fahrzeug (mit dem leerlaufenden Motor als Last) durchgeführt (auch hier steht wieder \textit{kursiv} das erwartete und beobachtete Ergebnis): \begin{itemize}
	\item Betätigung des Hauptschalters bei drehendem Motor: \textit{Die Stromzufuhr zum Motor wird unterbrochen. Trotzdem kann der Motor noch ungehindert auslaufen. Der Motor lief ohne Spannung langsam aus.}
	\item Mit einer Wärmebildkamera werden sämtliche Anschlussstellen im Fahrzeug (auch bereits bestehende) untersucht: \textit{Schlechte Kontakte können damit erkannt und behoben werden. Die einzige überdurchschnittlich erwärmte Stelle war der Shunt, was so zu erwarten war.}
	\item Das $230$ VAC-Netz wird angeschlossen: \textit{Das $12$ VDC-System fällt - ausser beim BMS - aus, folglich wird auch der Hauptschalter geöffnet. Beim Einstecken des Ladekabels wurde der Motor ausgeschaltet und die Hinterräder drehten langsam aus.}
	\item Das alte Messgerät des Fahrzeugs wird angeschlossen: \textit{Die Anzeige der verbleibenden Ladung ist proportional zur Restladung der stärker entladenen Batterie und die Stromwerte sind plausibel. Die Plausibilität konnte mit den Werten aus dem BMS (Strom und Ladezustand) abgeglichen werden.}
\end{itemize}

Ausserdem können im Fahrzeug einige Tests nun zum ersten Mal durchgeführt werden: \begin{itemize}
	\item Mit beiden Batterien wird der Stufenschalter durchgegangen: \textit{Die Drehzahl der Hinterräder erhöht sich mit jeder Stufe. Dieses Verhalten konnte grob bestätigt werden. Aufgrund des Leerlaufes des Reihenschlussmotores sind die Werte allerdings mit Vorsicht zu geniessen.}
	\item Die korrekte Verbindung zwischen den Batterien wird untersucht, das heisst die Zugehörigkeit der Steuerungen und der Ladegeräte: \textit{Die Steuerungen und Ladegeräte überwachen bzw. laden jeweils nur die zugehörige Batterie. Die korrekte Verkabelung wurde bestätigt.}
\end{itemize}
\newpage


\color{black}