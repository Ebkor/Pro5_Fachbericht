\section{Steuerung}
In diesem Kapitel werden die einzelnen Schaltvorgänge und -zustände des Fahrzeuges analysiert und beschrieben. Dabei wird besonders auf das Zusammenspiel von mechanischem zum elektrischem Bereich eingegangen, aber auch rein mechanische Vorgänge sollen vorgestellt werden. \textcolor{blue}{Die für den Fahrer relevanten Informationen wurden zusammenfassen in einer Kurzanleitung zusammengefasst. Diese ist im Anhang unter \ref{app:quickstart} zu finden.}

\subsection{Stufenschalter}
Die wohl direkteste Verbindung von mechanischer und elektrischer Steuerung geschieht mit dem Stufenschalter. Dieser besitzt dabei mechanisch mehrere Zustände, die die benötigten elektrischen Schaltungen herstellen.

\paragraph{Parken}
Im Zustand Parken ist der \textsc{Detroit} durch die Feststellbremse gesichert. Diese blockiert die Hinterräder und unterbricht gleichzeitig den Hauptstromkreis durch den Cut-Out-Switch (Originaler Hauptschalter). Gleichzeitig zeigt der Stufenschalter in der neutralen Position vertikal nach oben, womit der Reverse-Switch (Schalter für Ankerpolaritätswechsel) ebenfalls keinen Kontakt macht. Somit ist der Hauptstromkreis sogar zweifach unterbrochen.

\paragraph{Parken zu Neutral}
Im Zustand Neutral ist der Oldtimer nicht mehr durch die Feststellbremse gesichert. Um diese zu lösen sind zwei Schritte gleichzeitig notwendig. Zum einen soll der Stufenschalter in waagerechte Position gebracht werden, um ihn dann anschliessend zu sich zu ziehen. Das ist sogleich die Motorbremse, welche die eigentliche Bremsung bei Bedarf unterstützen kann. Damit wird der Sicherungshebel des Cut-Out-Switch betätigt. Somit ist es nun möglich mit einem zweiten Schritt die Feststellbremse zu betätigen, wodurch sich diese und sogleich der Cut-Out-Switch aus der Verankerung heben lassen, um sich zu lösen und den Hauptstromkreis zu schliessen.

\paragraph{Neutral zu Parken}
Durch die Betätigung der Feststellbremse wird eine Feder über Zahnräder gespannt, der Cut-Out-Switch unterbricht den Stromkreis und verriegelt diese Position mechanisch. Zu beachten ist, dass die Feststellbremse mit voller Kraft durchgedrückt wird, damit der Stromkreis klar unterbrochen ist. \textcolor{blue}{Im aktuellen Zustand ist dies rein elektrisch nicht mehr zwingend nötig, da die Ausschaltung mit einem modernen gasisolierten Hauptschalter erfolgt und der Cut-Out-Switch nicht mehr in Funktion ist.}

\paragraph{Neutral zu Vorwärts}
Durch das Stellen des Stufenschalters in waagrechte Position wird der Reverse-Switch in die Position Vorwärts geschaltet. Durch Wegdrücken des Stufenschalters rastet dieser ein. Dadurch wird ein weiterer Hebel in Bewegung gesetzt, welcher bis hin zum Stufenschalter führt. So dreht dieser nun in Position eins und das Fahrzeug fährt im ersten Gang los. Ist das Anfahren nun geschehen, kann in den zweiten Gang geschaltet werden. Dies erfolgt durch weiteres Wegdrücken des Stufenschalters, bis dieser wieder einrastet. Höhere Gänge bis Stufe fünf können mit der selben Methode erreicht werden.

\paragraph{Neutral zu Rückwärts}
Bei einer 45$^\circ$-Stellung des Stufenschalters nach oben verschiebt sich der Reverse-Switch in Position Rückwärts, womit der Stromfluss umkehrt. Durch gewohntes Schalten in den ersten Gang dreht der Stufenschalter wieder in Position eins und das Fahrzeug lässt sich rückwärts bewegen.