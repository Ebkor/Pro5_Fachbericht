\section{Steuerung}
In diesem Kapitel werden die einzelnen Schaltvorgänge und -zustände des Fahrzeuges analysiert und beschrieben. Dabei wird besonders auf das Zusammenspiel von mechanisch zu elektrisch eingegangen, aber auch rein mechanische Vorgänge sollen vorgestellt werden.

\subsection{Ganghebel}
Die wohl direkteste Verbindung von mechanischer und elektrischer Steuerung geschieht mit dem Ganghebel. Dieser kennt dabei mechanisch mehrere Zustände, die die benötigten elektrischen Schaltungen herstellen.

\paragraph{Parken}
Im Zustand Parken ist der Detroit durch die Handbremse gesichert. Diese blockiert die Hinterräder und unterbricht gleichzeitig den Hauptstromkreis durch den Cut-Out-Switch. Gleichzeitig zeigt der Ganghebel in der neutralen Position vertikal nach oben, womit der Reverse-Switch ebenfalls keinen Kontakt macht. Somit ist der Hauptstromkreis sogar zweifach unterbrochen.

\paragraph{Parken zu Neutral}
Im Zustand Neutral ist der Oldtimer nicht mehr durch die Handbremse gesichert. Um diese zu lösen sind zwei Schritte gleichzeitig notwendig. Zum einen soll der Ganghebel in waagerechte Position gebracht werden um ihn dann anschliessend zu sich zu ziehen. Das ist sogleich die Motorbremse, welche die eigentliche Bremsung bei Bedarf unterstützen kann. Damit wird der Sicherungshebel des Cut-Out-Switch betätigt. Somit ist es nun möglich mit dem zweiten Schritt die Handbremse zu betätigen, wodurch sich die Handbremse und sogleich der Cut-Out-Switch aus der Verankerung heben lassen um sich zu lösen und den Hauptstromkreis zu schliessen.

\paragraph{Neutral zu Parken}
Durch die Betätigung der Handbremse wird eine Feder über Zahnräder gespannt und der Cut-Out-Switch unterbricht den Stromkreis. Zu beachten ist, dass die Handbremse mit voller Kraft durchgedrückt wird, damit der Stromkreis klar unterbrochen ist.

\paragraph{Neutral zu Vorwärts}
Durch das Stellen des Ganghebels in waagrechte Position wird der Reverse-Switch in die Position Vorwärts geschaltet. Das Wegdrücken des Ganghebels wird mit einem Einrasten. Dies setzt einen Hebel in Bewegung, welcher bis hin zum Stufenschalter führt. So dreht dieser nun in Position 1 und das Fahrzeug fährt im 1. Gang los. Ist das Anfahren nun geschehen, kann in den 2. Gang geschaltet werden. Dies erfolgt durch weiteres Wegdrücken des Ganghebels, bis dieser wieder einrastet. Höhere Gänge bis zum 5. können mit der selben Methode erreicht werden.

\paragraph{Neutral zu Rückwärts}
Bei einer 45$^\circ$-Stellung des Ganghebels verschiebt sich der Reverse-Switch in Position Rückwärts, womit der Stromfluss, wie der Name des Switch schon sagt, umkehrt. Durch gewohntes Schalten in den ersten Gang dreht der Stufenschalter wieder in Position 1 und das Fahrzeug lässt sich rückwärts lenken.