\chapter*{b\quad Persönliche Anmerkung}\addcontentsline{toc}{chapter}{b.\ \ Persönliche Anmerkung}
Obwohl zu Beginn angedacht war, bereits im Projekt 5 das Fahrzeug zum Laufen zu bringen, dauerte der Abschluss der Arbeit doch deutlich länger. Insbesondere wurden viele handwerkliche Arbeiten sowie die Batterie selbst massiv unterschätzt, sodass diese den zeitlichen Rahmen sprengten. Aber auch das Auffinden von Informationen ist bei solch einem alten Fahrzeug nicht einfach und oft nur durch Nachverfolgen der Leitungen möglich.

Trotzdem hat uns die Arbeit sehr viel Spass gemacht. Ein solches Projekt hat man -- wie Felix Jenni bereits zu Beginn erwähnte -- vermutlich nur einmal im Leben. Doch auch abgesehen vom speziellen Charme des Detroits bot die Arbeit vieles: So konnte moderne Technik mit alten Methoden kombiniert werden. Fingerspitzengefühl für das Bestücken von Platinen traf auf handwerkliches Können beim Verlegen von Antriebsleitungen. Ein hundertjähriges Fahrzeug kann fast schon mit einem modernen Tesla konkurrieren. Aus all diesen Gründen hat uns der Umbau viel Freude bereitet.

Trotzdem soll auf einige Punkte eingegangen werden, welche die Arbeit am Projekt in die Länge gezogen haben. Dies ist nicht als Rechtfertigung von unserer Seite gedacht, sondern soll nur den Prozess nachvollziehbar gestalten.

Ein ursprünglich weiterer Wunsch des Kunden war es, das Fahrzeug -- genauer gesagt dessen Batterie -- als Pufferspeicher für seine Solaranlage benutzen zu können. Das der damit verbundene Aufwand zur Erstellung einer Schnellladevorrichtung nicht umsetzbar ist, hat sich schnell gezeigt. Trotzdem wurde an der grossen Batterie festgehalten. Die Parallelschaltung der Zellen, aber auch die physikalisch grossen und schweren Batterien waren im Umgang nicht immer leicht zu handhaben.

Viele Punkte wurden nach einer ersten Betrachtung mit einem "`das machen wir dann noch schnell"' beurteilt. Dies kann sich aber durchaus in die Länge ziehen, da der Teufel wie so oft im Detail steckt. Ein Beispiel dafür ist die Beleuchtung: Wurde davon ausgegangen, dass die Verkabelung im Grund genommen korrekt ist und lediglich kleine Anpassungen gemacht werden müssen. Tatsächlich mussten aber einige Kabel ersetzt oder komplett neu verlegt werden und auch beinahe alle Sockel wurden ersetzt.

Ein weiterer Punkt, welcher viel Zeit benötigt hat, war das Testen der Batterie. So werden für ein komplettes Entladen und Aufladen einer Batterie zwei Arbeitstage benötigt. Muss dabei ein Parameter getestet werden, so vergeht jedes Mal viel Zeit. Auch wenn währenddessen andere Arbeiten verrichtet werden konnten, blockierte die Batterie so doch einige weitere Arbeiten.

Trotz diesen Punkten konnten die Arbeiten am Fahrzeug ohne Reduktion des Funktionsumfanges und aller nötigen Tests durchgeführt werden. Umso grösser war die Freude, aber auch die Erleichterung, als die erste Fahrt mit dem Detroit unternommen wurde. Wir werden unsere Bachelorarbeit damit in guter Erinnerung behalten!