\chapter{Der Detroit im Originalzustand}
In diesem ersten Kapitel soll der Detroit, wie er im Jahre 1918 war, vorgestellt werden. Dabei wird sowohl ein Blick in die Geschichte mit den damaligen Elektrofahrzeugen, als auch auf den Detroit selbst eingegangen. Von der Funktion des Detroits werden nur die Punkte erläutert, die sich im Vergleich zu heute geändert haben. Dies sind die Batterie und das zugehörige Ladegerät.

\section{Geschichte}

In diesem Kapitel soll kurz die Firma Detroit Electric Car und der Detroit selbst aus dem Jahre 1918 vorgestellt werden.

\subsection{Firma}

Der Detroit aus dem Jahre 1918 ist bei weitem nicht der erste seiner Art. Seit dem Jahr 1907 fabrizierte die amerikanische Firma aus Detroit, welche bis dahin Pferdekutschen produzierte, Elektroautos. In den 1910er Jahren erreichten sie ihren Höhepunkt mit 1000 bis 2000 verkauften Autos pro Jahr. Mehrheitlich Frauen fuhren Detroits, da sie sich zu schade für normale Autos waren. Diese hatten damals noch keinen elektrischen Anlasser und somit musste der Motor mühsam per Hand angekurbelt werden. Auch waren damals Schadstofffilter noch nicht bekannt, womit sehr viel ungesunder Qualm in die Umwelt und die Gesichter der Damen geblasen wurde. Die Firma fabrizierte ihre Fahrzeuge bis in die späten 1930er Jahre, danach wurden Elektroautos nicht weiter gefördert und die mächtige Ölindustrie unterstützte Fahrzeuge mit Verbrennungsmotoren. Somit war dies das Ende der Firma Detroit Electric Car.

In der heutigen Zeit ereignet sich mehrheitlich wieder ein Umdenken in den Köpfen der Menschen. Elektroautos sind umweltfreundlich, günstiger im Unterhalt und modern, obwohl die Geschichte der Elektrofahrzeuge weiter zurückreicht als die der Benzin- und Dieselfahrzeuge.

\subsection{Detroit aus 1918}

Bei dem Fahrzeug handelt es sich um einen Detroit Electric Car aus dem Jahre 1918. Die korrekte Modellbezeichnung ist leider unbekannt, da es bei Detroit üblich war, die älteren Modelle zurückzukaufen und zu restaurieren. So ist es möglich, dass ein Fahrzeug mehrere Modellbezeichnungen besitzt. Weiter sind auch die früheren Besitzer dieses Detroits unbekannt. Es ist lediglich bekannt das unser Auftraggeber, Herr Jäger, das Fahrzeug von Herrn Vögtlin gekauft hat. Vor diesem Kauf wurde das Fahrzeug in den USA restauriert und von ihm ersteigert.

\newpage