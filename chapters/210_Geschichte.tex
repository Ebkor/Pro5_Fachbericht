\chapter{Der Detroit im Originalzustand}
In diesem ersten Kapitel soll der \textsc{Detroit}, wie er im Jahre 1918 war, vorgestellt werden. Dabei wird sowohl ein Blick auf die geschichtliche Entwicklung der Elektrofahrzeuge als auch auf den \textsc{Detroit} selbst geworfen. Von der Funktion des Detroits werden nur die Punkte erläutert, die sich im Vergleich zu heute geändert haben. Dies sind die Batterie und das zugehörige Ladegerät. Alle unveränderten Funktionen werden im Kapitel \ref{aktuell} "`Der Detroit im aktuellen Zustand"' vorgestellt.

\section{Geschichte}

Nachfolgend soll kurz die Firma \textsc{The Detroit Electric Car Company} vorgestellt werden. Im Anschluss wird spezifisch auf das Projektfahrzeug eingegangen.

\subsection{Firma}

Der \textsc{Detroit} aus dem Jahre 1918 ist bei weitem nicht der erste seiner Art. Seit dem Jahr 1907 fertigte die amerikanische Firma aus Detroit, welche bis dahin Pferdekutschen produzierte, Elektroautos. In den 1910er Jahren erreichten sie ihren Höhepunkt mit 1000 bis 2000 verkauften Autos pro Jahr. Mehrheitlich Frauen fuhren \textsc{Detroit}s, da sie sich zu schade für normale Autos waren. Diese Benzin- und Dieselfahrzeuge hatten damals noch keinen elektrischen Anlasser und somit musste der Motor mühsam per Hand angekurbelt werden. Auch waren damals Schadstofffilter noch unbekannt, womit sehr viele ungesunde Abgase in die Umwelt und die Gesichter der Damen geblasen wurde. Die Firma fabrizierte ihre Fahrzeuge bis in die späten 1930er Jahre. Danach wurden Elektroautos nicht weiter gefördert und die mächtige Ölindustrie unterstützte Fahrzeuge mit Verbrennungsmotoren. Somit war dies das Ende der Firma \textsc{Detroit Electric Car}.
In der heutigen Zeit ereignet sich mehrheitlich wieder ein Umdenken in den Köpfen der Menschen. Elektroautos sind umweltfreundlicher, günstiger im Unterhalt und modern, obwohl die Geschichte der Elektrofahrzeuge weiter zurückreicht als die der Benzinfahrzeuge.

\subsection{Der \textsc{Detroit} von 1918}
Beim Projektfahrzeug handelt es sich um einen blauen \textsc{Detroit Electric Car} aus dem Jahre 1918. Die korrekte Modellbezeichnung ist leider unbekannt, da es bei \textsc{Detroit} üblich war, die älteren Modelle zurückzukaufen und zu restaurieren. Weiter sind auch die früheren Besitzer dieses \textsc{Detroits} unbekannt. Es ist lediglich bekannt, dass unser Auftraggeber, Herr Jäger, das Fahrzeug von einem Bekannten gekauft hat. Vor diesem Kauf wurde das Fahrzeug in den USA restauriert. Jedoch stand das Fahrzeug über Jahre in einer Garage, da dieses nicht funktionierte und wurde erst anfangs 2017 an die FHNW Brugg-Windisch überführt, um die Fahrtüchtigkeit und die Umrüstung von Blei- auf Lithium-Ionen-Batterien sicherzustellen.