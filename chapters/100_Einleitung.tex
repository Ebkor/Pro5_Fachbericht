\chapter{Einleitung}

Oldtimer sind begehrte Sammlerobjekte. Leider ist es aber die alte Technik, die diesen Fahrzeugen oft zum Verhängnis wird. Häufig existieren keine passenden Ersatzteile mehr, sodass diese in langer Handarbeit nachgebaut werden müssen. Viele Anpassungen müssen durchgeführt werden, um den heutigen Sicherheitsbestimmungen Folge zu leisten. Ein anderer Ansatz ist deshalb, einen Teil der oftmals unsichtbaren Technik zu modernisieren und so die Probleme, beispielsweise der Ersatzteilbeschaffung, zu umgehen, was jedoch der historischen Korrektheit widerspricht.

Genau diese Vorgehensweise wurde beim vorliegenden Projekt gewählt: Ein hundertjähriges Elektrofahrzeug von 1918 soll auf moderne Lithium-Ionen-Akkumulatoren umgerüstet werden. Da das Fahrzeug, insbesondere die Batterie, bereits einmal umgebaut wurde, war sie nicht mehr von historischem Wert, sodass diese Arbeiten mit gutem Gewissen ausgeführt werden konnten. Gleichzeitig sollen die anderen elektrischen Teile des Fahrzeugs begutachtet und gewartet werden. Ebenfalls soll die originale Schaltung möglichst beibehalten werden.

Die neue Batterie, die aus einem verunfallten Elektrofahrzeug stammt, musste so auf den \linebreak\textsc{Detroit} angepasst werden, dass an dessen Schaltung keine Änderung gemacht werden musste. Das Batteriemanagementsystem zur Überwachung und Ausgleichung der Einzelzellen sowie das Ladegerät mussten jedoch erneuert werden, da die Komponenten vom Unfallfahrzeug nicht verwendbar waren. Da solche Bauteile bereits existieren, wurde auf eine Eigenentwicklung verzichtet und entsprechende Produkte wurden eingekauft.

So wurde evaluiert, welche Produkte zum \textsc{Detroit} passen und diese in das Fahrzeug eingebaut. Dabei war oftmals handwerkliches Geschick von Nöten um die neuen Komponenten unsichtbar im Fahrzeug unterzubringen. Für das Verständnis der Funktion mussten
viele originale Baugruppen geöffnet werden, wobei diese bei dieser Gelegenheit gerade gereinigt sowie auf ihre korrekte Funktion überprüft wurden. Damit konnte nachvollzogen werden, wie diese funktionieren, um die neuen Komponenten entsprechend anzupassen.

Dieser Bericht beschränkt sich bewusst nicht nur auf die elektrotechnische Entwicklung. So werden alle Komponenten, an denen gearbeitet wurde, erläutert, um so möglichst viele Informationen in diesem Bericht zu präsentieren. Das Thema soll nicht nur Ingenieure ansprechen, sondern auch von Laien verstanden werden. Aus diesem Grund finden sich immer wieder Erklärungen zu elektrotechnischen Vorgängen, sodass die Arbeit ohne viel Vorkenntnisse verstanden werden kann. Weiter dient der Bericht auch als Betriebsdokument für spätere Umbauten oder Reparaturen.