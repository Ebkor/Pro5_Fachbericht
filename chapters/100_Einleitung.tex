\chapter{Einleitung}

Oldtimer sind begehrte Sammlerobjekte. Leider ist es aber auch die alte Technik, die diesen Fahrzeugen oft zum Verhängnis wird. So müssen Ersatzteile mit viel Handarbeit neu produziert oder müssen weitere Anpassungen gemacht werden, um den aktuellen Sicherheitsbestimmungen zu genügen. Ein anderer Ansatz ist deshalb, einen Teil der oftmals unsichtbaren Technik zu modernisieren und so die Probleme, beispielsweise der Ersatzteilbeschaffung, zu umgehen, was jedoch der historischen Korrektheit widerspricht.

Genau diese Vorgehensweise wurde beim vorliegenden Projekt gewählt: Ein beinahe einhundert jähriges Elektrofahrzeug von 1918 soll auf moderne Lithium-Ionen-Akkumulatoren umgerüstet werden. Da das Fahrzeug, insbesondere die Batterie, bereits einmal umgebaut wurde, war diese nicht mehr von historischem Wert, sodass diese Arbeiten mit gutem Gewissen ausgeführt werden konnten. Gleichzeitig sollte auch das sonstige Fahrzeug, insbesondere der elektrische Teil, begutachtet und gewartet werden, wobei die originale Schaltung beibehalten wird.

Die neue Batterie, die aus einem verunfallten Elektrofahrzeug stammt, musste so auf den Detroit angepasst werden, das an dessen Schaltung keine Anpassung gemacht werden musste. Auch das Batteriemanagementsystem zur Überwachung und Ausgleichung der Einzelzellen sowie das Ladegerät musste erneuert werden, da die Komponenten vom Unfallfahrzeug nicht übernommen werden konnten. Da solche Bauteile bereits existieren, wurde auf eine Eigenentwicklung verzichtet und auf käufliche Produkte zurück gegriffen.

So musste evaluiert werden, welche Produkte zum Detroit passen und diese in das Fahrzeug eingebaut werden. Dabei war oftmals handwerkliches Geschick von Nöten, um die neuen Komponenten unsichtbar im Fahrzeug unterzubringen. Für das Verständnis der Funktion mussten viele originale Baugruppen geöffnet werden. Damit konnte nachvollzogen werden, wie diese funktionieren, um die neuen Komponenten entsprechend anzupassen.

Dieser Bericht beschränkt sich bewusst nicht nur auf die elektrotechnische Entwicklung. So sollen alle Komponenten, an denen gearbeitet wurde, erläutert werden, um so möglichst viele Informationen in diesem Bericht zu präsentieren. Diese Informationen sind dabei so gehalten, dass sie nicht nur von Ingenieuren, sondern auch von Laien verstanden werden können. Aus diesem Grund wurden auch immer wieder Erklärungen zu elektrotechnischen Vorgängen eingefügt, sodass die Arbeit mit möglichst wenig Vorkenntnissen verstanden werden kann.