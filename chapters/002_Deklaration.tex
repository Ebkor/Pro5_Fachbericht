\chapter*{a\quad Deklaration Projektarbeit 5 \& 6}\color{blue}\addcontentsline{toc}{chapter}{a.\ \ Deklaration Projektarbeit 5 \& 6}
Um einen Bericht, welcher den kompletten Umbau des Fahrzeuges dokumentiert, zu erhalten, wurde beschlossen, für die Projektarbeiten 5 \& 6 einen gemeinsamen Bericht zu erstellen. Dieser ist bewusst ausführlich, um sämtliche Details des Fahrzeuges, welche erwähnenswert sind, erwähnen zu können. Die folgende Deklaration soll aufzeigen, welche Inhalte zu welcher Projektarbeit erstellt wurden. Im Projekt 5 lag dabei der Fokus in der Lösungsfindung, während im Projekt 6 die gewählte Lösung umgesetzt wurde, weswegen in dieser zweiten Arbeit nur noch Änderungen oder neue Erkentnisse dokumentiert wurden.

\section*{Unverändert aus dem Bericht zum Projekt 5 übernommen}
\begin{compactitem}
\item \textbf{2. Der Detroit im Originalzustand}
\item 2.1 Geschichte
\item 2.2 Funktion der Bleibatterie
\item 2.3 Originale Lademethoden
\item 2.4 Alternative Konzepte zur Spannungsregelung
\item \textbf{3. Das Peugeot-Unfallauto}
\item 3.1 Batterie
\item 3.2 Stromrichter und Motor
\item 3.3 Lademodi
\item 4.2 Antrieb
\item 4.4 Steuerung
\item 4.5 Das neue Ladegerät
\item \textbf{B. Original Starkstromschema}
\item \textbf{D. Schema Steuerplatine}
\end{compactitem}

\section*{Mit Änderungen aus dem Bericht zum Projekt 5 übernommen}
\begin{compactitem}
\item \textbf{1. Einleitung}
\item \textbf{4. Der Detroit im aktuellen Zustand}
\item 4.1 Batterie
\item 4.3 Hilfsstromversorgung
\item \textbf{5. Test des Detroits}
\item 5.1 Test des Fahrzeugs
\item 5.2 Test der Batterie
\item 5.3 Test von Batterie und Fahrzeug
\item \textbf{6. Schlusswort}
\item \textbf{7. Literaturverzeichnis}
\item \textbf{A. Projektauftrag}
\item \textbf{F. Parameter BMS}
\end{compactitem}

\section*{Im Projekt 6 neu erstellt}
\begin{compactitem}
\item 4.6 Modifizierte Baugruppen
\item \textbf{C. Original Lichtschema}
\item \textbf{E. Zeichnungen neuer Shunt}
\item \textbf{G. Bilder vom "`Innenleben"' des Detroits}
\item G.1 Mechanisch
\item G.2 Elektrisch
\end{compactitem}