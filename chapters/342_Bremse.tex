\subsection{Bremsen des Detroits}\label{bremse}

In diesem Kapitel werden die verschiedenen Bremstechniken des \textsc{Detroits} behandelt. Zum einen ist das die Trommelbremse, welche die Räder direkt bremst und zum anderen die Motorbremse, die den ganzen Antriebsstrang bremst. Zu guter Letzt wird noch ein Blick auf die elektrische Bremse gerichtet. Diese ist im \textsc{Detroit} nicht verbaut, gehören jedoch bei modernen Elektrofahrzeugen zum Standard.

\paragraph{Trommelbremse}

Grundsätzlich ist die Trommelbremse eine Reibungsbremse bei welcher Bremsbeläge auf eine zylindrische Fläche wirken. Diese Art der Bremsung ist veraltet und wird heutzutage durch die Scheibenbremse ersetzt. Durch betätigen der Bremse mit dem Fuss wird ein Hebel in Bewegung versetzt, welcher an den Bremsen der beiden hinteren Rädern bremst. Die Vorderräder des \textsc{Detroits} sind in jedem Fall ungebremst. Die normale Bremse kann auch durch die Handbremse arretiert werden. Dabei ist die Feder der Bremse immer angezogen.

\paragraph{Motorbremse}

Die Motorbremse ist sozusagen eine Manöverbremse, die bei kurzen und sanften Bremsungen eingesetzt werden kann. Dabei drücken Bremsbeläge direkt auf die Schwungmasse des Rotors, wodurch dieser gebremst wird. Diese kann durch Ziehen am Ganghebel betätigt werden.

\paragraph{Elektrische Bremse}
Bei modernen elektrischen Fahrzeugen wird das Fahrzeug elektrisch gebremst. Dazu wird der Motor als Generator verwendet und die Energie entweder an den Ursprungsort (Netz oder Batterie) zurück gespeist oder über einen Heizwiderstand in Wärme umgewandelt. Dies hat gegenüber herkömmlichen mechanischen Bremsen den Vorteil, dass die Bremse keinen Verschleiss aufweist und auch nicht überhitzen kann, sofern die Energie beispielsweise an ein Netz oder eine Batterie abgegeben werden kann.

Das Erzeugen elektrischer Energie ist mit einer Gleichstrommaschine möglich. Zu diesem Zweck muss die induzierte Spannung (siehe Kapitel \ref{gm}) im Anker grösser werden als die Spannung, welche an den Klemmen angelegt wird, also beispielsweise die Batteriespannung. Ist dies der Fall, kehrt sich die Flussrichtung des Stromes um, wodurch die Maschine Energie liefert und als Generator arbeitet.

Mit dem Motor des \textsc{Detroits} könnte diese Funktion rein theoretisch ausgeführt werden, es ist jedoch aus mehreren Gründen nicht praxistauglich. Die Schaltung des Stufenschalters schaltet den Motor in jedem Fall als Reihenschlussmotor. Wird bei dieser Motorverschaltung von aussen Moment zugeführt, so beschleunigt sich lediglich der Anker und es wird weniger Strom für den Antrieb benötigt. Durch den geringeren Stromfluss sinkt jedoch auch die Erregung und mit ihm die induzierte Spannung, es wird also wieder ein neues Gleichgewicht hergestellt.

Rein theoretisch betrachtet wäre es jedoch möglich, mit einer anderen Verschaltung des Stufenschalters ebenfalls eine fremderregte Maschine zu verschalten. Dadurch wäre es möglich, dass die induzierte Spannung des Ankers grösser wird als die Klemmenspannung der Batterien und sich dadurch die Stromrichtung umdreht, ohne dabei das Erregerfeld zu beeinflussen. Um damit aber den Strom, und damit das Bremsmoment, halbwegs konstant zu halten wäre es nötig, entweder die Erregung oder die Klemmenspannung des Motors feinstufig regeln zu können. Dies würde einerseits eine saubere Regelung, andererseits auch zumindest im Falle der Klemmenspannung einen Stromrichter erfordern. Vermutlich wäre jedoch auch ein zweiter Stromrichter nötig gewesen, um die hohen Ströme bei niedriger Spannung für die Erregerwicklungen herzustellen.

Wird ausserdem betrachtet, dass die Leistung des Fahrzeugs sowieso nicht für steile Bergfahrten gereicht hätte, bei denen bei der Abfahrt (auch aufgrund der damals schlechteren Strassen) wirklich stark gebremst worden wäre und auch die maximal erreichbare Geschwindigkeit verhältnismässig klein war, kommt man zum Schluss, dass die rekuperierbare Energie nicht besonders gross geworden wäre (insbesondere im Vergleich zu heutigen, deutlich leistungsfähigeren Fahrzeugen). Zusammen mit dem grossen Aufwand für die elektrische Bremse hätte dies ein sehr schlechtes Verhältnis von Aufwand zu Nutzen ergeben.