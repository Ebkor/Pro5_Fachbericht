\section{Verbesserungsm�glichkeiten}
\color{blue}
W�hrend den Tests, aber auch schon beim Zusammenbau, wurden einige Schwachstellen gefunden, welche nicht verbessert wurden. Auf diese Punkte soll in diesem Kapitel eingegangen werden aus dem Grund, um solche Fehler bei einem zuk�nftigen Umbau von Anfang an zu vermeiden oder sie auch im Anschluss an dieses Projekt noch zu verbessern. 

\subsection{Batteriekisten}
Die beiden Kisten f�r die Batterien sind sehr gross und entsprechend auch schwer. Ausserdem konnten sie nur mit Anpassungen beide im Heck des Fahrzeugs platziert werden, sodass dort der Platz f�r die Anschl�sse sehr knapp ist. Um eine m�glichst hohe Batteriekapazit�t zu erm�glichen wurden jeweils drei Zellen parallel geschaltet, worin der Grund f�r die grossen Batteriekisten zu finden ist. Ausserdem befinden sich in den Batteriekisten jeweils noch die Dioden, die Sicherungen sowie die Batteriemanagementsysteme.

Eine Batteriekiste, welche auf Parallelschaltungen verzichten w�rde, k�nnte beide Batterien sowie das wichtige Zubeh�r (siehe oben) beinhalten und w�re immer noch kleiner als eine einzelne aktuelle Batteriekiste. Dagegen spricht hingegen die Reichweite, welche aufgrund der deutlich geringeren Batteriekapazit�t nur noch ungef�hr einem Drittel entspricht.

Ein weiterer Punkt, welcher bei einer neuen Batterie unbedingt beachtet werden sollte, ist die Zug�nglichkeit zu Verschleissteilen, insbesondere den Sicherungen. Der schlechteste Fall, welcher aktuell eintreten kann, ist das Ausl�sen der Sicherung von Batterie eins. Um diese zu ersetzen muss Batterie zwei (welche auf Batterie eins liegt) entfernt werden und erst anschliessend kann der Deckel von Batterie eins ge�ffnet werden, womit die Sicherung ersetzt werden kann. Eine neue Batteriekiste sollte unbedingt einen direkten Zugang (beispielsweise eine Seiten�ffnung) zur Sicherung erm�glichen.

\subsection{Steuerplatine}
Wie bereits unter \ref{steuerplatine} beschrieben wurde besitzt die Steuerplatine einen Fehler. Dieser Fehler konnte mit einer Schaltung auf einer zus�tzlichen Leiterplatte behoben werden, diese L�sung ist aber nicht optimal. Eine sch�ne L�sung w�re eine neue Steuerplatine, welche entweder bereits diese Inverterschaltung beinhaltet oder aber eine komplett andere Schaltung zur Ansteuerung der Komponenten.

\subsection{Tachometer}
Im Fahrzeug ist ein alter Tachometer verbaut, welcher nicht nur die aktuelle Geschwindigkeit in Meilen pro Stunde anzeigen kann, sondern auch die Tagesstrecke sowie die Gesamtstrecke misst. Leider funktioniert dieser Tachometer nicht, da das Zahnrad, welches die Drehung des Rades abgreift, nicht mit dem Rad in Eingriff ist. Dies ist in Abbildung \ref{fig:Zahnrad} gezeigt:

\color{black}