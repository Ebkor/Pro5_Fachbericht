\chapter{Schlusswort}
\color{red}
Während dem Projekt konnten die Arbeiten am Fahrzeug nicht beendet werden. Dies lag zum einen daran, dass die handwerklichen Arbeiten zu Beginn unterschätzt wurden und so viel Zeit dafür benötigt wurde. Zum anderen war aber auch die Suche sowie die Lieferung des Zubehörs zur Batterie langwierig.
\color{black}

\color{blue}
Sämtliche Arbeiten am Fahrzeug konnten beendet werden. Das Fahrzeug wurde in einem betriebssicheren und fahrtüchtigen Zustand dem Auftraggeber übergeben. Zum einen konnten sämtliche theoretischen Überlegungen auf Blatt gebracht werden und zum anderen wurden eben diese Theorie direkt am Fahrzeug umgesetzt.
\color{black}

Die Funktion des Fahrzeuges konnte gezeigt werden. Die verwendete Schaltung des Reihenschlussmotores kann, insbesondere für diese Zeit, als durchaus elegante Lösung angesehen werden. So wird, unabhängig von der Fahrstufe, stets nur wenig Energie in Widerständen in Abwärme umgewandelt. Auch stehen wie bei modernen Fahrzeugen bei niedrigeren Geschwindigkeiten grössere Kräfte zur Verfügung.

Am Fahrzeug wurden einige Anpassungen durchgeführt. So wurden sämtliche Schaltkontakte abgeschliffen, um eine korrekte Funktion zu garantieren. Ebenfalls wurden neue Leitungen verlegt. Dabei wurde darauf geachtet, dass die originale Schaltung nicht verändert wird, sondern lediglich die nötigen Anpassungen zur Batterie vorgenommen werden.

\color{red}
Die Batterie konnte soweit vorbereitet werden, dass sie zum Fahrzeug passt. Das bedeutet, dass die Verschaltung inklusive dem Schutz theoretisch aufgebaut wurde, sowie die Komponenten dazu bestellt sind. Auch sind bereits sämtliche Parameter für das Batteriemanagementsystem bestimmt. Um die Batterie und deren Schutzschaltung dem Fahrzeug anzupassen, wurde ausserdem eine Steuerplatine erstellt, die die beiden Batterien auf das Fahrzeug abstimmt.

\color{blue}
Die Batterie konnte mit kleinen Schwierigkeiten ins Fahrzeug eingebaut werden. Die beiden Edelstahlgehäuse waren zu gross und passten nicht ohne einige Anpassungen an der Holzkarosserie in den Heckraum. Schlussendlich konnte dieses Problem jedoch behoben werden. Die Parameter des Batteriemanagementsystems wurden nach diversen Praxistests angepasst. Die erstellte Steuerplatine funktioniert nicht einwandfrei und sollte zu einem späteren Zeitpunkt angepasst werden.

\color{red}
Um die Arbeiten am Fahrzeug zu beenden sind vor allem noch handwerkliche Arbeiten und Tests nötig. So müssen die beiden Batterien aufgebaut, in das Fahrzeug eingebaut und mit diesem verbunden werden. Dem schliesst sich eine Testphase an, in welcher das Zusammenspiel zwischen Batterie und Fahrzeug überprüft wird.
\color{black}