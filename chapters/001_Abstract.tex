\chapter*{Abstract}\color{blue}
Elektrische Autos sind keine neuartige Erscheinung des 21. Jahrhunderts. Bereits vor einhundert Jahren waren die Vorteile elektrischer Fahrzeuge bekannt und diese deswegen verbreitet. An der Fachhochschule Nordwestschweiz wurde die Schaltung eines solchen, genau einhundert jährigen Autos -- ein \textsc{Detroit Electric Car} -- untersucht. Gleichzeitig wurde eine Umrüstung auf modernere Lithium-Ionen-Akkumulatoren aus einem Unfallfahrzeug vorgenommen.

Um die moderne Batterie mit dem alten Fahrzeug zu kombinieren, wurden umfangreiche Anpassungen vorgenommen. Beispielsweise konnte das Batteriemanagementsysten, welches die korrekte Funktion der Batterie überwacht und sicherstellt, nicht aus dem Unfallfahrzeug übernommen werden. Weitere Anpassungen waren nötig, damit die zwei getrennten Batterien weiterhin einzeln verschaltet werden konnten. Mittels Testfahrten konnte die Fahrtauglichkeit des umgebauten Fahrzeuges sichergestellt werden.\color{black}

\textbf{Keywords}: \textsc{Detroit Electric Car}, electric vehicle, battery, antique car