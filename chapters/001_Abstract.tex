\chapter*{Abstract}
Elektrische Autos sind keine neuartige Erscheinung des einundzwanzigsten Jahrhunderts. Bereits vor einhundert Jahren waren die Vorteile elektrischer Fahrzeuge bekannt und diese deswegen verbreitet. An der Fachhochschule Nordwestschweiz wurde die Schaltung eines solchen, beinahe einhundert jährigen Autos - ein \textsc{Detroit Electric Car} - untersucht, wobei gleichzeitig die Anpassung an moderne Lithium-Ionen-Akkumulatoren vorgenommen wurde.

Um die moderne Batterie mit dem alten Fahrzeug zu kombinieren, mussten umfangreiche Anpassungen vorgenommen werden. Beispielsweise konnte das Batteriemanagementsysten, welches die korrekte Funktion der Batterie überwacht und sicherstellt, nicht aus dem Unfallfahrzeug übernommen werden. Weitere Anpassungen waren nötig, um die für das Fahrzeug nötigen zwei getrennten Batterien verschalten zu können.\color{blue} Zusätzlich wurde eine Anleitung verfasst, um die spätere Reparatur eines solchen Oldtimers sicherzustellen.
\color{black}

\textbf{Keywords}: \textsc{Detroit Electric Car}, electric vehicle, battery, antique car