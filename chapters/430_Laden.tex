\section{Lademodi}
Der Peugeot konnte auf zwei verschiedene Arten geladen werden. Zum einen besitzt er ein internes Ladegerät. Mit einem Hilfskabel kann also die Verbindung zum $230$ VAC-Netz hergestellt werden, dessen Spannung im Peugeot selbst auf die benötigte Gleichspannung umgesetzt wird. Alternativ kann unter Umgehung des internen Gleichrichters direkt mit Gleichstrom geladen werden, was höhere Ladeleistungen ermöglicht.

\paragraph{Laden am $230$ VAC-Netz}
Zum Anschluss an Wechselspannungsnetz ist ein Hilfskabel nötig. Dieses Hilfskabel bildet zum einen den Übergang zwischen den verschiedenen Steckersystemen, zum anderen ist jedoch auch in das Hilfskabel eingebaut eine elektrische Schaltung. Diese findet ihren Platz in einer Box in der Mitte des Kabels. Diese Schaltung übernimmt dabei sowohl Schutzaufgaben (Überspannungs- und Überstromschutz), als auch die Kommunikation mit dem Fahrzeug. So teilt die Schaltung dem Fahrzeug den maximal beziehbaren Strom mit. Ausserdem sind in dieser Box mehrere Leuchtdioden verbaut, die den Zustand des Fahrzeuges sowie eventuelle Fehler anzeigen. Dieses Hilfskabel mit Box ist in Abbildung \todo{Bild Hilfskabel} zu sehen:

