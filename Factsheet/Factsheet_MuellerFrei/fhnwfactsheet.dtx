% \iffalse meta-comment
% -------------------------------------------------------- %
% Copyright (c) 2017 Raphael Frey rmfrey@runbox.com        %
% -------------------------------------------------------- %
%                                                          %
% This file  may be distributed and/or  modified under the %
% conditions of  the LaTeX Project Public  License, either %
% version 1.3c  of this  license or  (at your  option) any %
% later version. The latest version of this license is in: %
%                                                          %
%     http://www.latex-project.org/lppl.txt                %
%                                                          %
% -------------------------------------------------------- %
% \fi
%
%\CheckSum{615}
%
%
%^^A ------------------------------------------------------------------------- %
%^^A DRIVER, PROVIDESCLASS ETC. <<<
% \iffalse
%<*driver>
\ProvidesFile{fhnwfactsheet.dtx}
%</driver>
%<class>\NeedsTeXFormat{LaTeX2e}
%<class>\ProvidesClass{fhnwfactsheet}%
%<*class>
    [2017/07/12 v1.0.0 PDF-LaTeX class for FHNW fact sheets]
%</class>
%<*driver>
\expandafter\ifx\csname ifenglish\endcsname\relax
    \expandafter\newif\csname ifenglish\endcsname
    \englishtrue
\fi
\documentclass[a4paper,11pt,oneside]{ltxdoc}
\EnableCrossrefs
\CodelineIndex
\RecordChanges
% ----------------------------------------------- PACKAGES %
\ifenglish
    \usepackage[english]{babel}
\else
    \usepackage[ngerman]{babel}
\fi
\usepackage{xcolor-solarized}
\usepackage[%
    bookmarksnumbered=true,
    colorlinks=true,
    linkcolor=cyan!50!blue,
    citecolor=violet,
    urlcolor=purple]{hyperref}
\usepackage[light,nott]{kpfonts}
\usepackage[scaled=0.85]{DejaVuSansMono}
\usepackage{dirtree}
% ------------------------------------------ HELPER MACROS %
\def\pacname#1{\textsf{#1}}
\def\fhnwfs{\pacname{fhnwfactsheet}}
\def\fhnwenfull{University of Applied Sciences and Arts Northwestern Switzerland}
\def\fhnwdefull{Fachhochschule Nordwestschweiz}
\def\fhnw{FHNW}
\def\hten{School of Engineering}
\def\htde{Hochschule f\"ur Technik}
\def\code#1{\texttt{#1}}
\begin{document} % ****************************** DOCUMENT %
    \DocInput{fhnwfactsheet.dtx}
\end{document}
%</driver>
% \fi
%^^A >>>
%
%^^A ------------------------------------------------------------------------- %
%^^A TITLE, ABSTRACT, ToC <<<
% \GetFileInfo{fhnwfactsheet.dtx}
% \author{Raphael Frey \\[0.5em] \href{https://github.com/alpenwasser/fhnwfactsheet/}%
%                                     {\nolinkurl{https://github.com/alpenwasser/fhnwfactsheet/}}}
%
% \ifenglish
%   \title{The \fhnwfs\ Class\thanks{This document corresponds
%       to \fhnwfs~\fileversion, dated~\filedate.}}
%   \maketitle
%   \begin{abstract}
%        The  \fhnwfs\  class  offers  a   simple  interface  for  composing  a
%        fact  sheet for  projects  in A4  paper size  for  the Degree  Program
%        \emph{Electrical  and Information  Technology}  at the  \hten\ of  the
%        \fhnwenfull.
%
%        This  document  contains  the  documentation for  user  interface  and
%        describes the implementation  for any developer who  intends to modify
%        the class.
%   \end{abstract}
% \else
%   \title{Die \fhnwfs-Klasse\thanks{Dieses Dokument entspricht
%          \fhnwfs~\fileversion, datiert~\filedate.}}
%   \maketitle
%   \begin{abstract}
%        Die  \fhnwfs\ Klasse  dient zum  Erstellen von  Factsheets  f\"ur den
%        Studiengang \emph{Elektro- und Informationstechnik}  an der \htde\ der
%        \fhnwdefull.
%
%        Dieses    Dokument    enth\"alt    die   Dokumentation    f\"ur    das
%        Benutzer-Interface  und   beschreibt  die  Implementation   f\"ur  die
%        geneigte Engwicklerin, welche die Klasse modifizieren m\"ochte.
%   \end{abstract}
% \fi
%
% \tableofcontents
%^^A >>>
%
%^^A ------------------------------------------------------------------------- %
%^^A Basic Concept <<<
% \ifenglish
%    \section{Basic Concept}
%    \label{sec:concept}
%    The class  allows to build indivitual  standalone fact sheets as  well as
%    a  larger  document containing  multiple  fact  sheets as  a  collection,
%    including a title page, a table of contents etc. The underlying structure
%    in both scenarios is identical:
%    \dirtree{%^^A
%        .1 Master Document.
%        .2 Content Document.
%        .3 Further Ressources (e.g.~images).
%    }
%    \emph{Note:} 
%    \begin{itemize}
%        \item
%        \textsf{fhnwfactsheet} does not enforce any specifif file name
%        specification. Follow your teacher's instructions if you are creating a fact
%        sheet for submission to the school.
%        \item
%        Do not load additional packages! This will mess up the process of merging your
%        standalone fact sheet with others.
%        \textsf{fhnwfactsheet}loads most classes one would reasonably need to create
%        a fact sheet. If there really is a package which you absolutely, positively
%        need, contact the package maintainer and he will see if it can be integrated
%        into the class. This will then allow to push the class to everyone through an
%        update and things should (hopefully) not break.
%    \end{itemize}
%
%    \subsection{Master Document}
%    \label{subsec:masterdoc}
%    The master document is the  document on which you run \verb|pdflatex|. It
%    sets up the class and its  configuration and then loads the \verb|Content Document|.
%    If you are working on a  standalone fact sheet, this file will
%    be very minimal:
% \else
%    \section{Grundkonzept}
%    \label{sec:concept}
%    Die  Klasse   erlaubt  sowohl  das  Erstellen   von  einzelnen  Factsheets
%    als  eigenst\"andige  Dokumente  wie  auch das  Erstellen  einer  Sammlung
%    von   Factsheets   in   einerm  gr\"osseren   Dokument   mit   Titelseite,
%    Inhaltsverzeichnis etc. Die grundlegende Struktur ist dabeit identisch:
%    \dirtree{%^^A
%        .1 Master Document.
%        .2 Content Document.
%        .3 Weitere Ressourcen (z.B.~Bilder).
%    }
%
%    \emph{Beachte:} 
%    \begin{itemize}
%        \item
%        \textsf{fhnwfactsheet} erzwingt keine spezielle Dateinamen
%        oder Ordnerstruktur. Man beachte die Anweisung der Lehrpersonals diesbez\"uglich
%        wenn man ein Factsheet zur Einreichung bei der Schule erstellt.
%        \item
%        Es sollten keine zus\"atzlichen Packages geladen werden. Dies wird den Prozess
%        des Zusammenstellens verschiedener eigenst\"andiger Factsheets st\"oren.
%        \textsf{fhnwfactsheet} l\"adt die meisten Pakete, welche man \"ublicherweise
%        zur Erstellung eines Factsheets ben\"otigt. Falls wirklich ein Paket nicht
%        geladen ist, welches absolute, definitiv, sicherlich ben\"otigt wird, kontaktiere
%        man den Package-Maintainer (siehe erste Seite). Er wird sehen, ob sich das
%        zus\"atzliche Paket in die Klasse integrieren l\"asst, und falls ja, wird
%        das entsprechende Update zur Verf\"ugung stellen. Dies
%        wird hoffentlich einen problemlosen Arbeitsprozess f\"ur alle beteiligten
%        Parteien erm\"oglichen.
%    \end{itemize}
%
%    \subsection{Master Document}
%    \label{subsec:masterdoc}
%    Das  Master-Dokument  ist  das Dokument,  welches  durch  \verb|pdflatex|
%    verarbeitet  wird. Es  richtet  die  Klasse  und  Konfiguration  ein  und
%    l\"adt  anschliessend  den  eigentlichen  Inhalt  aus  dem  \verb|Content Document|.
%    Wenn ein eigenst\"andiges Factsheet  erstellt wird, ist dieses
%    Dokument ziemlich minimalistisch:
% \fi
%\begin{verbatim}
%\documentclass[german]{fhnwfactsheet}
%\begin{document}
%\includefactsheet{contentDocument.tex}
%\end{document}
%\end{verbatim}
% \ifenglish
%     Multiple \verb|Content Document|s can be included if you wish to create a collection
%     of fact sheets. Each fact sheet will be put on a single page. This also enables
%     the creation of a table of contents.
% \else
%    Mehrere \verb|Content Document|s k\"onnen geladen werden, wenn man eine Sammlung von
%    Factsheets erstellen m\"ochte. Jedes Factsheet wird auf einer einzelnen Seite
%    erstellt werden. Ein Inhaltsverzeichnis kann ebenfalls erstellt werden.
% \fi
%\begin{verbatim}
%\documentclass[german]{fhnwfactsheet}
%\begin{document}
%\pagenumbering{roman}
%\tableofcontents
%
%\newpage\pagenumbering{arabic}
%
%\includefactsheet{factsheet1Content.tex}
%\includefactsheet{factsheet2Content.tex}
%\includefactsheet{factsheet3Content.tex}
%\includefactsheet{factsheet4Content.tex}
%\end{document}
%\end{verbatim}
% \ifenglish
%    \subsection{Content Document}
%    \label{subsec:contentdoc}
%    The content document is where the content of your fact sheet actually resides.
%    it offers some specifics commands to enable consistent layout and typography
%    and places some restrictions which should prevent things from getting too
%    extravagant. See the \verb|examples| directory for some example code.
% \else
%    \subsection{Content Document}
%    \label{subsec:contentdoc}
%    Das \verb|Content Document| ist das Dokument, wo der eigentliche Inhalt
%    untergebracht ist. Es offeriert einige spezifische Befehle, um konsistentes
%    Layout und Typographie zu erm\"oglichen und setzt einige Restriktionen, um
%    allzu extravagante Ideen zu verhindern (oder zumindest zu erschweren).
%    Beispiele k\"onnen im Verzeichnis  \verb|examples| gefunden werden.
% \fi
%^^A >>>
%^^A ------------------------------------------------------------------------- %
%^^A INSTALLATION <<<
% \section{Installation}
% \label{sec:installation}
%
% \ifenglish
%    From \verb|fhnwfactsheet.dtx| and \verb|fhnwfactsheet.ins|, generate the class file \verb|fhnwfactsheet.cls|
%    by running
%\begin{verbatim}
%pdflatex fhnwfactsheet.ins
%\end{verbatim}
%    This will create the class file. If you wish to compile the documentation
%    (this document), run:
%\begin{verbatim}
%pdflatex fhnwfactsheet.dtx
%\end{verbatim}
%    After that, place \verb|fhnwfactsheet.cls| in your working directory and
%    enable it via the \verb|documentclass| command (see example for master document
%    above).
%
%    ^^A\inputminted[linenos=true,numbersep=4pt,fontsize=\footnotesize]{tex}{template.tex}
%   ^^A TODO: File structure tree with the needed files.
% \else
%    Aus \verb|fhnwfactsheet.dtx| und \verb|fhnwfactsheet.ins| kann das Klassen-Dokument
%    \verb|fhnwfactsheet.cls| generiert werden via:
%\begin{verbatim}
%pdflatex fhnwfactsheet.ins
%\end{verbatim}
%    Zum Kompilieren der Dokumentation (dieses Dokument):
%\begin{verbatim}
%pdflatex fhnwfactsheet.dtx
%\end{verbatim}
%    Anschliessend \verb|fhnwfactsheet.cls| in das Verzeichnis des Factsheet-Projekts
%    kopieren und via \verb|documentclass| laden (siehe Master Document oben).
% \fi
%^^A >>>
%
%^^A CLASS OPTIONS <<<
%
% \ifenglish
%    \section{Class Options}
%    \label{sec:options}
%
%    \DescribeMacro{english}
%    \DescribeMacro{german}
%    \DescribeMacro{deutsch}
%    These class options will load the corresponding \textsf{babel} language
%    as well as set some key words accordingly. If you don't specify any
%    language, the class defaults to German.
%
%    \noindent Examples:
% \else
%    \section{Klassen-Optionen}
%    \label{sec:options}
%
%    \DescribeMacro{english}
%    \DescribeMacro{german}
%    \DescribeMacro{deutsch}
%    Diese Klassne-Optionen laden die entsprechenden Konfigurationen in \textsf{babel}
%    und definieren einige Schl\"usselw\"orter in der jeweiligen Sprache. Wenn keine
%    Sprache spezifiziert wird, l\"adt die Klasse auf deutsch.
%    
%    \noindent Beispiele:
% \fi
%\begin{verbatim}
%\documentclass{fhnwfactsheet}          % German
%\documentclass[deutsch]{fhnwfactsheet} % German
%\documentclass[german]{fhnwfactsheet}  % German
%\documentclass[english]{fhnwfactsheet} % English
%\end{verbatim}
%^^A >>>

%^^A COMMANDS <<<
%
% \ifenglish
%    \section{Commands}
%    \label{sec:commands}
%
%    This section describes the commands which are implemented by \textsf{fhnwfactsheet}.
%
%    \subsection{Language-Specific Macros}
%    \label{subsec:language-specific}
%    These macros are used to name the people involved in the project as well as their
%    roles. None of these have to be specified, if they did not exist. They will be used
%    when defined, otherwise not.
%
%    \DescribeMacro{\team}
%    Comma-separated list of team members. \emph{Note:} Use \verb|Firstname Lastname|
%    notation to ensure consistency. Example:
%\begin{verbatim}
%\team{%
%    Team Leader,
%    Personwho Didallthework,
%    Personwho Wasalsothere,
%    Dudewho Wasnotthere}
%\end{verbatim}
%    \DescribeMacro{\client}
%    The client. The format is identical to the \verb|\team| command.
%
%    \DescribeMacro{\coach}
%    \DescribeMacro{\coaches}
%    The Project coaches. In case of a single person, use \verb|\coach|, in
%    case of multiple people, use \verb|\coaches|. \emph{Note:} Since these
%    options are mutually exclusive, the class will raise an error if both
%    macros have been specified.
%
%    \DescribeMacro{\expert}
%    \DescribeMacro{\experts}
%    Experts, for example in case of a thesis. Same as above, the class will
%    raise an error in case of both singular and plural form being specified.
%
%    \DescribeMacro{\partner}
%    \DescribeMacro{\partners}
%    Industry partners, if there were any (and they can be named).
%
%    \subsection{Content Generation}
%    \label{subsec:content-generation}
%    The following macros are used to create the content which goes onto the 
%    fact sheet page.
%    \DescribeMacro{\title}
%    The title for the page. This will also be used as the title in the table
%    of contents in case of a collection document (internally it is automagically
%    mapped to a \verb|\chapter|).
%\begin{verbatim}
%\title{Grand Title of Significant Magnitude}
%\end{verbatim}
%    \emph{Note:} The title will never wrap onto a new line. If it is too long for
%    a single line, it will simply run off the page. This is by design. Shorten
%    your title in that case.
%
%    \DescribeMacro{\fssummary}
%    This is where the introductory header text goes. Ideally, it should not be longer
%    than three or maximum four lines. However, the macro itself does not enforce
%    this. Example\footnote{Goethe's Faust -- Nacht; this will result in 3.5 lines.}
%    Hier kommt der Einleitungstext hinein. Idealerweise sollte die Einleitung nicht
%    l\"anger als drei bis maximal vier Zeilen sein. Das Macro erzwingt diese L\"ange
%    allerdings nicht. Beispiel
%\begin{verbatim}
%\fssummary{%
%    Habe nun, ach! Philosophie,
%    Juristerei und Medizin,
%    Und leider auch Theologie
%    Durchaus studiert, mit heißem Bemüh'n.
%    Da steh ich nun, ich armer Tor!
%    Und bin so klug als wie zuvor;
%    Heiße Magister, heiße Doktor gar,
%    Und ziehe schon an die zehen Jahr
%    Herauf, herab und quer und krumm
%    Meine Schüler an der Nase herum.}
%\end{verbatim}
%
%    \DescribeMacro{\fsgraphics}
%    This command creates a box which is put between \verb|\fssummary| and \verb|\fscontent|
%    with the intent of being filled with graphical content. There are not really any
%    restrictions placed on the graphics, so treat with care. Usually, one would probably
%    want to use \verb|\includegraphics| here.

%    For including multiple graphics, the use of \verb|minipage|s is advised.
%    \footnote{More about minipages and their sometimes surprising behavior:\newline
%        \href{https://github.com/alpenwasser/TeX/tree/master/minipages}.
%             {\nolinkurl{https://github.com/alpenwasser/TeX/tree/master/minipages}}}.
%
%    \DescribeMacro{\graphicscaption}
%    The caption for a picture. Can be used multiple times in case of multiple pictures.
%    Should be kept short.
%
%    \DescribeMacro{\graphicssource}
%    A source for the pictures. Is placed vertically in the right page margin. This, too
%    should be kept short (see example below).
%
%    An example for including two images along the horizonal axis:
%\begin{verbatim}
%\fsgraphics{%
%    \begin{minipage}{0.4\textwidth}
%        \includegraphics[height=60mm]{images/bild0.png}
%        \graphicscaption{Dies ist ein Bild}
%    \end{minipage}%
%    \begin{minipage}{0.6\textwidth}
%        \includegraphics[height=60mm]{images/bild1.png}
%        \graphicscaption{Dies ist ein anderes Bild}
%    \end{minipage}
%    \graphicssource{Wikipedia}}
%\end{verbatim}
%
%    \DescribeMacro{\fscontent}
%    The main content of your document, split into three columns. Partition via
%    the \verb|\section| command as usual (there are no \verb|subsection|s or 
%    even \verb|\subsubsection|s available).
%
%    You can also include graphics here, should you feel so inclined (see the
%    separate \emph{Sonart} example fact sheet). Example:
%  
%\begin{verbatim}    
%\section{Faust -- Nacht}
%\label{sec:faust-night}
%Und sehe, daß wir nichts wissen können!
%Das will mir schier das Herz verbrennen.
%Zwar bin ich gescheiter als all die Laffen,
%Doktoren, Magister, Schreiber und Pfaffen;
%Mich plagen keine Skrupel noch Zweifel,
%Fürchte mich weder vor Hölle noch Teufel -
%Dafür ist mir auch alle Freud´ entrissen,
%Bilde mir nicht ein, was Recht's zu wissen,
%Bilde mir nicht ein, ich könnte was lehren,
%Die Menschen zu bessern und zu bekehren.
%
%\section{Faust -- Mehr Nacht}
%\label{sec:faust-more-night}
%Auch hab´ ich weder Gut noch Geld,
%Noch Ehr´ und Herrlichkeit der Welt;
%Es möchte kein Hund so länger leben!
%Drum hab´ ich mich der Magie ergeben,
%Ob mir durch Geistes Kraft und Mund
%Nicht manch Geheimnis würde kund;
%Daß ich nicht mehr mit saurem Schweiß
%Zu sagen brauche, was ich nicht weiß;
%Daß ich erkenne, was die Welt
%Im Innersten zusammenhält,
%Schau´ alle Wirkenskraft und Samen,
%Und tu´ nicht mehr in Worten kramen.
%\end{verbatim}    
%    \DescribeMacro{\infobox}
%    A box which gets placed on the left side at the page bottom. Can
%    fundamentally contain pretty much any content you like. Well-suited
%    for highlighing key points or specifications about the project.
%    Example:
%\begin{verbatim}
%\infobox{Highlights}{%
%    \footnotesize
%    \setlength\tabcolsep{2pt} % make table a bit more compact horizontally
%    \begin{tabular}{ll}
%    Category 1 & Property A \\
%               & Property B \\
%               & Property C \\
%               & Property D \\
%    Category 2 & Property E \\
%               & Property F \\
%               & Property G \\
%               & Property H \\
%    \end{tabular}}
%\end{verbatim}
%
%    \DescribeMacro{\includefactsheet}
%    Loads a \verb|Content Document| in a \verb|Master Document| (see
%    Section~\ref{sec:concept}).
%
% \else
%    \section{Befehle}
%    \label{sec:commands}
%    Dieser Abschnitt beschreibt die Macros, welche von \textsf{fhnwfactsheet} implementiert werden.
%
%    \subsection{Sprachspezifische Macros}
%    \label{subsec:language-specific}
%    Diese Macros werden dazu verwendet,  die im Project involvierten Personen
%    und  ihre  Rollen zu  beschreiben. Rollen,  welche  nicht besetzt  waren,
%    k\"onnen weggelassen  werden. Keiner dieser Befehle muss  benutzt werden,
%    damit  das  Dokument kompiliert  werden  kann. Sie  werden benutzt,  wenn
%    definiert, andernfalls nicht.
%
%    \DescribeMacro{\team}
%    Komma-getrennte   Liste  der   Gruppenmitglieder.  \emph{Beachte:} Zwecks
%    Einhaltung     von      nsistenz     benutze      man     \verb|Vorname Nachname|-Notation. 
%    Beispiel:
%\begin{verbatim}
%\team{%
%    Team Leader,
%    Personwho Didallthework,
%    Personwho Wasalsothere,
%    Dudewho Wasnotthere}
%\end{verbatim}
%    \DescribeMacro{\client}
%    Auftraggeber. Das Format ist identisch zum \verb|\team|-Befehl.
%
%    \DescribeMacro{\coach}
%    \DescribeMacro{\coaches}
%    Die Fachcoaches. Im Falle einer Einzelperson benutze man \verb|\coach|,
%    im Falle von mehreren Personen \verb|\coaches|. \emph{Beachte:} Da diese
%    beiden Varianten sich gegenseitig ausschliessen, wird die Klasse einen
%    Fehler ausl\"osen, wenn beide spezifiziert worden sind.
%    
%    \DescribeMacro{\expert}
%    \DescribeMacro{\experts}
%    Allf\"allig beigezogene Experten (z.B. im Falle einer Thesis.) Auch hier
%    gilt die gleiche Regel betreffend Singular/Plural und Fehler-Ausl\"osung.
%
%    \DescribeMacro{\partner}
%    \DescribeMacro{\partners}
%    Industriepartner, falls vorhanden und falls sie genannt werden d\"urfen.
%
%    \subsection{Dokumenten-Inhalt}
%    \label{subsec:content-generation}
%    \DescribeMacro{\title}
%    Der Titel der Seite. Dieser Text wird auch f\"ur das Inhaltsverzeichnis verwendet
%    im Falle eines Sammlungs-Dokuments (intern wird er automagisch in ein \verb|\chapter|
%    gespeichert).
%\begin{verbatim}
%\title{Grandioser Title von einiger Wichtigkeit}
%\end{verbatim}
%    \emph{Beachte:}  Der Titel wird niemals einen Zeilenumbruch machen. Falls er
%    zu lange f\"ur eine einzelne Zeile ist, wird er einfach in den rechten Seitenrand
%    laufen. Dies ist Absicht. Man k\"urze in diesem Falle den Titel entsprechend.
%
%    \DescribeMacro{\fssummary}
%    Hier kommt der Einleitungstext hinein. Idealerweise sollte die Einleitung nicht
%    l\"anger als drei bis maximal vier Zeilen sein. Das Macro erwzingt diese L\"ange
%    allerdings nicht. Beispiel\footnote{Goethe's Faust -- Nacht; dies ergibt 3.5 Zeilen.}:
%\begin{verbatim}
%\fssummary{%
%    Habe nun, ach! Philosophie,
%    Juristerei und Medizin,
%    Und leider auch Theologie
%    Durchaus studiert, mit heissem Bem\"uh'n.
%    Da steh ich nun, ich armer Tor!
%    Und bin so klug als wie zuvor;
%    Heisse Magister, heisse Doktor gar,
%    Und ziehe schon an die zehen Jahr
%    Herauf, herab und quer und krumm
%    Meine Sch\"uler an der Nase herum.}
%\end{verbatim}
% 
%    \DescribeMacro{\fsgraphics}
%    Dieser Befehl wird eine Box zwischen \verb|\fssummary| und \verb|\fscontent|
%    setzen mit allf\"alligen Grafiken. In der Gestaltung des Inhalts von \verb|\fsgraphics|
%    ist man relativ frei. Sinnvollerweise kommt hier \verb|\includegraphics| zum Zug.
%    
%    Will man mehrere Grafiken einbinden und nebeneinander anordnen, kann man dies
%    mit \verb|minipage|s bewerkstelligen (siehe Beispiel weiter unten)
%    \footnote{Mehr zu Minipages und ihr manchmal etwas \"uberraschendes Verhalten:\newline
%        \href{https://github.com/alpenwasser/TeX/tree/master/minipages}.
%             {\nolinkurl{https://github.com/alpenwasser/TeX/tree/master/minipages}}}.
%
%    \DescribeMacro{\graphicscaption}
%    Die Legende f\"ur ein Bild. Kann mehrmals verwendet werden im Falle von mehreren
%    Bildern. Sinnvollerweise kurz halten.
%
%    \DescribeMacro{\graphicssource}
%    Quellenangabe f\"ur Bilder. Wird vertikal im rechten Seitenrand platziert. Auch hier
%    liegt die W\"urze in der K\"urze (siehe Beispiel).
%
%    Ein Beispiel zur Anordnung von zwei Bildern entlang der horizontalen Achse:
%\begin{verbatim}
%\fsgraphics{%
%    \begin{minipage}{0.4\textwidth}
%        \includegraphics[height=60mm]{images/bild0.png}
%        \graphicscaption{Dies ist ein Bild}
%    \end{minipage}%
%    \begin{minipage}{0.6\textwidth}
%        \includegraphics[height=60mm]{images/bild1.png}
%        \graphicscaption{Dies ist ein anderes Bild}
%    \end{minipage}
%    \graphicssource{Wikipedia}}
%\end{verbatim}
%
%    \DescribeMacro{\fscontent}
%    Der Hauptinhalt, aufgeteilt in drei Spalten. Der Text kann wie gehabt via
%    \verb|\section| unterteilt werden (\verb|\subsection| und \verb|\subsubsection|
%    sind jedoch deaktiviert).
% 
%    Auch Bilder k\"onnen eingebunden werden, falls man m\"ochte (siehe das
%    separate Beispiel zu \emph{Sonart}). Beispiel:

%\begin{verbatim}    
%\section{Faust -- Nacht}
%\label{sec:faust-night}
%
%Und sehe, daß wir nichts wissen k\"onnen!
%Das will mir schier das Herz verbrennen.
%Zwar bin ich gescheiter als all die Laffen,
%Doktoren, Magister, Schreiber und Pfaffen;
%Mich plagen keine Skrupel noch Zweifel,
%F\"urchte mich weder vor H\"olle noch Teufel -
%Daf\"ur ist mir auch alle Freud´ entrissen,
%Bilde mir nicht ein, was Recht's zu wissen,
%Bilde mir nicht ein, ich k\"onnte was lehren,
%Die Menschen zu bessern und zu bekehren.
%
%\section{Faust -- Mehr Nacht}
%\label{sec:faust-more-night}
%
%Auch hab´ ich weder Gut noch Geld,
%Noch Ehr´ und Herrlichkeit der Welt;
%Es m\"ochte kein Hund so l\"anger leben!
%Drum hab´ ich mich der Magie ergeben,
%Ob mir durch Geistes Kraft und Mund
%Nicht manch Geheimnis w\"urde kund;
%Dass ich nicht mehr mit saurem Schweiss
%Zu sagen brauche, was ich nicht weiss;
%Dass ich erkenne, was die Welt
%Im Innersten zusammenh\"alt,
%Schau´ alle Wirkenskraft und Samen,
%Und tu´ nicht mehr in Worten kramen.
%\end{verbatim}    
%    \DescribeMacro{\infobox}
%    Eine Box, welche links unten auf der Seite platziert wird. Kann
%    grunds\"atzlich mit mehr oder weniger beliebigem Inhalt gef\"ullt werden.
%    Bietet sich gut an, um besondere Highlights oder Eckdaten hervorzuheben.
%    Beispiel:
%\begin{verbatim}
%\infobox{Highlights}{%
%    \footnotesize
%    \setlength\tabcolsep{2pt} % make table a bit more compact horizontally
%    \begin{tabular}{ll}
%    Category 1 & Property A \\
%               & Property B \\
%               & Property C \\
%               & Property D \\
%    Category 2 & Property E \\
%               & Property F \\
%               & Property G \\
%               & Property H \\
%    \end{tabular}}
%\end{verbatim}
%    \DescribeMacro{\includefactsheet}
%    L\"adt ein \verb|Content Document| in einem \verb|Master Document| (siehe
%    Abschnitt~\ref{sec:concept}).
% \fi
%^^A >>>

%^^A ------------------------------------------------------------------------- %
%^^A EXAMPLES <<<
% \ifenglish\section{Examples}\else\section{Beispiele}\fi
% \label{sec:examples}
% \ifenglish
%    The \verb|examples| directory contains a few examples.
% \else
%    Das Verzeichnis \verb|examples| enth\"alt einige Beispiele.
% \fi
%^^A No options, just a few things to pay attention to: ... list of things
%^^A >>>

%^^A ------------------------------------------------------------------------- %
%^^A STYLE GUIDELINES <<<
%^^A \ifenglish\section{Style Guidelines}\else\section{Stil-Richtlinien}\fi
%^^A \label{sec:examples}
%^^A \ifenglish
%^^A    TODO
%^^A \else
%^^A    TODO
%^^A \fi
%^^A No options, just a few things to pay attention to: ... list of things
%^^A >>>
%
%^^A ------------------------------------------------------------------------- %
%^^A IMPLEMENTATION <<<
% \section{Implementation}
% \label{sec:implementation}
%
% \StopEventually{\PrintIndex}
% \ifenglish
% If you are merely intending to use this class, you probably do not need to read this.
% \else
% Normale Benutzer brauchen diesen Abschnitt vermutlich nicht zu lesen.
% \fi
%^^A Need the class guards to prevent the macro code from being included again 
%^^A in template.tex and the German .tex file
%\iffalse
%<*class> 
%\fi
%    \begin{macrocode}
\newif\if@english
\newif\if@german
\DeclareOption{english}{\@englishtrue\@germanfalse}
\DeclareOption{german}{\@germantrue\@englishfalse}
\DeclareOption{deutsch}{\@germantrue\@englishfalse}
\ExecuteOptions{english,german,deutsch}
\ProcessOptions

% Packages
\LoadClass{report}
\RequirePackage[utf8]{inputenc}
\RequirePackage[T1]{fontenc}
\RequirePackage{graphicx}
\RequirePackage{tikz}
\RequirePackage{rviewport}
\RequirePackage{multicol}
\RequirePackage[inline]{enumitem}
\RequirePackage{geometry}
\RequirePackage[scaled=0.95]{helvet}
\renewcommand\familydefault{\sfdefault}
\RequirePackage{xcolor}
\RequirePackage{rotating}
% Define document colos palette
\definecolor{parties}{HTML}{666666}
\definecolor{title}{HTML}{000066}
\definecolor{summary}{HTML}{666666}
\colorlet{boxbg}{cyan!50!blue!20!white}
\RequirePackage{tcolorbox}
\RequirePackage{booktabs}
\RequirePackage[binary-units=true]{siunitx}
\sisetup{range-phrase=--}
\RequirePackage[%
    hidelinks=true,
    bookmarksnumbered=true]{hyperref}
\if@english
    \RequirePackage[english]{babel}
\fi
\if@german
    \RequirePackage[ngerman]{babel}
\fi

% Page Geometry
\geometry{a4paper,twoside=false}
\newlength{\factpageouter}
\newlength{\factpageinner}
\newlength{\factpagetop}
\newlength{\factpagebottom}
\newlength{\factpagewidth}
\newlength{\factgraphicswidth}
\newlength{\testl}
\setlength{\factpageouter}{25mm}
\setlength{\factpageinner}{25mm}
\setlength{\factpagetop}{8mm}
%    \end{macrocode}
%\iffalse
%</class>
%\fi
% Needed so that page numbers do not slide down too far.
%\iffalse
%<*class> 
%\fi
%    \begin{macrocode}
\setlength{\factpagebottom}{30mm}
%    \end{macrocode}
%\iffalse
%</class>
%\fi
% Rest of page geometry layout:
%\iffalse
%<*class> 
%\fi
%    \begin{macrocode}
\setlength{\factpagewidth}{\paperwidth}
\addtolength{\factpagewidth}{-\factpageinner}
\addtolength{\factpagewidth}{-\factpageouter}
\setlength{\factgraphicswidth}{50mm}
\geometry{outer=\factpageouter,inner=\factpageinner,%
          top=\factpagetop,    bottom=\factpagebottom}

\let\newcol\newpage
%    \end{macrocode}
%\iffalse
%</class>
%\fi
% We put the entire  page inside a minipage environment to  make it not take
% up more space than a single page. We  also set the height of that minipage
% to zero so that  it stays on the first page no matter  how much content is
% put inside it.
%\iffalse
%<*class> 
%\fi
%    \begin{macrocode}
\newenvironment{factpage}
    {\noindent\begin{minipage}[t][0pt]{\factpagewidth}}
    {\end{minipage}}
%    \end{macrocode}
%\iffalse
%</class>
%\fi
% Redefine Section Headings  to not be numbered by default. This  is done by
% adding a * after its sixth argument. The rest of the command is similar to
% what can be found in The LaTeX article.cls file.
%\iffalse
%<*class> 
%\fi
%    \begin{macrocode}
\renewcommand\section{\@startsection {section}{1}{\z@}%
                                     {-2.5ex \@plus -1ex \@minus -.2ex}%
                                     {1ex}%
                                     %{1ex \@plus.2ex}% default
                                     {\large\bfseries\color{title}}*}
%    \end{macrocode}
%\iffalse
%</class>
%\fi
% Having subsections  or even  lower-tiered headings in  a document  such as
% this one would be ridiculous. Remove them to make sure they are not used.
%\iffalse
%<*class> 
%\fi
%    \begin{macrocode}
\let\subsection\@undefined
\let\subsubsection\@undefined
\let\paragraph\@undefined
\let\subparagraph\@undefined
%    \end{macrocode}
%\iffalse
%</class>
%\fi
% Define language-specific terms.
%\iffalse
%<*class> 
%\fi
%    \begin{macrocode}
\iflanguage{english}{%
    \newcommand*{\lang@team}    {\textbf{Project Team: }}
    \newcommand*{\lang@client}  {\textbf{Client: }}
    \newcommand*{\lang@coach}   {\textbf{Coach: }}
    \newcommand*{\lang@coaches} {\textbf{Coaches: }}
    \newcommand*{\lang@expert}  {\textbf{Expert: }}
    \newcommand*{\lang@experts} {\textbf{Experts: }}
    \newcommand*{\lang@partner} {\textbf{Industry Partner: }}
    \newcommand*{\lang@partners}{\textbf{Industry Partners: }}
    \newcommand*{\lang@source}  {\textit{Source: }}
}{}
\iflanguage{ngerman}{%
    \newcommand*{\lang@team}    {\textbf{Arbeitsgruppe: }}
    \newcommand*{\lang@client}  {\textbf{Auftraggeber: }}
    \newcommand*{\lang@coach}   {\textbf{Betreuer: }}
    \newcommand*{\lang@coaches} {\textbf{Betreuer: }}
    \newcommand*{\lang@expert}  {\textbf{Experte: }}
    \newcommand*{\lang@experts} {\textbf{Experten: }}
    \newcommand*{\lang@partner} {\textbf{Industriepartner: }}
    \newcommand*{\lang@partners}{\textbf{Industriepartner: }}
    \newcommand*{\lang@source}  {\textit{Quelle: }}
}{}
%    \end{macrocode}
%\iffalse
%</class>
%\fi
% Save those terms in the appropriate macros.
%\iffalse
%<*class> 
%\fi
%    \begin{macrocode}
\newcommand*{\team}[1]
    {\gdef\@team    {{\footnotesize\color{parties}\lang@team#1}}}
\newcommand*{\client}[1]
    {\gdef\@client  {{\footnotesize\color{parties}\lang@client#1}}}
\newcommand*{\coach}[1]
    {\gdef\@coach   {{\footnotesize\color{parties}\lang@coach#1}}}
\newcommand*{\coaches}[1]
    {\gdef\@coaches {{\footnotesize\color{parties}\lang@coaches#1}}}
\newcommand*{\expert}[1]
    {\gdef\@expert  {{\footnotesize\color{parties}\lang@expert#1}}}
\newcommand*{\experts}[1]
    {\gdef\@experts {{\footnotesize\color{parties}\lang@experts#1}}}
\newcommand*{\partner}[1]
    {\gdef\@partner {{\footnotesize\color{parties}\lang@partner#1}}}
\newcommand*{\partners}[1]
    {\gdef\@partners{{\footnotesize\color{parties}\lang@partners#1}}}
%    \end{macrocode}
%\iffalse
%</class>
%\fi
% Redefine \verb|maketitle| to do our bidding:
%\iffalse
%<*class> 
%\fi
%    \begin{macrocode}
\renewcommand\maketitle{\par
    \begingroup
        {\Huge\color{title}\@title}\par
        \vspace{2em}
        \global\let\maketitle\relax
        \global\let\@maketitle\relax
        \global\let\title\relax
        \global\let\date\relax
        \global\let\@title\@empty
        \global\let\@date\@empty
    \endgroup
}
%    \end{macrocode}
%\iffalse
%</class>
%\fi
% For testing purposes:
%\iffalse
%<*class> 
%\fi
%    \begin{macrocode}
\newlength{\placeholderbase}
\newcommand\placeholderpics{%
    \setlength{\placeholderbase}{48mm}
    \begin{tikzpicture}[x=\placeholderbase,y=\placeholderbase,draw=black!50!white,very thick]
        \draw (0,0) -- (1.618,1);
        \draw (0,1) -- (1.618,0);
        \draw (0,0) -- (1.618,0) -- (1.618,1) -- (0,1) -- cycle;
        \node[draw,fill=white] at (0.809,0.5) {\large Beautiful Picture!};
    \end{tikzpicture}
    \begin{tikzpicture}[x=\placeholderbase,y=\placeholderbase,draw=black!50!white,very thick]
        \draw (0,0) -- (1.618,1);
        \draw (0,1) -- (1.618,0);
        \draw (0,0) -- (1.618,0) -- (1.618,1) -- (0,1) -- cycle;
        \node[draw,fill=white] at (0.809,0.5) {\large Amazing Graphics!};
    \end{tikzpicture}\par
    {\footnotesize Beautiful Picture and Amazing Graphics}
}
%    \end{macrocode}
%\iffalse
%</class>
%\fi
% Store the concents of \verb|\fscontent| in \verb|\fs@content|
%\iffalse
%<*class> 
%\fi
%    \begin{macrocode}
\newcommand\fscontent[1]{\gdef\fs@content{#1}}
%    \end{macrocode}
%\iffalse
%</class>
%\fi
% Print the contents of \verb|\fs@content|
%\iffalse
%<*class> 
%\fi
%    \begin{macrocode}
\newcommand\fs@makecontent{\par
    \vspace{1em}\begin{multicols}{3}
        \ifdef{\fs@content}{\fs@content}\par
    \end{multicols}
}
%    \end{macrocode}
%\iffalse
%</class>
%\fi
% Store the concents of \verb|\fssummary| in \verb|\fs@summary|
%\iffalse
%<*class> 
%\fi
%    \begin{macrocode}
\newcommand\fssummary[1]{\gdef\fs@summary{#1}}
%    \end{macrocode}
%\iffalse
%</class>
%\fi
% Print the contents of \verb|\fs@summary|
%\iffalse
%<*class> 
%\fi
%    \begin{macrocode}
\newcommand\fs@makesummary{\par
    \vspace{0.25em}\noindent\begin{minipage}{\textwidth}\large
        \ifdef{\fs@summary}{\color{summary}\fs@summary}\par
    \end{minipage}%
}
%    \end{macrocode}
%\iffalse
%</class>
%\fi
% Store the concents of \verb|\fsgraphics| in \verb|\fs@graphics|
%\iffalse
%<*class> 
%\fi
%    \begin{macrocode}
\newcommand\fsgraphics[1]{\gdef\fs@graphics{#1}}
\newcommand\graphicssource[1]{\gdef\fs@graphicssource{#1}}
%    \end{macrocode}
%\iffalse
%</class>
%\fi
% Put the contents of \verb|\graphicscaption| inside an \verb|\mbox| so that it does not wrap.
%\iffalse
%<*class> 
%\fi
%    \begin{macrocode}
\newcommand\graphicscaption[1]{\mbox{#1}}
%    \end{macrocode}
%\iffalse
%</class>
%\fi
% Print the contents of \verb|\fs@graphics|
%\iffalse
%<*class> 
%\fi
%    \begin{macrocode}
\newcommand\fs@makegraphics{\par
    \vspace{1em}\noindent\begin{minipage}[b]{\textwidth}%
        \footnotesize\raggedright
        \ifdef{\fs@graphics}{\fs@graphics}\par\par
        \ifdef{\fs@graphicscaption}{\fs@graphicscaption}\par
    \end{minipage}%
    \begin{minipage}[t]{2ex}\footnotesize\color{parties}
        \ifdef{\fs@graphicssource}{%
            \hspace*{2ex}%
            \turnbox{90}{%
                \hspace*{2ex} \lang@source \fs@graphicssource}}\par
    \end{minipage}%
}
%    \end{macrocode}
%\iffalse
%</class>
%\fi
% Store the content of \verb|\infobox| and print it:
%\iffalse
%<*class> 
%\fi
%    \begin{macrocode}
\newcommand\infobox[2]{%
    \gdef\fs@infoboxtitle{#1}%
    \gdef\fs@infoboxcontent{\footnotesize #2}%
}
\newcommand\fs@makeinfobox{
    \begin{tcolorbox}[
            title=\fs@infoboxtitle,
            width=0.66\textwidth,
            arc=0mm,
            before=,
            after=\hfill]
        \fs@infoboxcontent
    \end{tcolorbox}
}
\newcommand\tightlist{%
    \setlength{\itemsep}{0pt}\setlength{\parskip}{0pt}%
}
%    \end{macrocode}
%\iffalse
%</class>
%\fi
% Errors:
%\iffalse
%<*class> 
%\fi
%    \begin{macrocode}
\newcommand\fs@checkinput{%
    \ifdef{\@coach}{%
        \ifdef{\@coaches}{%
            \ClassError{fhnwfactsheet}{Both 'coach' and 'coaches' field defined. Undefine one.}
                {Both the 'coach' and the 'coaches' field are defined. Only one of them should be.}
        }{}%
    }{}%
    \ifdef{\@expert}{%
        \ifdef{\@experts}{%
            \ClassError{fhnwfactsheet}{Both 'expert' and 'experts' field defined. Undefine one.}
                {Both the 'expert' and the 'experts' field are defined. Only one of them should be.}
        }{}%
    }{}%
    \ifdef{\@partner}{%
        \ifdef{\@partners}{%
            \ClassError{fhnwfactsheet}{Both 'partner' and 'partners' field defined. Undefine one.}
                {Both the 'partner' and the 'partners' field are defined. Only one of them should be.}
        }{}%
    }{}%
    \ifdef{\fs@infoboxcontent}{}
        {%
            \ClassError{fhnwfactsheet}{Did you forget to specify your infobox?}
            {The infobox contents are not defined. Make sure your infobox command is properly called.}
        }%
    \ifdef{\fs@infoboxtitle}{}
        {%
            \ClassError{fhnwfactsheet}{Did you forget to specify your infobox?}
            {The infobox title is not defined. Make sure your infobox command is properly called.}
        }%
}
%    \end{macrocode}
%\iffalse
%</class>
%\fi
% Chapter code. Taken mostly from \verb|classes.dtx|.
%\iffalse
%<*class> 
%\fi
%    \begin{macrocode}
\renewcommand\chapter{%\if@openright\cleardoublepage\else\clearpage\fi
                    \thispagestyle{plain}%
                    \global\@topnum\z@
                    \@afterindentfalse
                    \secdef\@chapter\@schapter}
\def\@chapter[#1]#2{\ifnum \c@secnumdepth >\m@ne
                         \refstepcounter{chapter}%
                         \typeout{\@chapapp\space\thechapter.}%
                         \addcontentsline{toc}{chapter}{#1}%
                                   %{\protect\numberline{\thechapter}#1}%
                    \else
                      \addcontentsline{toc}{chapter}{#1}%
                    \fi
                    \chaptermark{#1}%
                    \addtocontents{lof}{\protect\addvspace{10\p@}}%
                    \addtocontents{lot}{\protect\addvspace{10\p@}}%
                    \if@twocolumn
                      \@topnewpage[\@makechapterhead{#2}]%
                    \else
                      \@makechapterhead{#2}%
                      \@afterheading
                    \fi}
\def\@makechapterhead#1{%
    \begingroup
        \mbox{\Huge\color{title}#1}\par
        \vspace{2em}
    \endgroup
}
\def\@makeschapterhead#1{%
  \vspace*{50\p@}%
  {\parindent \z@ \raggedright
    \normalfont
    \interlinepenalty\@M
    \Huge\color{title} #1\par\nobreak
    \vskip 40\p@
  }}
\renewcommand*\l@chapter[2]{%
  \ifnum \c@tocdepth >\m@ne
    \addpenalty{-\@highpenalty}%
    \vskip 1.0em \@plus\p@
    \setlength\@tempdima{1.5em}%
    \begingroup
      \parindent \z@ \rightskip \@pnumwidth
      \parfillskip -\@pnumwidth
      \leavevmode 
      \advance\leftskip\@tempdima
      \hskip -\leftskip
      %\@dottedtocline{0}{0em}{0em}{#1}{#2}\par
      #1\nobreak\hfil \nobreak\hb@xt@\@pnumwidth{\hss #2}\par
      \penalty\@highpenalty
    \endgroup
  \fi}
%    \end{macrocode}
%\iffalse
%</class>
%\fi
% Including a fact sheet content file in a master document:
%\iffalse
%<*class> 
%\fi
%    \begin{macrocode}
\newcommand*\includefactsheet[1]{\input{#1}\makefactsheet\newpage}
%    \end{macrocode}
%\iffalse
%</class>
%\fi
% Output routine:
%\iffalse
%<*class> 
%\fi
%    \begin{macrocode}
\newcommand\makefactsheet{
    \begin{factpage}
        \fs@checkinput
        \chapter{\@title}
        % Must be after \chapter
        % Note: We do not want to overwrite the \chapter pagestyle{plain} everywhere.
        \thispagestyle{empty}
        \fs@makesummary
        \vspace{1ex}
        \fs@makegraphics
        \vspace{1ex}
        \fs@makecontent
        \vspace{1ex}
        \tcbset{
            colback=boxbg,
            colframe=title,
            boxrule=0.5pt,
            toptitle=2pt,
            bottomtitle=1pt,
            titlerule=0.5pt,
        }
        \fs@makeinfobox
        \begin{minipage}[b]{0.30\textwidth}
            \ifdef{\@team}{\@team}{}\par
            \ifdef{\@client}{\@client}{}\par
            \ifdef{\@coach}{\@coach}{}\par
            \ifdef{\@coaches}{\@coaches}{}\par
            \ifdef{\@expert}{\@expert}{}\par
            \ifdef{\@experts}{\@experts}{}\par
            \ifdef{\@partner}{\@partner}{}\par
            \ifdef{\@partners}{\@partners}{}\par
        \end{minipage}
    \end{factpage}\raisebox{6ex}{\hspace{3em}\thepage}
    % Undefine for next fact sheet
    \let\fs@graphics\@undefined
    \let\fs@graphicssource\@undefined
    \let\fs@graphicscaption\@undefined
    \let\@team\@undefined
    \let\@client\@undefined
    \let\@coach\@undefined
    \let\@coaches\@undefined
    \let\@expert\@undefined
    \let\@experts\@undefined
    \let\@partner\@undefined
    \let\@partners\@undefined
    \let\fs@infoboxtitle\@undefined
    \let\fs@infoboxcontent\@undefined
}
%    \end{macrocode}
%\iffalse
%</class>
%\fi
%
% \Finale
%^^A >>>
%
%^^A ------------------------------------------------------------------------- %
%^^A TEMPLATE <<<
% \iffalse
%<*template>
\documentclass{fhnwfactsheet}
\usepackage{lipsum}
%\usepackage{FiraSans}

%% ================================================================= SET TITLE %
\title{Phenomenal Title!}
\author{Glorious Author!}

%% ============================================================== END PREAMBLE %
\begin{document}
\begin{factpage}
%% ============================================================= BEGIN CONTENT %

%% ------------------------------------------------------------------ OVERVIEW %
\begin{overview}
    \lipsum[2]
\end{overview}

\begin{content}

    \section{This is a section}
    \lipsum[2]
    \section{This is another section}
    \lipsum[4]
    \section{And another one!}
    \lipsum[5]

\end{content}

\lipsum[1]

\factgraphics{a-stunning-picture.png}{a-very-nice-graphic.png}

%% ------------------------------------------------------------------ STUDENTS %
\begin{students}
    \lipsum[2]
\end{students}

%% ------------------------------------------------------------------- COACHES %
\begin{coaches}
    \lipsum[2]
\end{coaches}

%% ----------------------------------------------------------------- PARTNERS %
\begin{partners}
    \lipsum[2]
\end{partners}

%% ============================================================== END CONTENT %
\end{factpage}
\end{document}
%</template>
% \fi
%^^A >>>
%
%^^A ------------------------------------------------------------------------- %
%^^A MULTI-LANGUAGE SUPPORT <<<
% \iffalse
%<*de>
\expandafter\newif\csname ifenglish\endcsname
\englishfalse
% \iffalse meta-comment
% -------------------------------------------------------- %
% Copyright (c) 2017 Raphael Frey rmfrey@runbox.com        %
% -------------------------------------------------------- %
%                                                          %
% This file  may be distributed and/or  modified under the %
% conditions of  the LaTeX Project Public  License, either %
% version 1.3c  of this  license or  (at your  option) any %
% later version. The latest version of this license is in: %
%                                                          %
%     http://www.latex-project.org/lppl.txt                %
%                                                          %
% -------------------------------------------------------- %
% \fi
%
%\CheckSum{615}
%
%
%^^A ------------------------------------------------------------------------- %
%^^A DRIVER, PROVIDESCLASS ETC. <<<
% \iffalse
%<*driver>
\ProvidesFile{fhnwfactsheet.dtx}
%</driver>
%<class>\NeedsTeXFormat{LaTeX2e}
%<class>\ProvidesClass{fhnwfactsheet}%
%<*class>
    [2017/07/12 v1.0.0 PDF-LaTeX class for FHNW fact sheets]
%</class>
%<*driver>
\expandafter\ifx\csname ifenglish\endcsname\relax
    \expandafter\newif\csname ifenglish\endcsname
    \englishtrue
\fi
\documentclass[a4paper,11pt,oneside]{ltxdoc}
\EnableCrossrefs
\CodelineIndex
\RecordChanges
% ----------------------------------------------- PACKAGES %
\ifenglish
    \usepackage[english]{babel}
\else
    \usepackage[ngerman]{babel}
\fi
\usepackage{xcolor-solarized}
\usepackage[%
    bookmarksnumbered=true,
    colorlinks=true,
    linkcolor=cyan!50!blue,
    citecolor=violet,
    urlcolor=purple]{hyperref}
\usepackage[light,nott]{kpfonts}
\usepackage[scaled=0.85]{DejaVuSansMono}
\usepackage{dirtree}
% ------------------------------------------ HELPER MACROS %
\def\pacname#1{\textsf{#1}}
\def\fhnwfs{\pacname{fhnwfactsheet}}
\def\fhnwenfull{University of Applied Sciences and Arts Northwestern Switzerland}
\def\fhnwdefull{Fachhochschule Nordwestschweiz}
\def\fhnw{FHNW}
\def\hten{School of Engineering}
\def\htde{Hochschule f\"ur Technik}
\def\code#1{\texttt{#1}}
\begin{document} % ****************************** DOCUMENT %
    \DocInput{fhnwfactsheet.dtx}
\end{document}
%</driver>
% \fi
%^^A >>>
%
%^^A ------------------------------------------------------------------------- %
%^^A TITLE, ABSTRACT, ToC <<<
% \GetFileInfo{fhnwfactsheet.dtx}
% \author{Raphael Frey \\[0.5em] \href{https://github.com/alpenwasser/fhnwfactsheet/}%
%                                     {\nolinkurl{https://github.com/alpenwasser/fhnwfactsheet/}}}
%
% \ifenglish
%   \title{The \fhnwfs\ Class\thanks{This document corresponds
%       to \fhnwfs~\fileversion, dated~\filedate.}}
%   \maketitle
%   \begin{abstract}
%        The  \fhnwfs\  class  offers  a   simple  interface  for  composing  a
%        fact  sheet for  projects  in A4  paper size  for  the Degree  Program
%        \emph{Electrical  and Information  Technology}  at the  \hten\ of  the
%        \fhnwenfull.
%
%        This  document  contains  the  documentation for  user  interface  and
%        describes the implementation  for any developer who  intends to modify
%        the class.
%   \end{abstract}
% \else
%   \title{Die \fhnwfs-Klasse\thanks{Dieses Dokument entspricht
%          \fhnwfs~\fileversion, datiert~\filedate.}}
%   \maketitle
%   \begin{abstract}
%        Die  \fhnwfs\ Klasse  dient zum  Erstellen von  Factsheets  f\"ur den
%        Studiengang \emph{Elektro- und Informationstechnik}  an der \htde\ der
%        \fhnwdefull.
%
%        Dieses    Dokument    enth\"alt    die   Dokumentation    f\"ur    das
%        Benutzer-Interface  und   beschreibt  die  Implementation   f\"ur  die
%        geneigte Engwicklerin, welche die Klasse modifizieren m\"ochte.
%   \end{abstract}
% \fi
%
% \tableofcontents
%^^A >>>
%
%^^A ------------------------------------------------------------------------- %
%^^A Basic Concept <<<
% \ifenglish
%    \section{Basic Concept}
%    \label{sec:concept}
%    The class  allows to build indivitual  standalone fact sheets as  well as
%    a  larger  document containing  multiple  fact  sheets as  a  collection,
%    including a title page, a table of contents etc. The underlying structure
%    in both scenarios is identical:
%    \dirtree{%^^A
%        .1 Master Document.
%        .2 Content Document.
%        .3 Further Ressources (e.g.~images).
%    }
%    \emph{Note:} 
%    \begin{itemize}
%        \item
%        \textsf{fhnwfactsheet} does not enforce any specifif file name
%        specification. Follow your teacher's instructions if you are creating a fact
%        sheet for submission to the school.
%        \item
%        Do not load additional packages! This will mess up the process of merging your
%        standalone fact sheet with others.
%        \textsf{fhnwfactsheet}loads most classes one would reasonably need to create
%        a fact sheet. If there really is a package which you absolutely, positively
%        need, contact the package maintainer and he will see if it can be integrated
%        into the class. This will then allow to push the class to everyone through an
%        update and things should (hopefully) not break.
%    \end{itemize}
%
%    \subsection{Master Document}
%    \label{subsec:masterdoc}
%    The master document is the  document on which you run \verb|pdflatex|. It
%    sets up the class and its  configuration and then loads the \verb|Content Document|.
%    If you are working on a  standalone fact sheet, this file will
%    be very minimal:
% \else
%    \section{Grundkonzept}
%    \label{sec:concept}
%    Die  Klasse   erlaubt  sowohl  das  Erstellen   von  einzelnen  Factsheets
%    als  eigenst\"andige  Dokumente  wie  auch das  Erstellen  einer  Sammlung
%    von   Factsheets   in   einerm  gr\"osseren   Dokument   mit   Titelseite,
%    Inhaltsverzeichnis etc. Die grundlegende Struktur ist dabeit identisch:
%    \dirtree{%^^A
%        .1 Master Document.
%        .2 Content Document.
%        .3 Weitere Ressourcen (z.B.~Bilder).
%    }
%
%    \emph{Beachte:} 
%    \begin{itemize}
%        \item
%        \textsf{fhnwfactsheet} erzwingt keine spezielle Dateinamen
%        oder Ordnerstruktur. Man beachte die Anweisung der Lehrpersonals diesbez\"uglich
%        wenn man ein Factsheet zur Einreichung bei der Schule erstellt.
%        \item
%        Es sollten keine zus\"atzlichen Packages geladen werden. Dies wird den Prozess
%        des Zusammenstellens verschiedener eigenst\"andiger Factsheets st\"oren.
%        \textsf{fhnwfactsheet} l\"adt die meisten Pakete, welche man \"ublicherweise
%        zur Erstellung eines Factsheets ben\"otigt. Falls wirklich ein Paket nicht
%        geladen ist, welches absolute, definitiv, sicherlich ben\"otigt wird, kontaktiere
%        man den Package-Maintainer (siehe erste Seite). Er wird sehen, ob sich das
%        zus\"atzliche Paket in die Klasse integrieren l\"asst, und falls ja, wird
%        das entsprechende Update zur Verf\"ugung stellen. Dies
%        wird hoffentlich einen problemlosen Arbeitsprozess f\"ur alle beteiligten
%        Parteien erm\"oglichen.
%    \end{itemize}
%
%    \subsection{Master Document}
%    \label{subsec:masterdoc}
%    Das  Master-Dokument  ist  das Dokument,  welches  durch  \verb|pdflatex|
%    verarbeitet  wird. Es  richtet  die  Klasse  und  Konfiguration  ein  und
%    l\"adt  anschliessend  den  eigentlichen  Inhalt  aus  dem  \verb|Content Document|.
%    Wenn ein eigenst\"andiges Factsheet  erstellt wird, ist dieses
%    Dokument ziemlich minimalistisch:
% \fi
%\begin{verbatim}
%\documentclass[german]{fhnwfactsheet}
%\begin{document}
%\includefactsheet{contentDocument.tex}
%\end{document}
%\end{verbatim}
% \ifenglish
%     Multiple \verb|Content Document|s can be included if you wish to create a collection
%     of fact sheets. Each fact sheet will be put on a single page. This also enables
%     the creation of a table of contents.
% \else
%    Mehrere \verb|Content Document|s k\"onnen geladen werden, wenn man eine Sammlung von
%    Factsheets erstellen m\"ochte. Jedes Factsheet wird auf einer einzelnen Seite
%    erstellt werden. Ein Inhaltsverzeichnis kann ebenfalls erstellt werden.
% \fi
%\begin{verbatim}
%\documentclass[german]{fhnwfactsheet}
%\begin{document}
%\pagenumbering{roman}
%\tableofcontents
%
%\newpage\pagenumbering{arabic}
%
%\includefactsheet{factsheet1Content.tex}
%\includefactsheet{factsheet2Content.tex}
%\includefactsheet{factsheet3Content.tex}
%\includefactsheet{factsheet4Content.tex}
%\end{document}
%\end{verbatim}
% \ifenglish
%    \subsection{Content Document}
%    \label{subsec:contentdoc}
%    The content document is where the content of your fact sheet actually resides.
%    it offers some specifics commands to enable consistent layout and typography
%    and places some restrictions which should prevent things from getting too
%    extravagant. See the \verb|examples| directory for some example code.
% \else
%    \subsection{Content Document}
%    \label{subsec:contentdoc}
%    Das \verb|Content Document| ist das Dokument, wo der eigentliche Inhalt
%    untergebracht ist. Es offeriert einige spezifische Befehle, um konsistentes
%    Layout und Typographie zu erm\"oglichen und setzt einige Restriktionen, um
%    allzu extravagante Ideen zu verhindern (oder zumindest zu erschweren).
%    Beispiele k\"onnen im Verzeichnis  \verb|examples| gefunden werden.
% \fi
%^^A >>>
%^^A ------------------------------------------------------------------------- %
%^^A INSTALLATION <<<
% \section{Installation}
% \label{sec:installation}
%
% \ifenglish
%    From \verb|fhnwfactsheet.dtx| and \verb|fhnwfactsheet.ins|, generate the class file \verb|fhnwfactsheet.cls|
%    by running
%\begin{verbatim}
%pdflatex fhnwfactsheet.ins
%\end{verbatim}
%    This will create the class file. If you wish to compile the documentation
%    (this document), run:
%\begin{verbatim}
%pdflatex fhnwfactsheet.dtx
%\end{verbatim}
%    After that, place \verb|fhnwfactsheet.cls| in your working directory and
%    enable it via the \verb|documentclass| command (see example for master document
%    above).
%
%    ^^A\inputminted[linenos=true,numbersep=4pt,fontsize=\footnotesize]{tex}{template.tex}
%   ^^A TODO: File structure tree with the needed files.
% \else
%    Aus \verb|fhnwfactsheet.dtx| und \verb|fhnwfactsheet.ins| kann das Klassen-Dokument
%    \verb|fhnwfactsheet.cls| generiert werden via:
%\begin{verbatim}
%pdflatex fhnwfactsheet.ins
%\end{verbatim}
%    Zum Kompilieren der Dokumentation (dieses Dokument):
%\begin{verbatim}
%pdflatex fhnwfactsheet.dtx
%\end{verbatim}
%    Anschliessend \verb|fhnwfactsheet.cls| in das Verzeichnis des Factsheet-Projekts
%    kopieren und via \verb|documentclass| laden (siehe Master Document oben).
% \fi
%^^A >>>
%
%^^A CLASS OPTIONS <<<
%
% \ifenglish
%    \section{Class Options}
%    \label{sec:options}
%
%    \DescribeMacro{english}
%    \DescribeMacro{german}
%    \DescribeMacro{deutsch}
%    These class options will load the corresponding \textsf{babel} language
%    as well as set some key words accordingly. If you don't specify any
%    language, the class defaults to German.
%
%    \noindent Examples:
% \else
%    \section{Klassen-Optionen}
%    \label{sec:options}
%
%    \DescribeMacro{english}
%    \DescribeMacro{german}
%    \DescribeMacro{deutsch}
%    Diese Klassne-Optionen laden die entsprechenden Konfigurationen in \textsf{babel}
%    und definieren einige Schl\"usselw\"orter in der jeweiligen Sprache. Wenn keine
%    Sprache spezifiziert wird, l\"adt die Klasse auf deutsch.
%    
%    \noindent Beispiele:
% \fi
%\begin{verbatim}
%\documentclass{fhnwfactsheet}          % German
%\documentclass[deutsch]{fhnwfactsheet} % German
%\documentclass[german]{fhnwfactsheet}  % German
%\documentclass[english]{fhnwfactsheet} % English
%\end{verbatim}
%^^A >>>

%^^A COMMANDS <<<
%
% \ifenglish
%    \section{Commands}
%    \label{sec:commands}
%
%    This section describes the commands which are implemented by \textsf{fhnwfactsheet}.
%
%    \subsection{Language-Specific Macros}
%    \label{subsec:language-specific}
%    These macros are used to name the people involved in the project as well as their
%    roles. None of these have to be specified, if they did not exist. They will be used
%    when defined, otherwise not.
%
%    \DescribeMacro{\team}
%    Comma-separated list of team members. \emph{Note:} Use \verb|Firstname Lastname|
%    notation to ensure consistency. Example:
%\begin{verbatim}
%\team{%
%    Team Leader,
%    Personwho Didallthework,
%    Personwho Wasalsothere,
%    Dudewho Wasnotthere}
%\end{verbatim}
%    \DescribeMacro{\client}
%    The client. The format is identical to the \verb|\team| command.
%
%    \DescribeMacro{\coach}
%    \DescribeMacro{\coaches}
%    The Project coaches. In case of a single person, use \verb|\coach|, in
%    case of multiple people, use \verb|\coaches|. \emph{Note:} Since these
%    options are mutually exclusive, the class will raise an error if both
%    macros have been specified.
%
%    \DescribeMacro{\expert}
%    \DescribeMacro{\experts}
%    Experts, for example in case of a thesis. Same as above, the class will
%    raise an error in case of both singular and plural form being specified.
%
%    \DescribeMacro{\partner}
%    \DescribeMacro{\partners}
%    Industry partners, if there were any (and they can be named).
%
%    \subsection{Content Generation}
%    \label{subsec:content-generation}
%    The following macros are used to create the content which goes onto the 
%    fact sheet page.
%    \DescribeMacro{\title}
%    The title for the page. This will also be used as the title in the table
%    of contents in case of a collection document (internally it is automagically
%    mapped to a \verb|\chapter|).
%\begin{verbatim}
%\title{Grand Title of Significant Magnitude}
%\end{verbatim}
%    \emph{Note:} The title will never wrap onto a new line. If it is too long for
%    a single line, it will simply run off the page. This is by design. Shorten
%    your title in that case.
%
%    \DescribeMacro{\fssummary}
%    This is where the introductory header text goes. Ideally, it should not be longer
%    than three or maximum four lines. However, the macro itself does not enforce
%    this. Example\footnote{Goethe's Faust -- Nacht; this will result in 3.5 lines.}
%    Hier kommt der Einleitungstext hinein. Idealerweise sollte die Einleitung nicht
%    l\"anger als drei bis maximal vier Zeilen sein. Das Macro erzwingt diese L\"ange
%    allerdings nicht. Beispiel
%\begin{verbatim}
%\fssummary{%
%    Habe nun, ach! Philosophie,
%    Juristerei und Medizin,
%    Und leider auch Theologie
%    Durchaus studiert, mit heißem Bemüh'n.
%    Da steh ich nun, ich armer Tor!
%    Und bin so klug als wie zuvor;
%    Heiße Magister, heiße Doktor gar,
%    Und ziehe schon an die zehen Jahr
%    Herauf, herab und quer und krumm
%    Meine Schüler an der Nase herum.}
%\end{verbatim}
%
%    \DescribeMacro{\fsgraphics}
%    This command creates a box which is put between \verb|\fssummary| and \verb|\fscontent|
%    with the intent of being filled with graphical content. There are not really any
%    restrictions placed on the graphics, so treat with care. Usually, one would probably
%    want to use \verb|\includegraphics| here.

%    For including multiple graphics, the use of \verb|minipage|s is advised.
%    \footnote{More about minipages and their sometimes surprising behavior:\newline
%        \href{https://github.com/alpenwasser/TeX/tree/master/minipages}.
%             {\nolinkurl{https://github.com/alpenwasser/TeX/tree/master/minipages}}}.
%
%    \DescribeMacro{\graphicscaption}
%    The caption for a picture. Can be used multiple times in case of multiple pictures.
%    Should be kept short.
%
%    \DescribeMacro{\graphicssource}
%    A source for the pictures. Is placed vertically in the right page margin. This, too
%    should be kept short (see example below).
%
%    An example for including two images along the horizonal axis:
%\begin{verbatim}
%\fsgraphics{%
%    \begin{minipage}{0.4\textwidth}
%        \includegraphics[height=60mm]{images/bild0.png}
%        \graphicscaption{Dies ist ein Bild}
%    \end{minipage}%
%    \begin{minipage}{0.6\textwidth}
%        \includegraphics[height=60mm]{images/bild1.png}
%        \graphicscaption{Dies ist ein anderes Bild}
%    \end{minipage}
%    \graphicssource{Wikipedia}}
%\end{verbatim}
%
%    \DescribeMacro{\fscontent}
%    The main content of your document, split into three columns. Partition via
%    the \verb|\section| command as usual (there are no \verb|subsection|s or 
%    even \verb|\subsubsection|s available).
%
%    You can also include graphics here, should you feel so inclined (see the
%    separate \emph{Sonart} example fact sheet). Example:
%  
%\begin{verbatim}    
%\section{Faust -- Nacht}
%\label{sec:faust-night}
%Und sehe, daß wir nichts wissen können!
%Das will mir schier das Herz verbrennen.
%Zwar bin ich gescheiter als all die Laffen,
%Doktoren, Magister, Schreiber und Pfaffen;
%Mich plagen keine Skrupel noch Zweifel,
%Fürchte mich weder vor Hölle noch Teufel -
%Dafür ist mir auch alle Freud´ entrissen,
%Bilde mir nicht ein, was Recht's zu wissen,
%Bilde mir nicht ein, ich könnte was lehren,
%Die Menschen zu bessern und zu bekehren.
%
%\section{Faust -- Mehr Nacht}
%\label{sec:faust-more-night}
%Auch hab´ ich weder Gut noch Geld,
%Noch Ehr´ und Herrlichkeit der Welt;
%Es möchte kein Hund so länger leben!
%Drum hab´ ich mich der Magie ergeben,
%Ob mir durch Geistes Kraft und Mund
%Nicht manch Geheimnis würde kund;
%Daß ich nicht mehr mit saurem Schweiß
%Zu sagen brauche, was ich nicht weiß;
%Daß ich erkenne, was die Welt
%Im Innersten zusammenhält,
%Schau´ alle Wirkenskraft und Samen,
%Und tu´ nicht mehr in Worten kramen.
%\end{verbatim}    
%    \DescribeMacro{\infobox}
%    A box which gets placed on the left side at the page bottom. Can
%    fundamentally contain pretty much any content you like. Well-suited
%    for highlighing key points or specifications about the project.
%    Example:
%\begin{verbatim}
%\infobox{Highlights}{%
%    \footnotesize
%    \setlength\tabcolsep{2pt} % make table a bit more compact horizontally
%    \begin{tabular}{ll}
%    Category 1 & Property A \\
%               & Property B \\
%               & Property C \\
%               & Property D \\
%    Category 2 & Property E \\
%               & Property F \\
%               & Property G \\
%               & Property H \\
%    \end{tabular}}
%\end{verbatim}
%
%    \DescribeMacro{\includefactsheet}
%    Loads a \verb|Content Document| in a \verb|Master Document| (see
%    Section~\ref{sec:concept}).
%
% \else
%    \section{Befehle}
%    \label{sec:commands}
%    Dieser Abschnitt beschreibt die Macros, welche von \textsf{fhnwfactsheet} implementiert werden.
%
%    \subsection{Sprachspezifische Macros}
%    \label{subsec:language-specific}
%    Diese Macros werden dazu verwendet,  die im Project involvierten Personen
%    und  ihre  Rollen zu  beschreiben. Rollen,  welche  nicht besetzt  waren,
%    k\"onnen weggelassen  werden. Keiner dieser Befehle muss  benutzt werden,
%    damit  das  Dokument kompiliert  werden  kann. Sie  werden benutzt,  wenn
%    definiert, andernfalls nicht.
%
%    \DescribeMacro{\team}
%    Komma-getrennte   Liste  der   Gruppenmitglieder.  \emph{Beachte:} Zwecks
%    Einhaltung     von      nsistenz     benutze      man     \verb|Vorname Nachname|-Notation. 
%    Beispiel:
%\begin{verbatim}
%\team{%
%    Team Leader,
%    Personwho Didallthework,
%    Personwho Wasalsothere,
%    Dudewho Wasnotthere}
%\end{verbatim}
%    \DescribeMacro{\client}
%    Auftraggeber. Das Format ist identisch zum \verb|\team|-Befehl.
%
%    \DescribeMacro{\coach}
%    \DescribeMacro{\coaches}
%    Die Fachcoaches. Im Falle einer Einzelperson benutze man \verb|\coach|,
%    im Falle von mehreren Personen \verb|\coaches|. \emph{Beachte:} Da diese
%    beiden Varianten sich gegenseitig ausschliessen, wird die Klasse einen
%    Fehler ausl\"osen, wenn beide spezifiziert worden sind.
%    
%    \DescribeMacro{\expert}
%    \DescribeMacro{\experts}
%    Allf\"allig beigezogene Experten (z.B. im Falle einer Thesis.) Auch hier
%    gilt die gleiche Regel betreffend Singular/Plural und Fehler-Ausl\"osung.
%
%    \DescribeMacro{\partner}
%    \DescribeMacro{\partners}
%    Industriepartner, falls vorhanden und falls sie genannt werden d\"urfen.
%
%    \subsection{Dokumenten-Inhalt}
%    \label{subsec:content-generation}
%    \DescribeMacro{\title}
%    Der Titel der Seite. Dieser Text wird auch f\"ur das Inhaltsverzeichnis verwendet
%    im Falle eines Sammlungs-Dokuments (intern wird er automagisch in ein \verb|\chapter|
%    gespeichert).
%\begin{verbatim}
%\title{Grandioser Title von einiger Wichtigkeit}
%\end{verbatim}
%    \emph{Beachte:}  Der Titel wird niemals einen Zeilenumbruch machen. Falls er
%    zu lange f\"ur eine einzelne Zeile ist, wird er einfach in den rechten Seitenrand
%    laufen. Dies ist Absicht. Man k\"urze in diesem Falle den Titel entsprechend.
%
%    \DescribeMacro{\fssummary}
%    Hier kommt der Einleitungstext hinein. Idealerweise sollte die Einleitung nicht
%    l\"anger als drei bis maximal vier Zeilen sein. Das Macro erwzingt diese L\"ange
%    allerdings nicht. Beispiel\footnote{Goethe's Faust -- Nacht; dies ergibt 3.5 Zeilen.}:
%\begin{verbatim}
%\fssummary{%
%    Habe nun, ach! Philosophie,
%    Juristerei und Medizin,
%    Und leider auch Theologie
%    Durchaus studiert, mit heissem Bem\"uh'n.
%    Da steh ich nun, ich armer Tor!
%    Und bin so klug als wie zuvor;
%    Heisse Magister, heisse Doktor gar,
%    Und ziehe schon an die zehen Jahr
%    Herauf, herab und quer und krumm
%    Meine Sch\"uler an der Nase herum.}
%\end{verbatim}
% 
%    \DescribeMacro{\fsgraphics}
%    Dieser Befehl wird eine Box zwischen \verb|\fssummary| und \verb|\fscontent|
%    setzen mit allf\"alligen Grafiken. In der Gestaltung des Inhalts von \verb|\fsgraphics|
%    ist man relativ frei. Sinnvollerweise kommt hier \verb|\includegraphics| zum Zug.
%    
%    Will man mehrere Grafiken einbinden und nebeneinander anordnen, kann man dies
%    mit \verb|minipage|s bewerkstelligen (siehe Beispiel weiter unten)
%    \footnote{Mehr zu Minipages und ihr manchmal etwas \"uberraschendes Verhalten:\newline
%        \href{https://github.com/alpenwasser/TeX/tree/master/minipages}.
%             {\nolinkurl{https://github.com/alpenwasser/TeX/tree/master/minipages}}}.
%
%    \DescribeMacro{\graphicscaption}
%    Die Legende f\"ur ein Bild. Kann mehrmals verwendet werden im Falle von mehreren
%    Bildern. Sinnvollerweise kurz halten.
%
%    \DescribeMacro{\graphicssource}
%    Quellenangabe f\"ur Bilder. Wird vertikal im rechten Seitenrand platziert. Auch hier
%    liegt die W\"urze in der K\"urze (siehe Beispiel).
%
%    Ein Beispiel zur Anordnung von zwei Bildern entlang der horizontalen Achse:
%\begin{verbatim}
%\fsgraphics{%
%    \begin{minipage}{0.4\textwidth}
%        \includegraphics[height=60mm]{images/bild0.png}
%        \graphicscaption{Dies ist ein Bild}
%    \end{minipage}%
%    \begin{minipage}{0.6\textwidth}
%        \includegraphics[height=60mm]{images/bild1.png}
%        \graphicscaption{Dies ist ein anderes Bild}
%    \end{minipage}
%    \graphicssource{Wikipedia}}
%\end{verbatim}
%
%    \DescribeMacro{\fscontent}
%    Der Hauptinhalt, aufgeteilt in drei Spalten. Der Text kann wie gehabt via
%    \verb|\section| unterteilt werden (\verb|\subsection| und \verb|\subsubsection|
%    sind jedoch deaktiviert).
% 
%    Auch Bilder k\"onnen eingebunden werden, falls man m\"ochte (siehe das
%    separate Beispiel zu \emph{Sonart}). Beispiel:

%\begin{verbatim}    
%\section{Faust -- Nacht}
%\label{sec:faust-night}
%
%Und sehe, daß wir nichts wissen k\"onnen!
%Das will mir schier das Herz verbrennen.
%Zwar bin ich gescheiter als all die Laffen,
%Doktoren, Magister, Schreiber und Pfaffen;
%Mich plagen keine Skrupel noch Zweifel,
%F\"urchte mich weder vor H\"olle noch Teufel -
%Daf\"ur ist mir auch alle Freud´ entrissen,
%Bilde mir nicht ein, was Recht's zu wissen,
%Bilde mir nicht ein, ich k\"onnte was lehren,
%Die Menschen zu bessern und zu bekehren.
%
%\section{Faust -- Mehr Nacht}
%\label{sec:faust-more-night}
%
%Auch hab´ ich weder Gut noch Geld,
%Noch Ehr´ und Herrlichkeit der Welt;
%Es m\"ochte kein Hund so l\"anger leben!
%Drum hab´ ich mich der Magie ergeben,
%Ob mir durch Geistes Kraft und Mund
%Nicht manch Geheimnis w\"urde kund;
%Dass ich nicht mehr mit saurem Schweiss
%Zu sagen brauche, was ich nicht weiss;
%Dass ich erkenne, was die Welt
%Im Innersten zusammenh\"alt,
%Schau´ alle Wirkenskraft und Samen,
%Und tu´ nicht mehr in Worten kramen.
%\end{verbatim}    
%    \DescribeMacro{\infobox}
%    Eine Box, welche links unten auf der Seite platziert wird. Kann
%    grunds\"atzlich mit mehr oder weniger beliebigem Inhalt gef\"ullt werden.
%    Bietet sich gut an, um besondere Highlights oder Eckdaten hervorzuheben.
%    Beispiel:
%\begin{verbatim}
%\infobox{Highlights}{%
%    \footnotesize
%    \setlength\tabcolsep{2pt} % make table a bit more compact horizontally
%    \begin{tabular}{ll}
%    Category 1 & Property A \\
%               & Property B \\
%               & Property C \\
%               & Property D \\
%    Category 2 & Property E \\
%               & Property F \\
%               & Property G \\
%               & Property H \\
%    \end{tabular}}
%\end{verbatim}
%    \DescribeMacro{\includefactsheet}
%    L\"adt ein \verb|Content Document| in einem \verb|Master Document| (siehe
%    Abschnitt~\ref{sec:concept}).
% \fi
%^^A >>>

%^^A ------------------------------------------------------------------------- %
%^^A EXAMPLES <<<
% \ifenglish\section{Examples}\else\section{Beispiele}\fi
% \label{sec:examples}
% \ifenglish
%    The \verb|examples| directory contains a few examples.
% \else
%    Das Verzeichnis \verb|examples| enth\"alt einige Beispiele.
% \fi
%^^A No options, just a few things to pay attention to: ... list of things
%^^A >>>

%^^A ------------------------------------------------------------------------- %
%^^A STYLE GUIDELINES <<<
%^^A \ifenglish\section{Style Guidelines}\else\section{Stil-Richtlinien}\fi
%^^A \label{sec:examples}
%^^A \ifenglish
%^^A    TODO
%^^A \else
%^^A    TODO
%^^A \fi
%^^A No options, just a few things to pay attention to: ... list of things
%^^A >>>
%
%^^A ------------------------------------------------------------------------- %
%^^A IMPLEMENTATION <<<
% \section{Implementation}
% \label{sec:implementation}
%
% \StopEventually{\PrintIndex}
% \ifenglish
% If you are merely intending to use this class, you probably do not need to read this.
% \else
% Normale Benutzer brauchen diesen Abschnitt vermutlich nicht zu lesen.
% \fi
%^^A Need the class guards to prevent the macro code from being included again 
%^^A in template.tex and the German .tex file
%\iffalse
%<*class> 
%\fi
%    \begin{macrocode}
\newif\if@english
\newif\if@german
\DeclareOption{english}{\@englishtrue\@germanfalse}
\DeclareOption{german}{\@germantrue\@englishfalse}
\DeclareOption{deutsch}{\@germantrue\@englishfalse}
\ExecuteOptions{english,german,deutsch}
\ProcessOptions

% Packages
\LoadClass{report}
\RequirePackage[utf8]{inputenc}
\RequirePackage[T1]{fontenc}
\RequirePackage{graphicx}
\RequirePackage{tikz}
\RequirePackage{rviewport}
\RequirePackage{multicol}
\RequirePackage[inline]{enumitem}
\RequirePackage{geometry}
\RequirePackage[scaled=0.95]{helvet}
\renewcommand\familydefault{\sfdefault}
\RequirePackage{xcolor}
\RequirePackage{rotating}
% Define document colos palette
\definecolor{parties}{HTML}{666666}
\definecolor{title}{HTML}{000066}
\definecolor{summary}{HTML}{666666}
\colorlet{boxbg}{cyan!50!blue!20!white}
\RequirePackage{tcolorbox}
\RequirePackage{booktabs}
\RequirePackage[binary-units=true]{siunitx}
\sisetup{range-phrase=--}
\RequirePackage[%
    hidelinks=true,
    bookmarksnumbered=true]{hyperref}
\if@english
    \RequirePackage[english]{babel}
\fi
\if@german
    \RequirePackage[ngerman]{babel}
\fi

% Page Geometry
\geometry{a4paper,twoside=false}
\newlength{\factpageouter}
\newlength{\factpageinner}
\newlength{\factpagetop}
\newlength{\factpagebottom}
\newlength{\factpagewidth}
\newlength{\factgraphicswidth}
\newlength{\testl}
\setlength{\factpageouter}{25mm}
\setlength{\factpageinner}{25mm}
\setlength{\factpagetop}{8mm}
%    \end{macrocode}
%\iffalse
%</class>
%\fi
% Needed so that page numbers do not slide down too far.
%\iffalse
%<*class> 
%\fi
%    \begin{macrocode}
\setlength{\factpagebottom}{30mm}
%    \end{macrocode}
%\iffalse
%</class>
%\fi
% Rest of page geometry layout:
%\iffalse
%<*class> 
%\fi
%    \begin{macrocode}
\setlength{\factpagewidth}{\paperwidth}
\addtolength{\factpagewidth}{-\factpageinner}
\addtolength{\factpagewidth}{-\factpageouter}
\setlength{\factgraphicswidth}{50mm}
\geometry{outer=\factpageouter,inner=\factpageinner,%
          top=\factpagetop,    bottom=\factpagebottom}

\let\newcol\newpage
%    \end{macrocode}
%\iffalse
%</class>
%\fi
% We put the entire  page inside a minipage environment to  make it not take
% up more space than a single page. We  also set the height of that minipage
% to zero so that  it stays on the first page no matter  how much content is
% put inside it.
%\iffalse
%<*class> 
%\fi
%    \begin{macrocode}
\newenvironment{factpage}
    {\noindent\begin{minipage}[t][0pt]{\factpagewidth}}
    {\end{minipage}}
%    \end{macrocode}
%\iffalse
%</class>
%\fi
% Redefine Section Headings  to not be numbered by default. This  is done by
% adding a * after its sixth argument. The rest of the command is similar to
% what can be found in The LaTeX article.cls file.
%\iffalse
%<*class> 
%\fi
%    \begin{macrocode}
\renewcommand\section{\@startsection {section}{1}{\z@}%
                                     {-2.5ex \@plus -1ex \@minus -.2ex}%
                                     {1ex}%
                                     %{1ex \@plus.2ex}% default
                                     {\large\bfseries\color{title}}*}
%    \end{macrocode}
%\iffalse
%</class>
%\fi
% Having subsections  or even  lower-tiered headings in  a document  such as
% this one would be ridiculous. Remove them to make sure they are not used.
%\iffalse
%<*class> 
%\fi
%    \begin{macrocode}
\let\subsection\@undefined
\let\subsubsection\@undefined
\let\paragraph\@undefined
\let\subparagraph\@undefined
%    \end{macrocode}
%\iffalse
%</class>
%\fi
% Define language-specific terms.
%\iffalse
%<*class> 
%\fi
%    \begin{macrocode}
\iflanguage{english}{%
    \newcommand*{\lang@team}    {\textbf{Project Team: }}
    \newcommand*{\lang@client}  {\textbf{Client: }}
    \newcommand*{\lang@coach}   {\textbf{Coach: }}
    \newcommand*{\lang@coaches} {\textbf{Coaches: }}
    \newcommand*{\lang@expert}  {\textbf{Expert: }}
    \newcommand*{\lang@experts} {\textbf{Experts: }}
    \newcommand*{\lang@partner} {\textbf{Industry Partner: }}
    \newcommand*{\lang@partners}{\textbf{Industry Partners: }}
    \newcommand*{\lang@source}  {\textit{Source: }}
}{}
\iflanguage{ngerman}{%
    \newcommand*{\lang@team}    {\textbf{Arbeitsgruppe: }}
    \newcommand*{\lang@client}  {\textbf{Auftraggeber: }}
    \newcommand*{\lang@coach}   {\textbf{Betreuer: }}
    \newcommand*{\lang@coaches} {\textbf{Betreuer: }}
    \newcommand*{\lang@expert}  {\textbf{Experte: }}
    \newcommand*{\lang@experts} {\textbf{Experten: }}
    \newcommand*{\lang@partner} {\textbf{Industriepartner: }}
    \newcommand*{\lang@partners}{\textbf{Industriepartner: }}
    \newcommand*{\lang@source}  {\textit{Quelle: }}
}{}
%    \end{macrocode}
%\iffalse
%</class>
%\fi
% Save those terms in the appropriate macros.
%\iffalse
%<*class> 
%\fi
%    \begin{macrocode}
\newcommand*{\team}[1]
    {\gdef\@team    {{\footnotesize\color{parties}\lang@team#1}}}
\newcommand*{\client}[1]
    {\gdef\@client  {{\footnotesize\color{parties}\lang@client#1}}}
\newcommand*{\coach}[1]
    {\gdef\@coach   {{\footnotesize\color{parties}\lang@coach#1}}}
\newcommand*{\coaches}[1]
    {\gdef\@coaches {{\footnotesize\color{parties}\lang@coaches#1}}}
\newcommand*{\expert}[1]
    {\gdef\@expert  {{\footnotesize\color{parties}\lang@expert#1}}}
\newcommand*{\experts}[1]
    {\gdef\@experts {{\footnotesize\color{parties}\lang@experts#1}}}
\newcommand*{\partner}[1]
    {\gdef\@partner {{\footnotesize\color{parties}\lang@partner#1}}}
\newcommand*{\partners}[1]
    {\gdef\@partners{{\footnotesize\color{parties}\lang@partners#1}}}
%    \end{macrocode}
%\iffalse
%</class>
%\fi
% Redefine \verb|maketitle| to do our bidding:
%\iffalse
%<*class> 
%\fi
%    \begin{macrocode}
\renewcommand\maketitle{\par
    \begingroup
        {\Huge\color{title}\@title}\par
        \vspace{2em}
        \global\let\maketitle\relax
        \global\let\@maketitle\relax
        \global\let\title\relax
        \global\let\date\relax
        \global\let\@title\@empty
        \global\let\@date\@empty
    \endgroup
}
%    \end{macrocode}
%\iffalse
%</class>
%\fi
% For testing purposes:
%\iffalse
%<*class> 
%\fi
%    \begin{macrocode}
\newlength{\placeholderbase}
\newcommand\placeholderpics{%
    \setlength{\placeholderbase}{48mm}
    \begin{tikzpicture}[x=\placeholderbase,y=\placeholderbase,draw=black!50!white,very thick]
        \draw (0,0) -- (1.618,1);
        \draw (0,1) -- (1.618,0);
        \draw (0,0) -- (1.618,0) -- (1.618,1) -- (0,1) -- cycle;
        \node[draw,fill=white] at (0.809,0.5) {\large Beautiful Picture!};
    \end{tikzpicture}
    \begin{tikzpicture}[x=\placeholderbase,y=\placeholderbase,draw=black!50!white,very thick]
        \draw (0,0) -- (1.618,1);
        \draw (0,1) -- (1.618,0);
        \draw (0,0) -- (1.618,0) -- (1.618,1) -- (0,1) -- cycle;
        \node[draw,fill=white] at (0.809,0.5) {\large Amazing Graphics!};
    \end{tikzpicture}\par
    {\footnotesize Beautiful Picture and Amazing Graphics}
}
%    \end{macrocode}
%\iffalse
%</class>
%\fi
% Store the concents of \verb|\fscontent| in \verb|\fs@content|
%\iffalse
%<*class> 
%\fi
%    \begin{macrocode}
\newcommand\fscontent[1]{\gdef\fs@content{#1}}
%    \end{macrocode}
%\iffalse
%</class>
%\fi
% Print the contents of \verb|\fs@content|
%\iffalse
%<*class> 
%\fi
%    \begin{macrocode}
\newcommand\fs@makecontent{\par
    \vspace{1em}\begin{multicols}{3}
        \ifdef{\fs@content}{\fs@content}\par
    \end{multicols}
}
%    \end{macrocode}
%\iffalse
%</class>
%\fi
% Store the concents of \verb|\fssummary| in \verb|\fs@summary|
%\iffalse
%<*class> 
%\fi
%    \begin{macrocode}
\newcommand\fssummary[1]{\gdef\fs@summary{#1}}
%    \end{macrocode}
%\iffalse
%</class>
%\fi
% Print the contents of \verb|\fs@summary|
%\iffalse
%<*class> 
%\fi
%    \begin{macrocode}
\newcommand\fs@makesummary{\par
    \vspace{0.25em}\noindent\begin{minipage}{\textwidth}\large
        \ifdef{\fs@summary}{\color{summary}\fs@summary}\par
    \end{minipage}%
}
%    \end{macrocode}
%\iffalse
%</class>
%\fi
% Store the concents of \verb|\fsgraphics| in \verb|\fs@graphics|
%\iffalse
%<*class> 
%\fi
%    \begin{macrocode}
\newcommand\fsgraphics[1]{\gdef\fs@graphics{#1}}
\newcommand\graphicssource[1]{\gdef\fs@graphicssource{#1}}
%    \end{macrocode}
%\iffalse
%</class>
%\fi
% Put the contents of \verb|\graphicscaption| inside an \verb|\mbox| so that it does not wrap.
%\iffalse
%<*class> 
%\fi
%    \begin{macrocode}
\newcommand\graphicscaption[1]{\mbox{#1}}
%    \end{macrocode}
%\iffalse
%</class>
%\fi
% Print the contents of \verb|\fs@graphics|
%\iffalse
%<*class> 
%\fi
%    \begin{macrocode}
\newcommand\fs@makegraphics{\par
    \vspace{1em}\noindent\begin{minipage}[b]{\textwidth}%
        \footnotesize\raggedright
        \ifdef{\fs@graphics}{\fs@graphics}\par\par
        \ifdef{\fs@graphicscaption}{\fs@graphicscaption}\par
    \end{minipage}%
    \begin{minipage}[t]{2ex}\footnotesize\color{parties}
        \ifdef{\fs@graphicssource}{%
            \hspace*{2ex}%
            \turnbox{90}{%
                \hspace*{2ex} \lang@source \fs@graphicssource}}\par
    \end{minipage}%
}
%    \end{macrocode}
%\iffalse
%</class>
%\fi
% Store the content of \verb|\infobox| and print it:
%\iffalse
%<*class> 
%\fi
%    \begin{macrocode}
\newcommand\infobox[2]{%
    \gdef\fs@infoboxtitle{#1}%
    \gdef\fs@infoboxcontent{\footnotesize #2}%
}
\newcommand\fs@makeinfobox{
    \begin{tcolorbox}[
            title=\fs@infoboxtitle,
            width=0.66\textwidth,
            arc=0mm,
            before=,
            after=\hfill]
        \fs@infoboxcontent
    \end{tcolorbox}
}
\newcommand\tightlist{%
    \setlength{\itemsep}{0pt}\setlength{\parskip}{0pt}%
}
%    \end{macrocode}
%\iffalse
%</class>
%\fi
% Errors:
%\iffalse
%<*class> 
%\fi
%    \begin{macrocode}
\newcommand\fs@checkinput{%
    \ifdef{\@coach}{%
        \ifdef{\@coaches}{%
            \ClassError{fhnwfactsheet}{Both 'coach' and 'coaches' field defined. Undefine one.}
                {Both the 'coach' and the 'coaches' field are defined. Only one of them should be.}
        }{}%
    }{}%
    \ifdef{\@expert}{%
        \ifdef{\@experts}{%
            \ClassError{fhnwfactsheet}{Both 'expert' and 'experts' field defined. Undefine one.}
                {Both the 'expert' and the 'experts' field are defined. Only one of them should be.}
        }{}%
    }{}%
    \ifdef{\@partner}{%
        \ifdef{\@partners}{%
            \ClassError{fhnwfactsheet}{Both 'partner' and 'partners' field defined. Undefine one.}
                {Both the 'partner' and the 'partners' field are defined. Only one of them should be.}
        }{}%
    }{}%
    \ifdef{\fs@infoboxcontent}{}
        {%
            \ClassError{fhnwfactsheet}{Did you forget to specify your infobox?}
            {The infobox contents are not defined. Make sure your infobox command is properly called.}
        }%
    \ifdef{\fs@infoboxtitle}{}
        {%
            \ClassError{fhnwfactsheet}{Did you forget to specify your infobox?}
            {The infobox title is not defined. Make sure your infobox command is properly called.}
        }%
}
%    \end{macrocode}
%\iffalse
%</class>
%\fi
% Chapter code. Taken mostly from \verb|classes.dtx|.
%\iffalse
%<*class> 
%\fi
%    \begin{macrocode}
\renewcommand\chapter{%\if@openright\cleardoublepage\else\clearpage\fi
                    \thispagestyle{plain}%
                    \global\@topnum\z@
                    \@afterindentfalse
                    \secdef\@chapter\@schapter}
\def\@chapter[#1]#2{\ifnum \c@secnumdepth >\m@ne
                         \refstepcounter{chapter}%
                         \typeout{\@chapapp\space\thechapter.}%
                         \addcontentsline{toc}{chapter}{#1}%
                                   %{\protect\numberline{\thechapter}#1}%
                    \else
                      \addcontentsline{toc}{chapter}{#1}%
                    \fi
                    \chaptermark{#1}%
                    \addtocontents{lof}{\protect\addvspace{10\p@}}%
                    \addtocontents{lot}{\protect\addvspace{10\p@}}%
                    \if@twocolumn
                      \@topnewpage[\@makechapterhead{#2}]%
                    \else
                      \@makechapterhead{#2}%
                      \@afterheading
                    \fi}
\def\@makechapterhead#1{%
    \begingroup
        \mbox{\Huge\color{title}#1}\par
        \vspace{2em}
    \endgroup
}
\def\@makeschapterhead#1{%
  \vspace*{50\p@}%
  {\parindent \z@ \raggedright
    \normalfont
    \interlinepenalty\@M
    \Huge\color{title} #1\par\nobreak
    \vskip 40\p@
  }}
\renewcommand*\l@chapter[2]{%
  \ifnum \c@tocdepth >\m@ne
    \addpenalty{-\@highpenalty}%
    \vskip 1.0em \@plus\p@
    \setlength\@tempdima{1.5em}%
    \begingroup
      \parindent \z@ \rightskip \@pnumwidth
      \parfillskip -\@pnumwidth
      \leavevmode 
      \advance\leftskip\@tempdima
      \hskip -\leftskip
      %\@dottedtocline{0}{0em}{0em}{#1}{#2}\par
      #1\nobreak\hfil \nobreak\hb@xt@\@pnumwidth{\hss #2}\par
      \penalty\@highpenalty
    \endgroup
  \fi}
%    \end{macrocode}
%\iffalse
%</class>
%\fi
% Including a fact sheet content file in a master document:
%\iffalse
%<*class> 
%\fi
%    \begin{macrocode}
\newcommand*\includefactsheet[1]{\input{#1}\makefactsheet\newpage}
%    \end{macrocode}
%\iffalse
%</class>
%\fi
% Output routine:
%\iffalse
%<*class> 
%\fi
%    \begin{macrocode}
\newcommand\makefactsheet{
    \begin{factpage}
        \fs@checkinput
        \chapter{\@title}
        % Must be after \chapter
        % Note: We do not want to overwrite the \chapter pagestyle{plain} everywhere.
        \thispagestyle{empty}
        \fs@makesummary
        \vspace{1ex}
        \fs@makegraphics
        \vspace{1ex}
        \fs@makecontent
        \vspace{1ex}
        \tcbset{
            colback=boxbg,
            colframe=title,
            boxrule=0.5pt,
            toptitle=2pt,
            bottomtitle=1pt,
            titlerule=0.5pt,
        }
        \fs@makeinfobox
        \begin{minipage}[b]{0.30\textwidth}
            \ifdef{\@team}{\@team}{}\par
            \ifdef{\@client}{\@client}{}\par
            \ifdef{\@coach}{\@coach}{}\par
            \ifdef{\@coaches}{\@coaches}{}\par
            \ifdef{\@expert}{\@expert}{}\par
            \ifdef{\@experts}{\@experts}{}\par
            \ifdef{\@partner}{\@partner}{}\par
            \ifdef{\@partners}{\@partners}{}\par
        \end{minipage}
    \end{factpage}\raisebox{6ex}{\hspace{3em}\thepage}
    % Undefine for next fact sheet
    \let\fs@graphics\@undefined
    \let\fs@graphicssource\@undefined
    \let\fs@graphicscaption\@undefined
    \let\@team\@undefined
    \let\@client\@undefined
    \let\@coach\@undefined
    \let\@coaches\@undefined
    \let\@expert\@undefined
    \let\@experts\@undefined
    \let\@partner\@undefined
    \let\@partners\@undefined
    \let\fs@infoboxtitle\@undefined
    \let\fs@infoboxcontent\@undefined
}
%    \end{macrocode}
%\iffalse
%</class>
%\fi
%
% \Finale
%^^A >>>
%
%^^A ------------------------------------------------------------------------- %
%^^A TEMPLATE <<<
% \iffalse
%<*template>
\documentclass{fhnwfactsheet}
\usepackage{lipsum}
%\usepackage{FiraSans}

%% ================================================================= SET TITLE %
\title{Phenomenal Title!}
\author{Glorious Author!}

%% ============================================================== END PREAMBLE %
\begin{document}
\begin{factpage}
%% ============================================================= BEGIN CONTENT %

%% ------------------------------------------------------------------ OVERVIEW %
\begin{overview}
    \lipsum[2]
\end{overview}

\begin{content}

    \section{This is a section}
    \lipsum[2]
    \section{This is another section}
    \lipsum[4]
    \section{And another one!}
    \lipsum[5]

\end{content}

\lipsum[1]

\factgraphics{a-stunning-picture.png}{a-very-nice-graphic.png}

%% ------------------------------------------------------------------ STUDENTS %
\begin{students}
    \lipsum[2]
\end{students}

%% ------------------------------------------------------------------- COACHES %
\begin{coaches}
    \lipsum[2]
\end{coaches}

%% ----------------------------------------------------------------- PARTNERS %
\begin{partners}
    \lipsum[2]
\end{partners}

%% ============================================================== END CONTENT %
\end{factpage}
\end{document}
%</template>
% \fi
%^^A >>>
%
%^^A ------------------------------------------------------------------------- %
%^^A MULTI-LANGUAGE SUPPORT <<<
% \iffalse
%<*de>
\expandafter\newif\csname ifenglish\endcsname
\englishfalse
% \iffalse meta-comment
% -------------------------------------------------------- %
% Copyright (c) 2017 Raphael Frey rmfrey@runbox.com        %
% -------------------------------------------------------- %
%                                                          %
% This file  may be distributed and/or  modified under the %
% conditions of  the LaTeX Project Public  License, either %
% version 1.3c  of this  license or  (at your  option) any %
% later version. The latest version of this license is in: %
%                                                          %
%     http://www.latex-project.org/lppl.txt                %
%                                                          %
% -------------------------------------------------------- %
% \fi
%
%\CheckSum{615}
%
%
%^^A ------------------------------------------------------------------------- %
%^^A DRIVER, PROVIDESCLASS ETC. <<<
% \iffalse
%<*driver>
\ProvidesFile{fhnwfactsheet.dtx}
%</driver>
%<class>\NeedsTeXFormat{LaTeX2e}
%<class>\ProvidesClass{fhnwfactsheet}%
%<*class>
    [2017/07/12 v1.0.0 PDF-LaTeX class for FHNW fact sheets]
%</class>
%<*driver>
\expandafter\ifx\csname ifenglish\endcsname\relax
    \expandafter\newif\csname ifenglish\endcsname
    \englishtrue
\fi
\documentclass[a4paper,11pt,oneside]{ltxdoc}
\EnableCrossrefs
\CodelineIndex
\RecordChanges
% ----------------------------------------------- PACKAGES %
\ifenglish
    \usepackage[english]{babel}
\else
    \usepackage[ngerman]{babel}
\fi
\usepackage{xcolor-solarized}
\usepackage[%
    bookmarksnumbered=true,
    colorlinks=true,
    linkcolor=cyan!50!blue,
    citecolor=violet,
    urlcolor=purple]{hyperref}
\usepackage[light,nott]{kpfonts}
\usepackage[scaled=0.85]{DejaVuSansMono}
\usepackage{dirtree}
% ------------------------------------------ HELPER MACROS %
\def\pacname#1{\textsf{#1}}
\def\fhnwfs{\pacname{fhnwfactsheet}}
\def\fhnwenfull{University of Applied Sciences and Arts Northwestern Switzerland}
\def\fhnwdefull{Fachhochschule Nordwestschweiz}
\def\fhnw{FHNW}
\def\hten{School of Engineering}
\def\htde{Hochschule f\"ur Technik}
\def\code#1{\texttt{#1}}
\begin{document} % ****************************** DOCUMENT %
    \DocInput{fhnwfactsheet.dtx}
\end{document}
%</driver>
% \fi
%^^A >>>
%
%^^A ------------------------------------------------------------------------- %
%^^A TITLE, ABSTRACT, ToC <<<
% \GetFileInfo{fhnwfactsheet.dtx}
% \author{Raphael Frey \\[0.5em] \href{https://github.com/alpenwasser/fhnwfactsheet/}%
%                                     {\nolinkurl{https://github.com/alpenwasser/fhnwfactsheet/}}}
%
% \ifenglish
%   \title{The \fhnwfs\ Class\thanks{This document corresponds
%       to \fhnwfs~\fileversion, dated~\filedate.}}
%   \maketitle
%   \begin{abstract}
%        The  \fhnwfs\  class  offers  a   simple  interface  for  composing  a
%        fact  sheet for  projects  in A4  paper size  for  the Degree  Program
%        \emph{Electrical  and Information  Technology}  at the  \hten\ of  the
%        \fhnwenfull.
%
%        This  document  contains  the  documentation for  user  interface  and
%        describes the implementation  for any developer who  intends to modify
%        the class.
%   \end{abstract}
% \else
%   \title{Die \fhnwfs-Klasse\thanks{Dieses Dokument entspricht
%          \fhnwfs~\fileversion, datiert~\filedate.}}
%   \maketitle
%   \begin{abstract}
%        Die  \fhnwfs\ Klasse  dient zum  Erstellen von  Factsheets  f\"ur den
%        Studiengang \emph{Elektro- und Informationstechnik}  an der \htde\ der
%        \fhnwdefull.
%
%        Dieses    Dokument    enth\"alt    die   Dokumentation    f\"ur    das
%        Benutzer-Interface  und   beschreibt  die  Implementation   f\"ur  die
%        geneigte Engwicklerin, welche die Klasse modifizieren m\"ochte.
%   \end{abstract}
% \fi
%
% \tableofcontents
%^^A >>>
%
%^^A ------------------------------------------------------------------------- %
%^^A Basic Concept <<<
% \ifenglish
%    \section{Basic Concept}
%    \label{sec:concept}
%    The class  allows to build indivitual  standalone fact sheets as  well as
%    a  larger  document containing  multiple  fact  sheets as  a  collection,
%    including a title page, a table of contents etc. The underlying structure
%    in both scenarios is identical:
%    \dirtree{%^^A
%        .1 Master Document.
%        .2 Content Document.
%        .3 Further Ressources (e.g.~images).
%    }
%    \emph{Note:} 
%    \begin{itemize}
%        \item
%        \textsf{fhnwfactsheet} does not enforce any specifif file name
%        specification. Follow your teacher's instructions if you are creating a fact
%        sheet for submission to the school.
%        \item
%        Do not load additional packages! This will mess up the process of merging your
%        standalone fact sheet with others.
%        \textsf{fhnwfactsheet}loads most classes one would reasonably need to create
%        a fact sheet. If there really is a package which you absolutely, positively
%        need, contact the package maintainer and he will see if it can be integrated
%        into the class. This will then allow to push the class to everyone through an
%        update and things should (hopefully) not break.
%    \end{itemize}
%
%    \subsection{Master Document}
%    \label{subsec:masterdoc}
%    The master document is the  document on which you run \verb|pdflatex|. It
%    sets up the class and its  configuration and then loads the \verb|Content Document|.
%    If you are working on a  standalone fact sheet, this file will
%    be very minimal:
% \else
%    \section{Grundkonzept}
%    \label{sec:concept}
%    Die  Klasse   erlaubt  sowohl  das  Erstellen   von  einzelnen  Factsheets
%    als  eigenst\"andige  Dokumente  wie  auch das  Erstellen  einer  Sammlung
%    von   Factsheets   in   einerm  gr\"osseren   Dokument   mit   Titelseite,
%    Inhaltsverzeichnis etc. Die grundlegende Struktur ist dabeit identisch:
%    \dirtree{%^^A
%        .1 Master Document.
%        .2 Content Document.
%        .3 Weitere Ressourcen (z.B.~Bilder).
%    }
%
%    \emph{Beachte:} 
%    \begin{itemize}
%        \item
%        \textsf{fhnwfactsheet} erzwingt keine spezielle Dateinamen
%        oder Ordnerstruktur. Man beachte die Anweisung der Lehrpersonals diesbez\"uglich
%        wenn man ein Factsheet zur Einreichung bei der Schule erstellt.
%        \item
%        Es sollten keine zus\"atzlichen Packages geladen werden. Dies wird den Prozess
%        des Zusammenstellens verschiedener eigenst\"andiger Factsheets st\"oren.
%        \textsf{fhnwfactsheet} l\"adt die meisten Pakete, welche man \"ublicherweise
%        zur Erstellung eines Factsheets ben\"otigt. Falls wirklich ein Paket nicht
%        geladen ist, welches absolute, definitiv, sicherlich ben\"otigt wird, kontaktiere
%        man den Package-Maintainer (siehe erste Seite). Er wird sehen, ob sich das
%        zus\"atzliche Paket in die Klasse integrieren l\"asst, und falls ja, wird
%        das entsprechende Update zur Verf\"ugung stellen. Dies
%        wird hoffentlich einen problemlosen Arbeitsprozess f\"ur alle beteiligten
%        Parteien erm\"oglichen.
%    \end{itemize}
%
%    \subsection{Master Document}
%    \label{subsec:masterdoc}
%    Das  Master-Dokument  ist  das Dokument,  welches  durch  \verb|pdflatex|
%    verarbeitet  wird. Es  richtet  die  Klasse  und  Konfiguration  ein  und
%    l\"adt  anschliessend  den  eigentlichen  Inhalt  aus  dem  \verb|Content Document|.
%    Wenn ein eigenst\"andiges Factsheet  erstellt wird, ist dieses
%    Dokument ziemlich minimalistisch:
% \fi
%\begin{verbatim}
%\documentclass[german]{fhnwfactsheet}
%\begin{document}
%\includefactsheet{contentDocument.tex}
%\end{document}
%\end{verbatim}
% \ifenglish
%     Multiple \verb|Content Document|s can be included if you wish to create a collection
%     of fact sheets. Each fact sheet will be put on a single page. This also enables
%     the creation of a table of contents.
% \else
%    Mehrere \verb|Content Document|s k\"onnen geladen werden, wenn man eine Sammlung von
%    Factsheets erstellen m\"ochte. Jedes Factsheet wird auf einer einzelnen Seite
%    erstellt werden. Ein Inhaltsverzeichnis kann ebenfalls erstellt werden.
% \fi
%\begin{verbatim}
%\documentclass[german]{fhnwfactsheet}
%\begin{document}
%\pagenumbering{roman}
%\tableofcontents
%
%\newpage\pagenumbering{arabic}
%
%\includefactsheet{factsheet1Content.tex}
%\includefactsheet{factsheet2Content.tex}
%\includefactsheet{factsheet3Content.tex}
%\includefactsheet{factsheet4Content.tex}
%\end{document}
%\end{verbatim}
% \ifenglish
%    \subsection{Content Document}
%    \label{subsec:contentdoc}
%    The content document is where the content of your fact sheet actually resides.
%    it offers some specifics commands to enable consistent layout and typography
%    and places some restrictions which should prevent things from getting too
%    extravagant. See the \verb|examples| directory for some example code.
% \else
%    \subsection{Content Document}
%    \label{subsec:contentdoc}
%    Das \verb|Content Document| ist das Dokument, wo der eigentliche Inhalt
%    untergebracht ist. Es offeriert einige spezifische Befehle, um konsistentes
%    Layout und Typographie zu erm\"oglichen und setzt einige Restriktionen, um
%    allzu extravagante Ideen zu verhindern (oder zumindest zu erschweren).
%    Beispiele k\"onnen im Verzeichnis  \verb|examples| gefunden werden.
% \fi
%^^A >>>
%^^A ------------------------------------------------------------------------- %
%^^A INSTALLATION <<<
% \section{Installation}
% \label{sec:installation}
%
% \ifenglish
%    From \verb|fhnwfactsheet.dtx| and \verb|fhnwfactsheet.ins|, generate the class file \verb|fhnwfactsheet.cls|
%    by running
%\begin{verbatim}
%pdflatex fhnwfactsheet.ins
%\end{verbatim}
%    This will create the class file. If you wish to compile the documentation
%    (this document), run:
%\begin{verbatim}
%pdflatex fhnwfactsheet.dtx
%\end{verbatim}
%    After that, place \verb|fhnwfactsheet.cls| in your working directory and
%    enable it via the \verb|documentclass| command (see example for master document
%    above).
%
%    ^^A\inputminted[linenos=true,numbersep=4pt,fontsize=\footnotesize]{tex}{template.tex}
%   ^^A TODO: File structure tree with the needed files.
% \else
%    Aus \verb|fhnwfactsheet.dtx| und \verb|fhnwfactsheet.ins| kann das Klassen-Dokument
%    \verb|fhnwfactsheet.cls| generiert werden via:
%\begin{verbatim}
%pdflatex fhnwfactsheet.ins
%\end{verbatim}
%    Zum Kompilieren der Dokumentation (dieses Dokument):
%\begin{verbatim}
%pdflatex fhnwfactsheet.dtx
%\end{verbatim}
%    Anschliessend \verb|fhnwfactsheet.cls| in das Verzeichnis des Factsheet-Projekts
%    kopieren und via \verb|documentclass| laden (siehe Master Document oben).
% \fi
%^^A >>>
%
%^^A CLASS OPTIONS <<<
%
% \ifenglish
%    \section{Class Options}
%    \label{sec:options}
%
%    \DescribeMacro{english}
%    \DescribeMacro{german}
%    \DescribeMacro{deutsch}
%    These class options will load the corresponding \textsf{babel} language
%    as well as set some key words accordingly. If you don't specify any
%    language, the class defaults to German.
%
%    \noindent Examples:
% \else
%    \section{Klassen-Optionen}
%    \label{sec:options}
%
%    \DescribeMacro{english}
%    \DescribeMacro{german}
%    \DescribeMacro{deutsch}
%    Diese Klassne-Optionen laden die entsprechenden Konfigurationen in \textsf{babel}
%    und definieren einige Schl\"usselw\"orter in der jeweiligen Sprache. Wenn keine
%    Sprache spezifiziert wird, l\"adt die Klasse auf deutsch.
%    
%    \noindent Beispiele:
% \fi
%\begin{verbatim}
%\documentclass{fhnwfactsheet}          % German
%\documentclass[deutsch]{fhnwfactsheet} % German
%\documentclass[german]{fhnwfactsheet}  % German
%\documentclass[english]{fhnwfactsheet} % English
%\end{verbatim}
%^^A >>>

%^^A COMMANDS <<<
%
% \ifenglish
%    \section{Commands}
%    \label{sec:commands}
%
%    This section describes the commands which are implemented by \textsf{fhnwfactsheet}.
%
%    \subsection{Language-Specific Macros}
%    \label{subsec:language-specific}
%    These macros are used to name the people involved in the project as well as their
%    roles. None of these have to be specified, if they did not exist. They will be used
%    when defined, otherwise not.
%
%    \DescribeMacro{\team}
%    Comma-separated list of team members. \emph{Note:} Use \verb|Firstname Lastname|
%    notation to ensure consistency. Example:
%\begin{verbatim}
%\team{%
%    Team Leader,
%    Personwho Didallthework,
%    Personwho Wasalsothere,
%    Dudewho Wasnotthere}
%\end{verbatim}
%    \DescribeMacro{\client}
%    The client. The format is identical to the \verb|\team| command.
%
%    \DescribeMacro{\coach}
%    \DescribeMacro{\coaches}
%    The Project coaches. In case of a single person, use \verb|\coach|, in
%    case of multiple people, use \verb|\coaches|. \emph{Note:} Since these
%    options are mutually exclusive, the class will raise an error if both
%    macros have been specified.
%
%    \DescribeMacro{\expert}
%    \DescribeMacro{\experts}
%    Experts, for example in case of a thesis. Same as above, the class will
%    raise an error in case of both singular and plural form being specified.
%
%    \DescribeMacro{\partner}
%    \DescribeMacro{\partners}
%    Industry partners, if there were any (and they can be named).
%
%    \subsection{Content Generation}
%    \label{subsec:content-generation}
%    The following macros are used to create the content which goes onto the 
%    fact sheet page.
%    \DescribeMacro{\title}
%    The title for the page. This will also be used as the title in the table
%    of contents in case of a collection document (internally it is automagically
%    mapped to a \verb|\chapter|).
%\begin{verbatim}
%\title{Grand Title of Significant Magnitude}
%\end{verbatim}
%    \emph{Note:} The title will never wrap onto a new line. If it is too long for
%    a single line, it will simply run off the page. This is by design. Shorten
%    your title in that case.
%
%    \DescribeMacro{\fssummary}
%    This is where the introductory header text goes. Ideally, it should not be longer
%    than three or maximum four lines. However, the macro itself does not enforce
%    this. Example\footnote{Goethe's Faust -- Nacht; this will result in 3.5 lines.}
%    Hier kommt der Einleitungstext hinein. Idealerweise sollte die Einleitung nicht
%    l\"anger als drei bis maximal vier Zeilen sein. Das Macro erzwingt diese L\"ange
%    allerdings nicht. Beispiel
%\begin{verbatim}
%\fssummary{%
%    Habe nun, ach! Philosophie,
%    Juristerei und Medizin,
%    Und leider auch Theologie
%    Durchaus studiert, mit heißem Bemüh'n.
%    Da steh ich nun, ich armer Tor!
%    Und bin so klug als wie zuvor;
%    Heiße Magister, heiße Doktor gar,
%    Und ziehe schon an die zehen Jahr
%    Herauf, herab und quer und krumm
%    Meine Schüler an der Nase herum.}
%\end{verbatim}
%
%    \DescribeMacro{\fsgraphics}
%    This command creates a box which is put between \verb|\fssummary| and \verb|\fscontent|
%    with the intent of being filled with graphical content. There are not really any
%    restrictions placed on the graphics, so treat with care. Usually, one would probably
%    want to use \verb|\includegraphics| here.

%    For including multiple graphics, the use of \verb|minipage|s is advised.
%    \footnote{More about minipages and their sometimes surprising behavior:\newline
%        \href{https://github.com/alpenwasser/TeX/tree/master/minipages}.
%             {\nolinkurl{https://github.com/alpenwasser/TeX/tree/master/minipages}}}.
%
%    \DescribeMacro{\graphicscaption}
%    The caption for a picture. Can be used multiple times in case of multiple pictures.
%    Should be kept short.
%
%    \DescribeMacro{\graphicssource}
%    A source for the pictures. Is placed vertically in the right page margin. This, too
%    should be kept short (see example below).
%
%    An example for including two images along the horizonal axis:
%\begin{verbatim}
%\fsgraphics{%
%    \begin{minipage}{0.4\textwidth}
%        \includegraphics[height=60mm]{images/bild0.png}
%        \graphicscaption{Dies ist ein Bild}
%    \end{minipage}%
%    \begin{minipage}{0.6\textwidth}
%        \includegraphics[height=60mm]{images/bild1.png}
%        \graphicscaption{Dies ist ein anderes Bild}
%    \end{minipage}
%    \graphicssource{Wikipedia}}
%\end{verbatim}
%
%    \DescribeMacro{\fscontent}
%    The main content of your document, split into three columns. Partition via
%    the \verb|\section| command as usual (there are no \verb|subsection|s or 
%    even \verb|\subsubsection|s available).
%
%    You can also include graphics here, should you feel so inclined (see the
%    separate \emph{Sonart} example fact sheet). Example:
%  
%\begin{verbatim}    
%\section{Faust -- Nacht}
%\label{sec:faust-night}
%Und sehe, daß wir nichts wissen können!
%Das will mir schier das Herz verbrennen.
%Zwar bin ich gescheiter als all die Laffen,
%Doktoren, Magister, Schreiber und Pfaffen;
%Mich plagen keine Skrupel noch Zweifel,
%Fürchte mich weder vor Hölle noch Teufel -
%Dafür ist mir auch alle Freud´ entrissen,
%Bilde mir nicht ein, was Recht's zu wissen,
%Bilde mir nicht ein, ich könnte was lehren,
%Die Menschen zu bessern und zu bekehren.
%
%\section{Faust -- Mehr Nacht}
%\label{sec:faust-more-night}
%Auch hab´ ich weder Gut noch Geld,
%Noch Ehr´ und Herrlichkeit der Welt;
%Es möchte kein Hund so länger leben!
%Drum hab´ ich mich der Magie ergeben,
%Ob mir durch Geistes Kraft und Mund
%Nicht manch Geheimnis würde kund;
%Daß ich nicht mehr mit saurem Schweiß
%Zu sagen brauche, was ich nicht weiß;
%Daß ich erkenne, was die Welt
%Im Innersten zusammenhält,
%Schau´ alle Wirkenskraft und Samen,
%Und tu´ nicht mehr in Worten kramen.
%\end{verbatim}    
%    \DescribeMacro{\infobox}
%    A box which gets placed on the left side at the page bottom. Can
%    fundamentally contain pretty much any content you like. Well-suited
%    for highlighing key points or specifications about the project.
%    Example:
%\begin{verbatim}
%\infobox{Highlights}{%
%    \footnotesize
%    \setlength\tabcolsep{2pt} % make table a bit more compact horizontally
%    \begin{tabular}{ll}
%    Category 1 & Property A \\
%               & Property B \\
%               & Property C \\
%               & Property D \\
%    Category 2 & Property E \\
%               & Property F \\
%               & Property G \\
%               & Property H \\
%    \end{tabular}}
%\end{verbatim}
%
%    \DescribeMacro{\includefactsheet}
%    Loads a \verb|Content Document| in a \verb|Master Document| (see
%    Section~\ref{sec:concept}).
%
% \else
%    \section{Befehle}
%    \label{sec:commands}
%    Dieser Abschnitt beschreibt die Macros, welche von \textsf{fhnwfactsheet} implementiert werden.
%
%    \subsection{Sprachspezifische Macros}
%    \label{subsec:language-specific}
%    Diese Macros werden dazu verwendet,  die im Project involvierten Personen
%    und  ihre  Rollen zu  beschreiben. Rollen,  welche  nicht besetzt  waren,
%    k\"onnen weggelassen  werden. Keiner dieser Befehle muss  benutzt werden,
%    damit  das  Dokument kompiliert  werden  kann. Sie  werden benutzt,  wenn
%    definiert, andernfalls nicht.
%
%    \DescribeMacro{\team}
%    Komma-getrennte   Liste  der   Gruppenmitglieder.  \emph{Beachte:} Zwecks
%    Einhaltung     von      nsistenz     benutze      man     \verb|Vorname Nachname|-Notation. 
%    Beispiel:
%\begin{verbatim}
%\team{%
%    Team Leader,
%    Personwho Didallthework,
%    Personwho Wasalsothere,
%    Dudewho Wasnotthere}
%\end{verbatim}
%    \DescribeMacro{\client}
%    Auftraggeber. Das Format ist identisch zum \verb|\team|-Befehl.
%
%    \DescribeMacro{\coach}
%    \DescribeMacro{\coaches}
%    Die Fachcoaches. Im Falle einer Einzelperson benutze man \verb|\coach|,
%    im Falle von mehreren Personen \verb|\coaches|. \emph{Beachte:} Da diese
%    beiden Varianten sich gegenseitig ausschliessen, wird die Klasse einen
%    Fehler ausl\"osen, wenn beide spezifiziert worden sind.
%    
%    \DescribeMacro{\expert}
%    \DescribeMacro{\experts}
%    Allf\"allig beigezogene Experten (z.B. im Falle einer Thesis.) Auch hier
%    gilt die gleiche Regel betreffend Singular/Plural und Fehler-Ausl\"osung.
%
%    \DescribeMacro{\partner}
%    \DescribeMacro{\partners}
%    Industriepartner, falls vorhanden und falls sie genannt werden d\"urfen.
%
%    \subsection{Dokumenten-Inhalt}
%    \label{subsec:content-generation}
%    \DescribeMacro{\title}
%    Der Titel der Seite. Dieser Text wird auch f\"ur das Inhaltsverzeichnis verwendet
%    im Falle eines Sammlungs-Dokuments (intern wird er automagisch in ein \verb|\chapter|
%    gespeichert).
%\begin{verbatim}
%\title{Grandioser Title von einiger Wichtigkeit}
%\end{verbatim}
%    \emph{Beachte:}  Der Titel wird niemals einen Zeilenumbruch machen. Falls er
%    zu lange f\"ur eine einzelne Zeile ist, wird er einfach in den rechten Seitenrand
%    laufen. Dies ist Absicht. Man k\"urze in diesem Falle den Titel entsprechend.
%
%    \DescribeMacro{\fssummary}
%    Hier kommt der Einleitungstext hinein. Idealerweise sollte die Einleitung nicht
%    l\"anger als drei bis maximal vier Zeilen sein. Das Macro erwzingt diese L\"ange
%    allerdings nicht. Beispiel\footnote{Goethe's Faust -- Nacht; dies ergibt 3.5 Zeilen.}:
%\begin{verbatim}
%\fssummary{%
%    Habe nun, ach! Philosophie,
%    Juristerei und Medizin,
%    Und leider auch Theologie
%    Durchaus studiert, mit heissem Bem\"uh'n.
%    Da steh ich nun, ich armer Tor!
%    Und bin so klug als wie zuvor;
%    Heisse Magister, heisse Doktor gar,
%    Und ziehe schon an die zehen Jahr
%    Herauf, herab und quer und krumm
%    Meine Sch\"uler an der Nase herum.}
%\end{verbatim}
% 
%    \DescribeMacro{\fsgraphics}
%    Dieser Befehl wird eine Box zwischen \verb|\fssummary| und \verb|\fscontent|
%    setzen mit allf\"alligen Grafiken. In der Gestaltung des Inhalts von \verb|\fsgraphics|
%    ist man relativ frei. Sinnvollerweise kommt hier \verb|\includegraphics| zum Zug.
%    
%    Will man mehrere Grafiken einbinden und nebeneinander anordnen, kann man dies
%    mit \verb|minipage|s bewerkstelligen (siehe Beispiel weiter unten)
%    \footnote{Mehr zu Minipages und ihr manchmal etwas \"uberraschendes Verhalten:\newline
%        \href{https://github.com/alpenwasser/TeX/tree/master/minipages}.
%             {\nolinkurl{https://github.com/alpenwasser/TeX/tree/master/minipages}}}.
%
%    \DescribeMacro{\graphicscaption}
%    Die Legende f\"ur ein Bild. Kann mehrmals verwendet werden im Falle von mehreren
%    Bildern. Sinnvollerweise kurz halten.
%
%    \DescribeMacro{\graphicssource}
%    Quellenangabe f\"ur Bilder. Wird vertikal im rechten Seitenrand platziert. Auch hier
%    liegt die W\"urze in der K\"urze (siehe Beispiel).
%
%    Ein Beispiel zur Anordnung von zwei Bildern entlang der horizontalen Achse:
%\begin{verbatim}
%\fsgraphics{%
%    \begin{minipage}{0.4\textwidth}
%        \includegraphics[height=60mm]{images/bild0.png}
%        \graphicscaption{Dies ist ein Bild}
%    \end{minipage}%
%    \begin{minipage}{0.6\textwidth}
%        \includegraphics[height=60mm]{images/bild1.png}
%        \graphicscaption{Dies ist ein anderes Bild}
%    \end{minipage}
%    \graphicssource{Wikipedia}}
%\end{verbatim}
%
%    \DescribeMacro{\fscontent}
%    Der Hauptinhalt, aufgeteilt in drei Spalten. Der Text kann wie gehabt via
%    \verb|\section| unterteilt werden (\verb|\subsection| und \verb|\subsubsection|
%    sind jedoch deaktiviert).
% 
%    Auch Bilder k\"onnen eingebunden werden, falls man m\"ochte (siehe das
%    separate Beispiel zu \emph{Sonart}). Beispiel:

%\begin{verbatim}    
%\section{Faust -- Nacht}
%\label{sec:faust-night}
%
%Und sehe, daß wir nichts wissen k\"onnen!
%Das will mir schier das Herz verbrennen.
%Zwar bin ich gescheiter als all die Laffen,
%Doktoren, Magister, Schreiber und Pfaffen;
%Mich plagen keine Skrupel noch Zweifel,
%F\"urchte mich weder vor H\"olle noch Teufel -
%Daf\"ur ist mir auch alle Freud´ entrissen,
%Bilde mir nicht ein, was Recht's zu wissen,
%Bilde mir nicht ein, ich k\"onnte was lehren,
%Die Menschen zu bessern und zu bekehren.
%
%\section{Faust -- Mehr Nacht}
%\label{sec:faust-more-night}
%
%Auch hab´ ich weder Gut noch Geld,
%Noch Ehr´ und Herrlichkeit der Welt;
%Es m\"ochte kein Hund so l\"anger leben!
%Drum hab´ ich mich der Magie ergeben,
%Ob mir durch Geistes Kraft und Mund
%Nicht manch Geheimnis w\"urde kund;
%Dass ich nicht mehr mit saurem Schweiss
%Zu sagen brauche, was ich nicht weiss;
%Dass ich erkenne, was die Welt
%Im Innersten zusammenh\"alt,
%Schau´ alle Wirkenskraft und Samen,
%Und tu´ nicht mehr in Worten kramen.
%\end{verbatim}    
%    \DescribeMacro{\infobox}
%    Eine Box, welche links unten auf der Seite platziert wird. Kann
%    grunds\"atzlich mit mehr oder weniger beliebigem Inhalt gef\"ullt werden.
%    Bietet sich gut an, um besondere Highlights oder Eckdaten hervorzuheben.
%    Beispiel:
%\begin{verbatim}
%\infobox{Highlights}{%
%    \footnotesize
%    \setlength\tabcolsep{2pt} % make table a bit more compact horizontally
%    \begin{tabular}{ll}
%    Category 1 & Property A \\
%               & Property B \\
%               & Property C \\
%               & Property D \\
%    Category 2 & Property E \\
%               & Property F \\
%               & Property G \\
%               & Property H \\
%    \end{tabular}}
%\end{verbatim}
%    \DescribeMacro{\includefactsheet}
%    L\"adt ein \verb|Content Document| in einem \verb|Master Document| (siehe
%    Abschnitt~\ref{sec:concept}).
% \fi
%^^A >>>

%^^A ------------------------------------------------------------------------- %
%^^A EXAMPLES <<<
% \ifenglish\section{Examples}\else\section{Beispiele}\fi
% \label{sec:examples}
% \ifenglish
%    The \verb|examples| directory contains a few examples.
% \else
%    Das Verzeichnis \verb|examples| enth\"alt einige Beispiele.
% \fi
%^^A No options, just a few things to pay attention to: ... list of things
%^^A >>>

%^^A ------------------------------------------------------------------------- %
%^^A STYLE GUIDELINES <<<
%^^A \ifenglish\section{Style Guidelines}\else\section{Stil-Richtlinien}\fi
%^^A \label{sec:examples}
%^^A \ifenglish
%^^A    TODO
%^^A \else
%^^A    TODO
%^^A \fi
%^^A No options, just a few things to pay attention to: ... list of things
%^^A >>>
%
%^^A ------------------------------------------------------------------------- %
%^^A IMPLEMENTATION <<<
% \section{Implementation}
% \label{sec:implementation}
%
% \StopEventually{\PrintIndex}
% \ifenglish
% If you are merely intending to use this class, you probably do not need to read this.
% \else
% Normale Benutzer brauchen diesen Abschnitt vermutlich nicht zu lesen.
% \fi
%^^A Need the class guards to prevent the macro code from being included again 
%^^A in template.tex and the German .tex file
%\iffalse
%<*class> 
%\fi
%    \begin{macrocode}
\newif\if@english
\newif\if@german
\DeclareOption{english}{\@englishtrue\@germanfalse}
\DeclareOption{german}{\@germantrue\@englishfalse}
\DeclareOption{deutsch}{\@germantrue\@englishfalse}
\ExecuteOptions{english,german,deutsch}
\ProcessOptions

% Packages
\LoadClass{report}
\RequirePackage[utf8]{inputenc}
\RequirePackage[T1]{fontenc}
\RequirePackage{graphicx}
\RequirePackage{tikz}
\RequirePackage{rviewport}
\RequirePackage{multicol}
\RequirePackage[inline]{enumitem}
\RequirePackage{geometry}
\RequirePackage[scaled=0.95]{helvet}
\renewcommand\familydefault{\sfdefault}
\RequirePackage{xcolor}
\RequirePackage{rotating}
% Define document colos palette
\definecolor{parties}{HTML}{666666}
\definecolor{title}{HTML}{000066}
\definecolor{summary}{HTML}{666666}
\colorlet{boxbg}{cyan!50!blue!20!white}
\RequirePackage{tcolorbox}
\RequirePackage{booktabs}
\RequirePackage[binary-units=true]{siunitx}
\sisetup{range-phrase=--}
\RequirePackage[%
    hidelinks=true,
    bookmarksnumbered=true]{hyperref}
\if@english
    \RequirePackage[english]{babel}
\fi
\if@german
    \RequirePackage[ngerman]{babel}
\fi

% Page Geometry
\geometry{a4paper,twoside=false}
\newlength{\factpageouter}
\newlength{\factpageinner}
\newlength{\factpagetop}
\newlength{\factpagebottom}
\newlength{\factpagewidth}
\newlength{\factgraphicswidth}
\newlength{\testl}
\setlength{\factpageouter}{25mm}
\setlength{\factpageinner}{25mm}
\setlength{\factpagetop}{8mm}
%    \end{macrocode}
%\iffalse
%</class>
%\fi
% Needed so that page numbers do not slide down too far.
%\iffalse
%<*class> 
%\fi
%    \begin{macrocode}
\setlength{\factpagebottom}{30mm}
%    \end{macrocode}
%\iffalse
%</class>
%\fi
% Rest of page geometry layout:
%\iffalse
%<*class> 
%\fi
%    \begin{macrocode}
\setlength{\factpagewidth}{\paperwidth}
\addtolength{\factpagewidth}{-\factpageinner}
\addtolength{\factpagewidth}{-\factpageouter}
\setlength{\factgraphicswidth}{50mm}
\geometry{outer=\factpageouter,inner=\factpageinner,%
          top=\factpagetop,    bottom=\factpagebottom}

\let\newcol\newpage
%    \end{macrocode}
%\iffalse
%</class>
%\fi
% We put the entire  page inside a minipage environment to  make it not take
% up more space than a single page. We  also set the height of that minipage
% to zero so that  it stays on the first page no matter  how much content is
% put inside it.
%\iffalse
%<*class> 
%\fi
%    \begin{macrocode}
\newenvironment{factpage}
    {\noindent\begin{minipage}[t][0pt]{\factpagewidth}}
    {\end{minipage}}
%    \end{macrocode}
%\iffalse
%</class>
%\fi
% Redefine Section Headings  to not be numbered by default. This  is done by
% adding a * after its sixth argument. The rest of the command is similar to
% what can be found in The LaTeX article.cls file.
%\iffalse
%<*class> 
%\fi
%    \begin{macrocode}
\renewcommand\section{\@startsection {section}{1}{\z@}%
                                     {-2.5ex \@plus -1ex \@minus -.2ex}%
                                     {1ex}%
                                     %{1ex \@plus.2ex}% default
                                     {\large\bfseries\color{title}}*}
%    \end{macrocode}
%\iffalse
%</class>
%\fi
% Having subsections  or even  lower-tiered headings in  a document  such as
% this one would be ridiculous. Remove them to make sure they are not used.
%\iffalse
%<*class> 
%\fi
%    \begin{macrocode}
\let\subsection\@undefined
\let\subsubsection\@undefined
\let\paragraph\@undefined
\let\subparagraph\@undefined
%    \end{macrocode}
%\iffalse
%</class>
%\fi
% Define language-specific terms.
%\iffalse
%<*class> 
%\fi
%    \begin{macrocode}
\iflanguage{english}{%
    \newcommand*{\lang@team}    {\textbf{Project Team: }}
    \newcommand*{\lang@client}  {\textbf{Client: }}
    \newcommand*{\lang@coach}   {\textbf{Coach: }}
    \newcommand*{\lang@coaches} {\textbf{Coaches: }}
    \newcommand*{\lang@expert}  {\textbf{Expert: }}
    \newcommand*{\lang@experts} {\textbf{Experts: }}
    \newcommand*{\lang@partner} {\textbf{Industry Partner: }}
    \newcommand*{\lang@partners}{\textbf{Industry Partners: }}
    \newcommand*{\lang@source}  {\textit{Source: }}
}{}
\iflanguage{ngerman}{%
    \newcommand*{\lang@team}    {\textbf{Arbeitsgruppe: }}
    \newcommand*{\lang@client}  {\textbf{Auftraggeber: }}
    \newcommand*{\lang@coach}   {\textbf{Betreuer: }}
    \newcommand*{\lang@coaches} {\textbf{Betreuer: }}
    \newcommand*{\lang@expert}  {\textbf{Experte: }}
    \newcommand*{\lang@experts} {\textbf{Experten: }}
    \newcommand*{\lang@partner} {\textbf{Industriepartner: }}
    \newcommand*{\lang@partners}{\textbf{Industriepartner: }}
    \newcommand*{\lang@source}  {\textit{Quelle: }}
}{}
%    \end{macrocode}
%\iffalse
%</class>
%\fi
% Save those terms in the appropriate macros.
%\iffalse
%<*class> 
%\fi
%    \begin{macrocode}
\newcommand*{\team}[1]
    {\gdef\@team    {{\footnotesize\color{parties}\lang@team#1}}}
\newcommand*{\client}[1]
    {\gdef\@client  {{\footnotesize\color{parties}\lang@client#1}}}
\newcommand*{\coach}[1]
    {\gdef\@coach   {{\footnotesize\color{parties}\lang@coach#1}}}
\newcommand*{\coaches}[1]
    {\gdef\@coaches {{\footnotesize\color{parties}\lang@coaches#1}}}
\newcommand*{\expert}[1]
    {\gdef\@expert  {{\footnotesize\color{parties}\lang@expert#1}}}
\newcommand*{\experts}[1]
    {\gdef\@experts {{\footnotesize\color{parties}\lang@experts#1}}}
\newcommand*{\partner}[1]
    {\gdef\@partner {{\footnotesize\color{parties}\lang@partner#1}}}
\newcommand*{\partners}[1]
    {\gdef\@partners{{\footnotesize\color{parties}\lang@partners#1}}}
%    \end{macrocode}
%\iffalse
%</class>
%\fi
% Redefine \verb|maketitle| to do our bidding:
%\iffalse
%<*class> 
%\fi
%    \begin{macrocode}
\renewcommand\maketitle{\par
    \begingroup
        {\Huge\color{title}\@title}\par
        \vspace{2em}
        \global\let\maketitle\relax
        \global\let\@maketitle\relax
        \global\let\title\relax
        \global\let\date\relax
        \global\let\@title\@empty
        \global\let\@date\@empty
    \endgroup
}
%    \end{macrocode}
%\iffalse
%</class>
%\fi
% For testing purposes:
%\iffalse
%<*class> 
%\fi
%    \begin{macrocode}
\newlength{\placeholderbase}
\newcommand\placeholderpics{%
    \setlength{\placeholderbase}{48mm}
    \begin{tikzpicture}[x=\placeholderbase,y=\placeholderbase,draw=black!50!white,very thick]
        \draw (0,0) -- (1.618,1);
        \draw (0,1) -- (1.618,0);
        \draw (0,0) -- (1.618,0) -- (1.618,1) -- (0,1) -- cycle;
        \node[draw,fill=white] at (0.809,0.5) {\large Beautiful Picture!};
    \end{tikzpicture}
    \begin{tikzpicture}[x=\placeholderbase,y=\placeholderbase,draw=black!50!white,very thick]
        \draw (0,0) -- (1.618,1);
        \draw (0,1) -- (1.618,0);
        \draw (0,0) -- (1.618,0) -- (1.618,1) -- (0,1) -- cycle;
        \node[draw,fill=white] at (0.809,0.5) {\large Amazing Graphics!};
    \end{tikzpicture}\par
    {\footnotesize Beautiful Picture and Amazing Graphics}
}
%    \end{macrocode}
%\iffalse
%</class>
%\fi
% Store the concents of \verb|\fscontent| in \verb|\fs@content|
%\iffalse
%<*class> 
%\fi
%    \begin{macrocode}
\newcommand\fscontent[1]{\gdef\fs@content{#1}}
%    \end{macrocode}
%\iffalse
%</class>
%\fi
% Print the contents of \verb|\fs@content|
%\iffalse
%<*class> 
%\fi
%    \begin{macrocode}
\newcommand\fs@makecontent{\par
    \vspace{1em}\begin{multicols}{3}
        \ifdef{\fs@content}{\fs@content}\par
    \end{multicols}
}
%    \end{macrocode}
%\iffalse
%</class>
%\fi
% Store the concents of \verb|\fssummary| in \verb|\fs@summary|
%\iffalse
%<*class> 
%\fi
%    \begin{macrocode}
\newcommand\fssummary[1]{\gdef\fs@summary{#1}}
%    \end{macrocode}
%\iffalse
%</class>
%\fi
% Print the contents of \verb|\fs@summary|
%\iffalse
%<*class> 
%\fi
%    \begin{macrocode}
\newcommand\fs@makesummary{\par
    \vspace{0.25em}\noindent\begin{minipage}{\textwidth}\large
        \ifdef{\fs@summary}{\color{summary}\fs@summary}\par
    \end{minipage}%
}
%    \end{macrocode}
%\iffalse
%</class>
%\fi
% Store the concents of \verb|\fsgraphics| in \verb|\fs@graphics|
%\iffalse
%<*class> 
%\fi
%    \begin{macrocode}
\newcommand\fsgraphics[1]{\gdef\fs@graphics{#1}}
\newcommand\graphicssource[1]{\gdef\fs@graphicssource{#1}}
%    \end{macrocode}
%\iffalse
%</class>
%\fi
% Put the contents of \verb|\graphicscaption| inside an \verb|\mbox| so that it does not wrap.
%\iffalse
%<*class> 
%\fi
%    \begin{macrocode}
\newcommand\graphicscaption[1]{\mbox{#1}}
%    \end{macrocode}
%\iffalse
%</class>
%\fi
% Print the contents of \verb|\fs@graphics|
%\iffalse
%<*class> 
%\fi
%    \begin{macrocode}
\newcommand\fs@makegraphics{\par
    \vspace{1em}\noindent\begin{minipage}[b]{\textwidth}%
        \footnotesize\raggedright
        \ifdef{\fs@graphics}{\fs@graphics}\par\par
        \ifdef{\fs@graphicscaption}{\fs@graphicscaption}\par
    \end{minipage}%
    \begin{minipage}[t]{2ex}\footnotesize\color{parties}
        \ifdef{\fs@graphicssource}{%
            \hspace*{2ex}%
            \turnbox{90}{%
                \hspace*{2ex} \lang@source \fs@graphicssource}}\par
    \end{minipage}%
}
%    \end{macrocode}
%\iffalse
%</class>
%\fi
% Store the content of \verb|\infobox| and print it:
%\iffalse
%<*class> 
%\fi
%    \begin{macrocode}
\newcommand\infobox[2]{%
    \gdef\fs@infoboxtitle{#1}%
    \gdef\fs@infoboxcontent{\footnotesize #2}%
}
\newcommand\fs@makeinfobox{
    \begin{tcolorbox}[
            title=\fs@infoboxtitle,
            width=0.66\textwidth,
            arc=0mm,
            before=,
            after=\hfill]
        \fs@infoboxcontent
    \end{tcolorbox}
}
\newcommand\tightlist{%
    \setlength{\itemsep}{0pt}\setlength{\parskip}{0pt}%
}
%    \end{macrocode}
%\iffalse
%</class>
%\fi
% Errors:
%\iffalse
%<*class> 
%\fi
%    \begin{macrocode}
\newcommand\fs@checkinput{%
    \ifdef{\@coach}{%
        \ifdef{\@coaches}{%
            \ClassError{fhnwfactsheet}{Both 'coach' and 'coaches' field defined. Undefine one.}
                {Both the 'coach' and the 'coaches' field are defined. Only one of them should be.}
        }{}%
    }{}%
    \ifdef{\@expert}{%
        \ifdef{\@experts}{%
            \ClassError{fhnwfactsheet}{Both 'expert' and 'experts' field defined. Undefine one.}
                {Both the 'expert' and the 'experts' field are defined. Only one of them should be.}
        }{}%
    }{}%
    \ifdef{\@partner}{%
        \ifdef{\@partners}{%
            \ClassError{fhnwfactsheet}{Both 'partner' and 'partners' field defined. Undefine one.}
                {Both the 'partner' and the 'partners' field are defined. Only one of them should be.}
        }{}%
    }{}%
    \ifdef{\fs@infoboxcontent}{}
        {%
            \ClassError{fhnwfactsheet}{Did you forget to specify your infobox?}
            {The infobox contents are not defined. Make sure your infobox command is properly called.}
        }%
    \ifdef{\fs@infoboxtitle}{}
        {%
            \ClassError{fhnwfactsheet}{Did you forget to specify your infobox?}
            {The infobox title is not defined. Make sure your infobox command is properly called.}
        }%
}
%    \end{macrocode}
%\iffalse
%</class>
%\fi
% Chapter code. Taken mostly from \verb|classes.dtx|.
%\iffalse
%<*class> 
%\fi
%    \begin{macrocode}
\renewcommand\chapter{%\if@openright\cleardoublepage\else\clearpage\fi
                    \thispagestyle{plain}%
                    \global\@topnum\z@
                    \@afterindentfalse
                    \secdef\@chapter\@schapter}
\def\@chapter[#1]#2{\ifnum \c@secnumdepth >\m@ne
                         \refstepcounter{chapter}%
                         \typeout{\@chapapp\space\thechapter.}%
                         \addcontentsline{toc}{chapter}{#1}%
                                   %{\protect\numberline{\thechapter}#1}%
                    \else
                      \addcontentsline{toc}{chapter}{#1}%
                    \fi
                    \chaptermark{#1}%
                    \addtocontents{lof}{\protect\addvspace{10\p@}}%
                    \addtocontents{lot}{\protect\addvspace{10\p@}}%
                    \if@twocolumn
                      \@topnewpage[\@makechapterhead{#2}]%
                    \else
                      \@makechapterhead{#2}%
                      \@afterheading
                    \fi}
\def\@makechapterhead#1{%
    \begingroup
        \mbox{\Huge\color{title}#1}\par
        \vspace{2em}
    \endgroup
}
\def\@makeschapterhead#1{%
  \vspace*{50\p@}%
  {\parindent \z@ \raggedright
    \normalfont
    \interlinepenalty\@M
    \Huge\color{title} #1\par\nobreak
    \vskip 40\p@
  }}
\renewcommand*\l@chapter[2]{%
  \ifnum \c@tocdepth >\m@ne
    \addpenalty{-\@highpenalty}%
    \vskip 1.0em \@plus\p@
    \setlength\@tempdima{1.5em}%
    \begingroup
      \parindent \z@ \rightskip \@pnumwidth
      \parfillskip -\@pnumwidth
      \leavevmode 
      \advance\leftskip\@tempdima
      \hskip -\leftskip
      %\@dottedtocline{0}{0em}{0em}{#1}{#2}\par
      #1\nobreak\hfil \nobreak\hb@xt@\@pnumwidth{\hss #2}\par
      \penalty\@highpenalty
    \endgroup
  \fi}
%    \end{macrocode}
%\iffalse
%</class>
%\fi
% Including a fact sheet content file in a master document:
%\iffalse
%<*class> 
%\fi
%    \begin{macrocode}
\newcommand*\includefactsheet[1]{\input{#1}\makefactsheet\newpage}
%    \end{macrocode}
%\iffalse
%</class>
%\fi
% Output routine:
%\iffalse
%<*class> 
%\fi
%    \begin{macrocode}
\newcommand\makefactsheet{
    \begin{factpage}
        \fs@checkinput
        \chapter{\@title}
        % Must be after \chapter
        % Note: We do not want to overwrite the \chapter pagestyle{plain} everywhere.
        \thispagestyle{empty}
        \fs@makesummary
        \vspace{1ex}
        \fs@makegraphics
        \vspace{1ex}
        \fs@makecontent
        \vspace{1ex}
        \tcbset{
            colback=boxbg,
            colframe=title,
            boxrule=0.5pt,
            toptitle=2pt,
            bottomtitle=1pt,
            titlerule=0.5pt,
        }
        \fs@makeinfobox
        \begin{minipage}[b]{0.30\textwidth}
            \ifdef{\@team}{\@team}{}\par
            \ifdef{\@client}{\@client}{}\par
            \ifdef{\@coach}{\@coach}{}\par
            \ifdef{\@coaches}{\@coaches}{}\par
            \ifdef{\@expert}{\@expert}{}\par
            \ifdef{\@experts}{\@experts}{}\par
            \ifdef{\@partner}{\@partner}{}\par
            \ifdef{\@partners}{\@partners}{}\par
        \end{minipage}
    \end{factpage}\raisebox{6ex}{\hspace{3em}\thepage}
    % Undefine for next fact sheet
    \let\fs@graphics\@undefined
    \let\fs@graphicssource\@undefined
    \let\fs@graphicscaption\@undefined
    \let\@team\@undefined
    \let\@client\@undefined
    \let\@coach\@undefined
    \let\@coaches\@undefined
    \let\@expert\@undefined
    \let\@experts\@undefined
    \let\@partner\@undefined
    \let\@partners\@undefined
    \let\fs@infoboxtitle\@undefined
    \let\fs@infoboxcontent\@undefined
}
%    \end{macrocode}
%\iffalse
%</class>
%\fi
%
% \Finale
%^^A >>>
%
%^^A ------------------------------------------------------------------------- %
%^^A TEMPLATE <<<
% \iffalse
%<*template>
\documentclass{fhnwfactsheet}
\usepackage{lipsum}
%\usepackage{FiraSans}

%% ================================================================= SET TITLE %
\title{Phenomenal Title!}
\author{Glorious Author!}

%% ============================================================== END PREAMBLE %
\begin{document}
\begin{factpage}
%% ============================================================= BEGIN CONTENT %

%% ------------------------------------------------------------------ OVERVIEW %
\begin{overview}
    \lipsum[2]
\end{overview}

\begin{content}

    \section{This is a section}
    \lipsum[2]
    \section{This is another section}
    \lipsum[4]
    \section{And another one!}
    \lipsum[5]

\end{content}

\lipsum[1]

\factgraphics{a-stunning-picture.png}{a-very-nice-graphic.png}

%% ------------------------------------------------------------------ STUDENTS %
\begin{students}
    \lipsum[2]
\end{students}

%% ------------------------------------------------------------------- COACHES %
\begin{coaches}
    \lipsum[2]
\end{coaches}

%% ----------------------------------------------------------------- PARTNERS %
\begin{partners}
    \lipsum[2]
\end{partners}

%% ============================================================== END CONTENT %
\end{factpage}
\end{document}
%</template>
% \fi
%^^A >>>
%
%^^A ------------------------------------------------------------------------- %
%^^A MULTI-LANGUAGE SUPPORT <<<
% \iffalse
%<*de>
\expandafter\newif\csname ifenglish\endcsname
\englishfalse
% \iffalse meta-comment
% -------------------------------------------------------- %
% Copyright (c) 2017 Raphael Frey rmfrey@runbox.com        %
% -------------------------------------------------------- %
%                                                          %
% This file  may be distributed and/or  modified under the %
% conditions of  the LaTeX Project Public  License, either %
% version 1.3c  of this  license or  (at your  option) any %
% later version. The latest version of this license is in: %
%                                                          %
%     http://www.latex-project.org/lppl.txt                %
%                                                          %
% -------------------------------------------------------- %
% \fi
%
%\CheckSum{615}
%
%
%^^A ------------------------------------------------------------------------- %
%^^A DRIVER, PROVIDESCLASS ETC. <<<
% \iffalse
%<*driver>
\ProvidesFile{fhnwfactsheet.dtx}
%</driver>
%<class>\NeedsTeXFormat{LaTeX2e}
%<class>\ProvidesClass{fhnwfactsheet}%
%<*class>
    [2017/07/12 v1.0.0 PDF-LaTeX class for FHNW fact sheets]
%</class>
%<*driver>
\expandafter\ifx\csname ifenglish\endcsname\relax
    \expandafter\newif\csname ifenglish\endcsname
    \englishtrue
\fi
\documentclass[a4paper,11pt,oneside]{ltxdoc}
\EnableCrossrefs
\CodelineIndex
\RecordChanges
% ----------------------------------------------- PACKAGES %
\ifenglish
    \usepackage[english]{babel}
\else
    \usepackage[ngerman]{babel}
\fi
\usepackage{xcolor-solarized}
\usepackage[%
    bookmarksnumbered=true,
    colorlinks=true,
    linkcolor=cyan!50!blue,
    citecolor=violet,
    urlcolor=purple]{hyperref}
\usepackage[light,nott]{kpfonts}
\usepackage[scaled=0.85]{DejaVuSansMono}
\usepackage{dirtree}
% ------------------------------------------ HELPER MACROS %
\def\pacname#1{\textsf{#1}}
\def\fhnwfs{\pacname{fhnwfactsheet}}
\def\fhnwenfull{University of Applied Sciences and Arts Northwestern Switzerland}
\def\fhnwdefull{Fachhochschule Nordwestschweiz}
\def\fhnw{FHNW}
\def\hten{School of Engineering}
\def\htde{Hochschule f\"ur Technik}
\def\code#1{\texttt{#1}}
\begin{document} % ****************************** DOCUMENT %
    \DocInput{fhnwfactsheet.dtx}
\end{document}
%</driver>
% \fi
%^^A >>>
%
%^^A ------------------------------------------------------------------------- %
%^^A TITLE, ABSTRACT, ToC <<<
% \GetFileInfo{fhnwfactsheet.dtx}
% \author{Raphael Frey \\[0.5em] \href{https://github.com/alpenwasser/fhnwfactsheet/}%
%                                     {\nolinkurl{https://github.com/alpenwasser/fhnwfactsheet/}}}
%
% \ifenglish
%   \title{The \fhnwfs\ Class\thanks{This document corresponds
%       to \fhnwfs~\fileversion, dated~\filedate.}}
%   \maketitle
%   \begin{abstract}
%        The  \fhnwfs\  class  offers  a   simple  interface  for  composing  a
%        fact  sheet for  projects  in A4  paper size  for  the Degree  Program
%        \emph{Electrical  and Information  Technology}  at the  \hten\ of  the
%        \fhnwenfull.
%
%        This  document  contains  the  documentation for  user  interface  and
%        describes the implementation  for any developer who  intends to modify
%        the class.
%   \end{abstract}
% \else
%   \title{Die \fhnwfs-Klasse\thanks{Dieses Dokument entspricht
%          \fhnwfs~\fileversion, datiert~\filedate.}}
%   \maketitle
%   \begin{abstract}
%        Die  \fhnwfs\ Klasse  dient zum  Erstellen von  Factsheets  f\"ur den
%        Studiengang \emph{Elektro- und Informationstechnik}  an der \htde\ der
%        \fhnwdefull.
%
%        Dieses    Dokument    enth\"alt    die   Dokumentation    f\"ur    das
%        Benutzer-Interface  und   beschreibt  die  Implementation   f\"ur  die
%        geneigte Engwicklerin, welche die Klasse modifizieren m\"ochte.
%   \end{abstract}
% \fi
%
% \tableofcontents
%^^A >>>
%
%^^A ------------------------------------------------------------------------- %
%^^A Basic Concept <<<
% \ifenglish
%    \section{Basic Concept}
%    \label{sec:concept}
%    The class  allows to build indivitual  standalone fact sheets as  well as
%    a  larger  document containing  multiple  fact  sheets as  a  collection,
%    including a title page, a table of contents etc. The underlying structure
%    in both scenarios is identical:
%    \dirtree{%^^A
%        .1 Master Document.
%        .2 Content Document.
%        .3 Further Ressources (e.g.~images).
%    }
%    \emph{Note:} 
%    \begin{itemize}
%        \item
%        \textsf{fhnwfactsheet} does not enforce any specifif file name
%        specification. Follow your teacher's instructions if you are creating a fact
%        sheet for submission to the school.
%        \item
%        Do not load additional packages! This will mess up the process of merging your
%        standalone fact sheet with others.
%        \textsf{fhnwfactsheet}loads most classes one would reasonably need to create
%        a fact sheet. If there really is a package which you absolutely, positively
%        need, contact the package maintainer and he will see if it can be integrated
%        into the class. This will then allow to push the class to everyone through an
%        update and things should (hopefully) not break.
%    \end{itemize}
%
%    \subsection{Master Document}
%    \label{subsec:masterdoc}
%    The master document is the  document on which you run \verb|pdflatex|. It
%    sets up the class and its  configuration and then loads the \verb|Content Document|.
%    If you are working on a  standalone fact sheet, this file will
%    be very minimal:
% \else
%    \section{Grundkonzept}
%    \label{sec:concept}
%    Die  Klasse   erlaubt  sowohl  das  Erstellen   von  einzelnen  Factsheets
%    als  eigenst\"andige  Dokumente  wie  auch das  Erstellen  einer  Sammlung
%    von   Factsheets   in   einerm  gr\"osseren   Dokument   mit   Titelseite,
%    Inhaltsverzeichnis etc. Die grundlegende Struktur ist dabeit identisch:
%    \dirtree{%^^A
%        .1 Master Document.
%        .2 Content Document.
%        .3 Weitere Ressourcen (z.B.~Bilder).
%    }
%
%    \emph{Beachte:} 
%    \begin{itemize}
%        \item
%        \textsf{fhnwfactsheet} erzwingt keine spezielle Dateinamen
%        oder Ordnerstruktur. Man beachte die Anweisung der Lehrpersonals diesbez\"uglich
%        wenn man ein Factsheet zur Einreichung bei der Schule erstellt.
%        \item
%        Es sollten keine zus\"atzlichen Packages geladen werden. Dies wird den Prozess
%        des Zusammenstellens verschiedener eigenst\"andiger Factsheets st\"oren.
%        \textsf{fhnwfactsheet} l\"adt die meisten Pakete, welche man \"ublicherweise
%        zur Erstellung eines Factsheets ben\"otigt. Falls wirklich ein Paket nicht
%        geladen ist, welches absolute, definitiv, sicherlich ben\"otigt wird, kontaktiere
%        man den Package-Maintainer (siehe erste Seite). Er wird sehen, ob sich das
%        zus\"atzliche Paket in die Klasse integrieren l\"asst, und falls ja, wird
%        das entsprechende Update zur Verf\"ugung stellen. Dies
%        wird hoffentlich einen problemlosen Arbeitsprozess f\"ur alle beteiligten
%        Parteien erm\"oglichen.
%    \end{itemize}
%
%    \subsection{Master Document}
%    \label{subsec:masterdoc}
%    Das  Master-Dokument  ist  das Dokument,  welches  durch  \verb|pdflatex|
%    verarbeitet  wird. Es  richtet  die  Klasse  und  Konfiguration  ein  und
%    l\"adt  anschliessend  den  eigentlichen  Inhalt  aus  dem  \verb|Content Document|.
%    Wenn ein eigenst\"andiges Factsheet  erstellt wird, ist dieses
%    Dokument ziemlich minimalistisch:
% \fi
%\begin{verbatim}
%\documentclass[german]{fhnwfactsheet}
%\begin{document}
%\includefactsheet{contentDocument.tex}
%\end{document}
%\end{verbatim}
% \ifenglish
%     Multiple \verb|Content Document|s can be included if you wish to create a collection
%     of fact sheets. Each fact sheet will be put on a single page. This also enables
%     the creation of a table of contents.
% \else
%    Mehrere \verb|Content Document|s k\"onnen geladen werden, wenn man eine Sammlung von
%    Factsheets erstellen m\"ochte. Jedes Factsheet wird auf einer einzelnen Seite
%    erstellt werden. Ein Inhaltsverzeichnis kann ebenfalls erstellt werden.
% \fi
%\begin{verbatim}
%\documentclass[german]{fhnwfactsheet}
%\begin{document}
%\pagenumbering{roman}
%\tableofcontents
%
%\newpage\pagenumbering{arabic}
%
%\includefactsheet{factsheet1Content.tex}
%\includefactsheet{factsheet2Content.tex}
%\includefactsheet{factsheet3Content.tex}
%\includefactsheet{factsheet4Content.tex}
%\end{document}
%\end{verbatim}
% \ifenglish
%    \subsection{Content Document}
%    \label{subsec:contentdoc}
%    The content document is where the content of your fact sheet actually resides.
%    it offers some specifics commands to enable consistent layout and typography
%    and places some restrictions which should prevent things from getting too
%    extravagant. See the \verb|examples| directory for some example code.
% \else
%    \subsection{Content Document}
%    \label{subsec:contentdoc}
%    Das \verb|Content Document| ist das Dokument, wo der eigentliche Inhalt
%    untergebracht ist. Es offeriert einige spezifische Befehle, um konsistentes
%    Layout und Typographie zu erm\"oglichen und setzt einige Restriktionen, um
%    allzu extravagante Ideen zu verhindern (oder zumindest zu erschweren).
%    Beispiele k\"onnen im Verzeichnis  \verb|examples| gefunden werden.
% \fi
%^^A >>>
%^^A ------------------------------------------------------------------------- %
%^^A INSTALLATION <<<
% \section{Installation}
% \label{sec:installation}
%
% \ifenglish
%    From \verb|fhnwfactsheet.dtx| and \verb|fhnwfactsheet.ins|, generate the class file \verb|fhnwfactsheet.cls|
%    by running
%\begin{verbatim}
%pdflatex fhnwfactsheet.ins
%\end{verbatim}
%    This will create the class file. If you wish to compile the documentation
%    (this document), run:
%\begin{verbatim}
%pdflatex fhnwfactsheet.dtx
%\end{verbatim}
%    After that, place \verb|fhnwfactsheet.cls| in your working directory and
%    enable it via the \verb|documentclass| command (see example for master document
%    above).
%
%    ^^A\inputminted[linenos=true,numbersep=4pt,fontsize=\footnotesize]{tex}{template.tex}
%   ^^A TODO: File structure tree with the needed files.
% \else
%    Aus \verb|fhnwfactsheet.dtx| und \verb|fhnwfactsheet.ins| kann das Klassen-Dokument
%    \verb|fhnwfactsheet.cls| generiert werden via:
%\begin{verbatim}
%pdflatex fhnwfactsheet.ins
%\end{verbatim}
%    Zum Kompilieren der Dokumentation (dieses Dokument):
%\begin{verbatim}
%pdflatex fhnwfactsheet.dtx
%\end{verbatim}
%    Anschliessend \verb|fhnwfactsheet.cls| in das Verzeichnis des Factsheet-Projekts
%    kopieren und via \verb|documentclass| laden (siehe Master Document oben).
% \fi
%^^A >>>
%
%^^A CLASS OPTIONS <<<
%
% \ifenglish
%    \section{Class Options}
%    \label{sec:options}
%
%    \DescribeMacro{english}
%    \DescribeMacro{german}
%    \DescribeMacro{deutsch}
%    These class options will load the corresponding \textsf{babel} language
%    as well as set some key words accordingly. If you don't specify any
%    language, the class defaults to German.
%
%    \noindent Examples:
% \else
%    \section{Klassen-Optionen}
%    \label{sec:options}
%
%    \DescribeMacro{english}
%    \DescribeMacro{german}
%    \DescribeMacro{deutsch}
%    Diese Klassne-Optionen laden die entsprechenden Konfigurationen in \textsf{babel}
%    und definieren einige Schl\"usselw\"orter in der jeweiligen Sprache. Wenn keine
%    Sprache spezifiziert wird, l\"adt die Klasse auf deutsch.
%    
%    \noindent Beispiele:
% \fi
%\begin{verbatim}
%\documentclass{fhnwfactsheet}          % German
%\documentclass[deutsch]{fhnwfactsheet} % German
%\documentclass[german]{fhnwfactsheet}  % German
%\documentclass[english]{fhnwfactsheet} % English
%\end{verbatim}
%^^A >>>

%^^A COMMANDS <<<
%
% \ifenglish
%    \section{Commands}
%    \label{sec:commands}
%
%    This section describes the commands which are implemented by \textsf{fhnwfactsheet}.
%
%    \subsection{Language-Specific Macros}
%    \label{subsec:language-specific}
%    These macros are used to name the people involved in the project as well as their
%    roles. None of these have to be specified, if they did not exist. They will be used
%    when defined, otherwise not.
%
%    \DescribeMacro{\team}
%    Comma-separated list of team members. \emph{Note:} Use \verb|Firstname Lastname|
%    notation to ensure consistency. Example:
%\begin{verbatim}
%\team{%
%    Team Leader,
%    Personwho Didallthework,
%    Personwho Wasalsothere,
%    Dudewho Wasnotthere}
%\end{verbatim}
%    \DescribeMacro{\client}
%    The client. The format is identical to the \verb|\team| command.
%
%    \DescribeMacro{\coach}
%    \DescribeMacro{\coaches}
%    The Project coaches. In case of a single person, use \verb|\coach|, in
%    case of multiple people, use \verb|\coaches|. \emph{Note:} Since these
%    options are mutually exclusive, the class will raise an error if both
%    macros have been specified.
%
%    \DescribeMacro{\expert}
%    \DescribeMacro{\experts}
%    Experts, for example in case of a thesis. Same as above, the class will
%    raise an error in case of both singular and plural form being specified.
%
%    \DescribeMacro{\partner}
%    \DescribeMacro{\partners}
%    Industry partners, if there were any (and they can be named).
%
%    \subsection{Content Generation}
%    \label{subsec:content-generation}
%    The following macros are used to create the content which goes onto the 
%    fact sheet page.
%    \DescribeMacro{\title}
%    The title for the page. This will also be used as the title in the table
%    of contents in case of a collection document (internally it is automagically
%    mapped to a \verb|\chapter|).
%\begin{verbatim}
%\title{Grand Title of Significant Magnitude}
%\end{verbatim}
%    \emph{Note:} The title will never wrap onto a new line. If it is too long for
%    a single line, it will simply run off the page. This is by design. Shorten
%    your title in that case.
%
%    \DescribeMacro{\fssummary}
%    This is where the introductory header text goes. Ideally, it should not be longer
%    than three or maximum four lines. However, the macro itself does not enforce
%    this. Example\footnote{Goethe's Faust -- Nacht; this will result in 3.5 lines.}
%    Hier kommt der Einleitungstext hinein. Idealerweise sollte die Einleitung nicht
%    l\"anger als drei bis maximal vier Zeilen sein. Das Macro erzwingt diese L\"ange
%    allerdings nicht. Beispiel
%\begin{verbatim}
%\fssummary{%
%    Habe nun, ach! Philosophie,
%    Juristerei und Medizin,
%    Und leider auch Theologie
%    Durchaus studiert, mit heißem Bemüh'n.
%    Da steh ich nun, ich armer Tor!
%    Und bin so klug als wie zuvor;
%    Heiße Magister, heiße Doktor gar,
%    Und ziehe schon an die zehen Jahr
%    Herauf, herab und quer und krumm
%    Meine Schüler an der Nase herum.}
%\end{verbatim}
%
%    \DescribeMacro{\fsgraphics}
%    This command creates a box which is put between \verb|\fssummary| and \verb|\fscontent|
%    with the intent of being filled with graphical content. There are not really any
%    restrictions placed on the graphics, so treat with care. Usually, one would probably
%    want to use \verb|\includegraphics| here.

%    For including multiple graphics, the use of \verb|minipage|s is advised.
%    \footnote{More about minipages and their sometimes surprising behavior:\newline
%        \href{https://github.com/alpenwasser/TeX/tree/master/minipages}.
%             {\nolinkurl{https://github.com/alpenwasser/TeX/tree/master/minipages}}}.
%
%    \DescribeMacro{\graphicscaption}
%    The caption for a picture. Can be used multiple times in case of multiple pictures.
%    Should be kept short.
%
%    \DescribeMacro{\graphicssource}
%    A source for the pictures. Is placed vertically in the right page margin. This, too
%    should be kept short (see example below).
%
%    An example for including two images along the horizonal axis:
%\begin{verbatim}
%\fsgraphics{%
%    \begin{minipage}{0.4\textwidth}
%        \includegraphics[height=60mm]{images/bild0.png}
%        \graphicscaption{Dies ist ein Bild}
%    \end{minipage}%
%    \begin{minipage}{0.6\textwidth}
%        \includegraphics[height=60mm]{images/bild1.png}
%        \graphicscaption{Dies ist ein anderes Bild}
%    \end{minipage}
%    \graphicssource{Wikipedia}}
%\end{verbatim}
%
%    \DescribeMacro{\fscontent}
%    The main content of your document, split into three columns. Partition via
%    the \verb|\section| command as usual (there are no \verb|subsection|s or 
%    even \verb|\subsubsection|s available).
%
%    You can also include graphics here, should you feel so inclined (see the
%    separate \emph{Sonart} example fact sheet). Example:
%  
%\begin{verbatim}    
%\section{Faust -- Nacht}
%\label{sec:faust-night}
%Und sehe, daß wir nichts wissen können!
%Das will mir schier das Herz verbrennen.
%Zwar bin ich gescheiter als all die Laffen,
%Doktoren, Magister, Schreiber und Pfaffen;
%Mich plagen keine Skrupel noch Zweifel,
%Fürchte mich weder vor Hölle noch Teufel -
%Dafür ist mir auch alle Freud´ entrissen,
%Bilde mir nicht ein, was Recht's zu wissen,
%Bilde mir nicht ein, ich könnte was lehren,
%Die Menschen zu bessern und zu bekehren.
%
%\section{Faust -- Mehr Nacht}
%\label{sec:faust-more-night}
%Auch hab´ ich weder Gut noch Geld,
%Noch Ehr´ und Herrlichkeit der Welt;
%Es möchte kein Hund so länger leben!
%Drum hab´ ich mich der Magie ergeben,
%Ob mir durch Geistes Kraft und Mund
%Nicht manch Geheimnis würde kund;
%Daß ich nicht mehr mit saurem Schweiß
%Zu sagen brauche, was ich nicht weiß;
%Daß ich erkenne, was die Welt
%Im Innersten zusammenhält,
%Schau´ alle Wirkenskraft und Samen,
%Und tu´ nicht mehr in Worten kramen.
%\end{verbatim}    
%    \DescribeMacro{\infobox}
%    A box which gets placed on the left side at the page bottom. Can
%    fundamentally contain pretty much any content you like. Well-suited
%    for highlighing key points or specifications about the project.
%    Example:
%\begin{verbatim}
%\infobox{Highlights}{%
%    \footnotesize
%    \setlength\tabcolsep{2pt} % make table a bit more compact horizontally
%    \begin{tabular}{ll}
%    Category 1 & Property A \\
%               & Property B \\
%               & Property C \\
%               & Property D \\
%    Category 2 & Property E \\
%               & Property F \\
%               & Property G \\
%               & Property H \\
%    \end{tabular}}
%\end{verbatim}
%
%    \DescribeMacro{\includefactsheet}
%    Loads a \verb|Content Document| in a \verb|Master Document| (see
%    Section~\ref{sec:concept}).
%
% \else
%    \section{Befehle}
%    \label{sec:commands}
%    Dieser Abschnitt beschreibt die Macros, welche von \textsf{fhnwfactsheet} implementiert werden.
%
%    \subsection{Sprachspezifische Macros}
%    \label{subsec:language-specific}
%    Diese Macros werden dazu verwendet,  die im Project involvierten Personen
%    und  ihre  Rollen zu  beschreiben. Rollen,  welche  nicht besetzt  waren,
%    k\"onnen weggelassen  werden. Keiner dieser Befehle muss  benutzt werden,
%    damit  das  Dokument kompiliert  werden  kann. Sie  werden benutzt,  wenn
%    definiert, andernfalls nicht.
%
%    \DescribeMacro{\team}
%    Komma-getrennte   Liste  der   Gruppenmitglieder.  \emph{Beachte:} Zwecks
%    Einhaltung     von      nsistenz     benutze      man     \verb|Vorname Nachname|-Notation. 
%    Beispiel:
%\begin{verbatim}
%\team{%
%    Team Leader,
%    Personwho Didallthework,
%    Personwho Wasalsothere,
%    Dudewho Wasnotthere}
%\end{verbatim}
%    \DescribeMacro{\client}
%    Auftraggeber. Das Format ist identisch zum \verb|\team|-Befehl.
%
%    \DescribeMacro{\coach}
%    \DescribeMacro{\coaches}
%    Die Fachcoaches. Im Falle einer Einzelperson benutze man \verb|\coach|,
%    im Falle von mehreren Personen \verb|\coaches|. \emph{Beachte:} Da diese
%    beiden Varianten sich gegenseitig ausschliessen, wird die Klasse einen
%    Fehler ausl\"osen, wenn beide spezifiziert worden sind.
%    
%    \DescribeMacro{\expert}
%    \DescribeMacro{\experts}
%    Allf\"allig beigezogene Experten (z.B. im Falle einer Thesis.) Auch hier
%    gilt die gleiche Regel betreffend Singular/Plural und Fehler-Ausl\"osung.
%
%    \DescribeMacro{\partner}
%    \DescribeMacro{\partners}
%    Industriepartner, falls vorhanden und falls sie genannt werden d\"urfen.
%
%    \subsection{Dokumenten-Inhalt}
%    \label{subsec:content-generation}
%    \DescribeMacro{\title}
%    Der Titel der Seite. Dieser Text wird auch f\"ur das Inhaltsverzeichnis verwendet
%    im Falle eines Sammlungs-Dokuments (intern wird er automagisch in ein \verb|\chapter|
%    gespeichert).
%\begin{verbatim}
%\title{Grandioser Title von einiger Wichtigkeit}
%\end{verbatim}
%    \emph{Beachte:}  Der Titel wird niemals einen Zeilenumbruch machen. Falls er
%    zu lange f\"ur eine einzelne Zeile ist, wird er einfach in den rechten Seitenrand
%    laufen. Dies ist Absicht. Man k\"urze in diesem Falle den Titel entsprechend.
%
%    \DescribeMacro{\fssummary}
%    Hier kommt der Einleitungstext hinein. Idealerweise sollte die Einleitung nicht
%    l\"anger als drei bis maximal vier Zeilen sein. Das Macro erwzingt diese L\"ange
%    allerdings nicht. Beispiel\footnote{Goethe's Faust -- Nacht; dies ergibt 3.5 Zeilen.}:
%\begin{verbatim}
%\fssummary{%
%    Habe nun, ach! Philosophie,
%    Juristerei und Medizin,
%    Und leider auch Theologie
%    Durchaus studiert, mit heissem Bem\"uh'n.
%    Da steh ich nun, ich armer Tor!
%    Und bin so klug als wie zuvor;
%    Heisse Magister, heisse Doktor gar,
%    Und ziehe schon an die zehen Jahr
%    Herauf, herab und quer und krumm
%    Meine Sch\"uler an der Nase herum.}
%\end{verbatim}
% 
%    \DescribeMacro{\fsgraphics}
%    Dieser Befehl wird eine Box zwischen \verb|\fssummary| und \verb|\fscontent|
%    setzen mit allf\"alligen Grafiken. In der Gestaltung des Inhalts von \verb|\fsgraphics|
%    ist man relativ frei. Sinnvollerweise kommt hier \verb|\includegraphics| zum Zug.
%    
%    Will man mehrere Grafiken einbinden und nebeneinander anordnen, kann man dies
%    mit \verb|minipage|s bewerkstelligen (siehe Beispiel weiter unten)
%    \footnote{Mehr zu Minipages und ihr manchmal etwas \"uberraschendes Verhalten:\newline
%        \href{https://github.com/alpenwasser/TeX/tree/master/minipages}.
%             {\nolinkurl{https://github.com/alpenwasser/TeX/tree/master/minipages}}}.
%
%    \DescribeMacro{\graphicscaption}
%    Die Legende f\"ur ein Bild. Kann mehrmals verwendet werden im Falle von mehreren
%    Bildern. Sinnvollerweise kurz halten.
%
%    \DescribeMacro{\graphicssource}
%    Quellenangabe f\"ur Bilder. Wird vertikal im rechten Seitenrand platziert. Auch hier
%    liegt die W\"urze in der K\"urze (siehe Beispiel).
%
%    Ein Beispiel zur Anordnung von zwei Bildern entlang der horizontalen Achse:
%\begin{verbatim}
%\fsgraphics{%
%    \begin{minipage}{0.4\textwidth}
%        \includegraphics[height=60mm]{images/bild0.png}
%        \graphicscaption{Dies ist ein Bild}
%    \end{minipage}%
%    \begin{minipage}{0.6\textwidth}
%        \includegraphics[height=60mm]{images/bild1.png}
%        \graphicscaption{Dies ist ein anderes Bild}
%    \end{minipage}
%    \graphicssource{Wikipedia}}
%\end{verbatim}
%
%    \DescribeMacro{\fscontent}
%    Der Hauptinhalt, aufgeteilt in drei Spalten. Der Text kann wie gehabt via
%    \verb|\section| unterteilt werden (\verb|\subsection| und \verb|\subsubsection|
%    sind jedoch deaktiviert).
% 
%    Auch Bilder k\"onnen eingebunden werden, falls man m\"ochte (siehe das
%    separate Beispiel zu \emph{Sonart}). Beispiel:

%\begin{verbatim}    
%\section{Faust -- Nacht}
%\label{sec:faust-night}
%
%Und sehe, daß wir nichts wissen k\"onnen!
%Das will mir schier das Herz verbrennen.
%Zwar bin ich gescheiter als all die Laffen,
%Doktoren, Magister, Schreiber und Pfaffen;
%Mich plagen keine Skrupel noch Zweifel,
%F\"urchte mich weder vor H\"olle noch Teufel -
%Daf\"ur ist mir auch alle Freud´ entrissen,
%Bilde mir nicht ein, was Recht's zu wissen,
%Bilde mir nicht ein, ich k\"onnte was lehren,
%Die Menschen zu bessern und zu bekehren.
%
%\section{Faust -- Mehr Nacht}
%\label{sec:faust-more-night}
%
%Auch hab´ ich weder Gut noch Geld,
%Noch Ehr´ und Herrlichkeit der Welt;
%Es m\"ochte kein Hund so l\"anger leben!
%Drum hab´ ich mich der Magie ergeben,
%Ob mir durch Geistes Kraft und Mund
%Nicht manch Geheimnis w\"urde kund;
%Dass ich nicht mehr mit saurem Schweiss
%Zu sagen brauche, was ich nicht weiss;
%Dass ich erkenne, was die Welt
%Im Innersten zusammenh\"alt,
%Schau´ alle Wirkenskraft und Samen,
%Und tu´ nicht mehr in Worten kramen.
%\end{verbatim}    
%    \DescribeMacro{\infobox}
%    Eine Box, welche links unten auf der Seite platziert wird. Kann
%    grunds\"atzlich mit mehr oder weniger beliebigem Inhalt gef\"ullt werden.
%    Bietet sich gut an, um besondere Highlights oder Eckdaten hervorzuheben.
%    Beispiel:
%\begin{verbatim}
%\infobox{Highlights}{%
%    \footnotesize
%    \setlength\tabcolsep{2pt} % make table a bit more compact horizontally
%    \begin{tabular}{ll}
%    Category 1 & Property A \\
%               & Property B \\
%               & Property C \\
%               & Property D \\
%    Category 2 & Property E \\
%               & Property F \\
%               & Property G \\
%               & Property H \\
%    \end{tabular}}
%\end{verbatim}
%    \DescribeMacro{\includefactsheet}
%    L\"adt ein \verb|Content Document| in einem \verb|Master Document| (siehe
%    Abschnitt~\ref{sec:concept}).
% \fi
%^^A >>>

%^^A ------------------------------------------------------------------------- %
%^^A EXAMPLES <<<
% \ifenglish\section{Examples}\else\section{Beispiele}\fi
% \label{sec:examples}
% \ifenglish
%    The \verb|examples| directory contains a few examples.
% \else
%    Das Verzeichnis \verb|examples| enth\"alt einige Beispiele.
% \fi
%^^A No options, just a few things to pay attention to: ... list of things
%^^A >>>

%^^A ------------------------------------------------------------------------- %
%^^A STYLE GUIDELINES <<<
%^^A \ifenglish\section{Style Guidelines}\else\section{Stil-Richtlinien}\fi
%^^A \label{sec:examples}
%^^A \ifenglish
%^^A    TODO
%^^A \else
%^^A    TODO
%^^A \fi
%^^A No options, just a few things to pay attention to: ... list of things
%^^A >>>
%
%^^A ------------------------------------------------------------------------- %
%^^A IMPLEMENTATION <<<
% \section{Implementation}
% \label{sec:implementation}
%
% \StopEventually{\PrintIndex}
% \ifenglish
% If you are merely intending to use this class, you probably do not need to read this.
% \else
% Normale Benutzer brauchen diesen Abschnitt vermutlich nicht zu lesen.
% \fi
%^^A Need the class guards to prevent the macro code from being included again 
%^^A in template.tex and the German .tex file
%\iffalse
%<*class> 
%\fi
%    \begin{macrocode}
\newif\if@english
\newif\if@german
\DeclareOption{english}{\@englishtrue\@germanfalse}
\DeclareOption{german}{\@germantrue\@englishfalse}
\DeclareOption{deutsch}{\@germantrue\@englishfalse}
\ExecuteOptions{english,german,deutsch}
\ProcessOptions

% Packages
\LoadClass{report}
\RequirePackage[utf8]{inputenc}
\RequirePackage[T1]{fontenc}
\RequirePackage{graphicx}
\RequirePackage{tikz}
\RequirePackage{rviewport}
\RequirePackage{multicol}
\RequirePackage[inline]{enumitem}
\RequirePackage{geometry}
\RequirePackage[scaled=0.95]{helvet}
\renewcommand\familydefault{\sfdefault}
\RequirePackage{xcolor}
\RequirePackage{rotating}
% Define document colos palette
\definecolor{parties}{HTML}{666666}
\definecolor{title}{HTML}{000066}
\definecolor{summary}{HTML}{666666}
\colorlet{boxbg}{cyan!50!blue!20!white}
\RequirePackage{tcolorbox}
\RequirePackage{booktabs}
\RequirePackage[binary-units=true]{siunitx}
\sisetup{range-phrase=--}
\RequirePackage[%
    hidelinks=true,
    bookmarksnumbered=true]{hyperref}
\if@english
    \RequirePackage[english]{babel}
\fi
\if@german
    \RequirePackage[ngerman]{babel}
\fi

% Page Geometry
\geometry{a4paper,twoside=false}
\newlength{\factpageouter}
\newlength{\factpageinner}
\newlength{\factpagetop}
\newlength{\factpagebottom}
\newlength{\factpagewidth}
\newlength{\factgraphicswidth}
\newlength{\testl}
\setlength{\factpageouter}{25mm}
\setlength{\factpageinner}{25mm}
\setlength{\factpagetop}{8mm}
%    \end{macrocode}
%\iffalse
%</class>
%\fi
% Needed so that page numbers do not slide down too far.
%\iffalse
%<*class> 
%\fi
%    \begin{macrocode}
\setlength{\factpagebottom}{30mm}
%    \end{macrocode}
%\iffalse
%</class>
%\fi
% Rest of page geometry layout:
%\iffalse
%<*class> 
%\fi
%    \begin{macrocode}
\setlength{\factpagewidth}{\paperwidth}
\addtolength{\factpagewidth}{-\factpageinner}
\addtolength{\factpagewidth}{-\factpageouter}
\setlength{\factgraphicswidth}{50mm}
\geometry{outer=\factpageouter,inner=\factpageinner,%
          top=\factpagetop,    bottom=\factpagebottom}

\let\newcol\newpage
%    \end{macrocode}
%\iffalse
%</class>
%\fi
% We put the entire  page inside a minipage environment to  make it not take
% up more space than a single page. We  also set the height of that minipage
% to zero so that  it stays on the first page no matter  how much content is
% put inside it.
%\iffalse
%<*class> 
%\fi
%    \begin{macrocode}
\newenvironment{factpage}
    {\noindent\begin{minipage}[t][0pt]{\factpagewidth}}
    {\end{minipage}}
%    \end{macrocode}
%\iffalse
%</class>
%\fi
% Redefine Section Headings  to not be numbered by default. This  is done by
% adding a * after its sixth argument. The rest of the command is similar to
% what can be found in The LaTeX article.cls file.
%\iffalse
%<*class> 
%\fi
%    \begin{macrocode}
\renewcommand\section{\@startsection {section}{1}{\z@}%
                                     {-2.5ex \@plus -1ex \@minus -.2ex}%
                                     {1ex}%
                                     %{1ex \@plus.2ex}% default
                                     {\large\bfseries\color{title}}*}
%    \end{macrocode}
%\iffalse
%</class>
%\fi
% Having subsections  or even  lower-tiered headings in  a document  such as
% this one would be ridiculous. Remove them to make sure they are not used.
%\iffalse
%<*class> 
%\fi
%    \begin{macrocode}
\let\subsection\@undefined
\let\subsubsection\@undefined
\let\paragraph\@undefined
\let\subparagraph\@undefined
%    \end{macrocode}
%\iffalse
%</class>
%\fi
% Define language-specific terms.
%\iffalse
%<*class> 
%\fi
%    \begin{macrocode}
\iflanguage{english}{%
    \newcommand*{\lang@team}    {\textbf{Project Team: }}
    \newcommand*{\lang@client}  {\textbf{Client: }}
    \newcommand*{\lang@coach}   {\textbf{Coach: }}
    \newcommand*{\lang@coaches} {\textbf{Coaches: }}
    \newcommand*{\lang@expert}  {\textbf{Expert: }}
    \newcommand*{\lang@experts} {\textbf{Experts: }}
    \newcommand*{\lang@partner} {\textbf{Industry Partner: }}
    \newcommand*{\lang@partners}{\textbf{Industry Partners: }}
    \newcommand*{\lang@source}  {\textit{Source: }}
}{}
\iflanguage{ngerman}{%
    \newcommand*{\lang@team}    {\textbf{Arbeitsgruppe: }}
    \newcommand*{\lang@client}  {\textbf{Auftraggeber: }}
    \newcommand*{\lang@coach}   {\textbf{Betreuer: }}
    \newcommand*{\lang@coaches} {\textbf{Betreuer: }}
    \newcommand*{\lang@expert}  {\textbf{Experte: }}
    \newcommand*{\lang@experts} {\textbf{Experten: }}
    \newcommand*{\lang@partner} {\textbf{Industriepartner: }}
    \newcommand*{\lang@partners}{\textbf{Industriepartner: }}
    \newcommand*{\lang@source}  {\textit{Quelle: }}
}{}
%    \end{macrocode}
%\iffalse
%</class>
%\fi
% Save those terms in the appropriate macros.
%\iffalse
%<*class> 
%\fi
%    \begin{macrocode}
\newcommand*{\team}[1]
    {\gdef\@team    {{\footnotesize\color{parties}\lang@team#1}}}
\newcommand*{\client}[1]
    {\gdef\@client  {{\footnotesize\color{parties}\lang@client#1}}}
\newcommand*{\coach}[1]
    {\gdef\@coach   {{\footnotesize\color{parties}\lang@coach#1}}}
\newcommand*{\coaches}[1]
    {\gdef\@coaches {{\footnotesize\color{parties}\lang@coaches#1}}}
\newcommand*{\expert}[1]
    {\gdef\@expert  {{\footnotesize\color{parties}\lang@expert#1}}}
\newcommand*{\experts}[1]
    {\gdef\@experts {{\footnotesize\color{parties}\lang@experts#1}}}
\newcommand*{\partner}[1]
    {\gdef\@partner {{\footnotesize\color{parties}\lang@partner#1}}}
\newcommand*{\partners}[1]
    {\gdef\@partners{{\footnotesize\color{parties}\lang@partners#1}}}
%    \end{macrocode}
%\iffalse
%</class>
%\fi
% Redefine \verb|maketitle| to do our bidding:
%\iffalse
%<*class> 
%\fi
%    \begin{macrocode}
\renewcommand\maketitle{\par
    \begingroup
        {\Huge\color{title}\@title}\par
        \vspace{2em}
        \global\let\maketitle\relax
        \global\let\@maketitle\relax
        \global\let\title\relax
        \global\let\date\relax
        \global\let\@title\@empty
        \global\let\@date\@empty
    \endgroup
}
%    \end{macrocode}
%\iffalse
%</class>
%\fi
% For testing purposes:
%\iffalse
%<*class> 
%\fi
%    \begin{macrocode}
\newlength{\placeholderbase}
\newcommand\placeholderpics{%
    \setlength{\placeholderbase}{48mm}
    \begin{tikzpicture}[x=\placeholderbase,y=\placeholderbase,draw=black!50!white,very thick]
        \draw (0,0) -- (1.618,1);
        \draw (0,1) -- (1.618,0);
        \draw (0,0) -- (1.618,0) -- (1.618,1) -- (0,1) -- cycle;
        \node[draw,fill=white] at (0.809,0.5) {\large Beautiful Picture!};
    \end{tikzpicture}
    \begin{tikzpicture}[x=\placeholderbase,y=\placeholderbase,draw=black!50!white,very thick]
        \draw (0,0) -- (1.618,1);
        \draw (0,1) -- (1.618,0);
        \draw (0,0) -- (1.618,0) -- (1.618,1) -- (0,1) -- cycle;
        \node[draw,fill=white] at (0.809,0.5) {\large Amazing Graphics!};
    \end{tikzpicture}\par
    {\footnotesize Beautiful Picture and Amazing Graphics}
}
%    \end{macrocode}
%\iffalse
%</class>
%\fi
% Store the concents of \verb|\fscontent| in \verb|\fs@content|
%\iffalse
%<*class> 
%\fi
%    \begin{macrocode}
\newcommand\fscontent[1]{\gdef\fs@content{#1}}
%    \end{macrocode}
%\iffalse
%</class>
%\fi
% Print the contents of \verb|\fs@content|
%\iffalse
%<*class> 
%\fi
%    \begin{macrocode}
\newcommand\fs@makecontent{\par
    \vspace{1em}\begin{multicols}{3}
        \ifdef{\fs@content}{\fs@content}\par
    \end{multicols}
}
%    \end{macrocode}
%\iffalse
%</class>
%\fi
% Store the concents of \verb|\fssummary| in \verb|\fs@summary|
%\iffalse
%<*class> 
%\fi
%    \begin{macrocode}
\newcommand\fssummary[1]{\gdef\fs@summary{#1}}
%    \end{macrocode}
%\iffalse
%</class>
%\fi
% Print the contents of \verb|\fs@summary|
%\iffalse
%<*class> 
%\fi
%    \begin{macrocode}
\newcommand\fs@makesummary{\par
    \vspace{0.25em}\noindent\begin{minipage}{\textwidth}\large
        \ifdef{\fs@summary}{\color{summary}\fs@summary}\par
    \end{minipage}%
}
%    \end{macrocode}
%\iffalse
%</class>
%\fi
% Store the concents of \verb|\fsgraphics| in \verb|\fs@graphics|
%\iffalse
%<*class> 
%\fi
%    \begin{macrocode}
\newcommand\fsgraphics[1]{\gdef\fs@graphics{#1}}
\newcommand\graphicssource[1]{\gdef\fs@graphicssource{#1}}
%    \end{macrocode}
%\iffalse
%</class>
%\fi
% Put the contents of \verb|\graphicscaption| inside an \verb|\mbox| so that it does not wrap.
%\iffalse
%<*class> 
%\fi
%    \begin{macrocode}
\newcommand\graphicscaption[1]{\mbox{#1}}
%    \end{macrocode}
%\iffalse
%</class>
%\fi
% Print the contents of \verb|\fs@graphics|
%\iffalse
%<*class> 
%\fi
%    \begin{macrocode}
\newcommand\fs@makegraphics{\par
    \vspace{1em}\noindent\begin{minipage}[b]{\textwidth}%
        \footnotesize\raggedright
        \ifdef{\fs@graphics}{\fs@graphics}\par\par
        \ifdef{\fs@graphicscaption}{\fs@graphicscaption}\par
    \end{minipage}%
    \begin{minipage}[t]{2ex}\footnotesize\color{parties}
        \ifdef{\fs@graphicssource}{%
            \hspace*{2ex}%
            \turnbox{90}{%
                \hspace*{2ex} \lang@source \fs@graphicssource}}\par
    \end{minipage}%
}
%    \end{macrocode}
%\iffalse
%</class>
%\fi
% Store the content of \verb|\infobox| and print it:
%\iffalse
%<*class> 
%\fi
%    \begin{macrocode}
\newcommand\infobox[2]{%
    \gdef\fs@infoboxtitle{#1}%
    \gdef\fs@infoboxcontent{\footnotesize #2}%
}
\newcommand\fs@makeinfobox{
    \begin{tcolorbox}[
            title=\fs@infoboxtitle,
            width=0.66\textwidth,
            arc=0mm,
            before=,
            after=\hfill]
        \fs@infoboxcontent
    \end{tcolorbox}
}
\newcommand\tightlist{%
    \setlength{\itemsep}{0pt}\setlength{\parskip}{0pt}%
}
%    \end{macrocode}
%\iffalse
%</class>
%\fi
% Errors:
%\iffalse
%<*class> 
%\fi
%    \begin{macrocode}
\newcommand\fs@checkinput{%
    \ifdef{\@coach}{%
        \ifdef{\@coaches}{%
            \ClassError{fhnwfactsheet}{Both 'coach' and 'coaches' field defined. Undefine one.}
                {Both the 'coach' and the 'coaches' field are defined. Only one of them should be.}
        }{}%
    }{}%
    \ifdef{\@expert}{%
        \ifdef{\@experts}{%
            \ClassError{fhnwfactsheet}{Both 'expert' and 'experts' field defined. Undefine one.}
                {Both the 'expert' and the 'experts' field are defined. Only one of them should be.}
        }{}%
    }{}%
    \ifdef{\@partner}{%
        \ifdef{\@partners}{%
            \ClassError{fhnwfactsheet}{Both 'partner' and 'partners' field defined. Undefine one.}
                {Both the 'partner' and the 'partners' field are defined. Only one of them should be.}
        }{}%
    }{}%
    \ifdef{\fs@infoboxcontent}{}
        {%
            \ClassError{fhnwfactsheet}{Did you forget to specify your infobox?}
            {The infobox contents are not defined. Make sure your infobox command is properly called.}
        }%
    \ifdef{\fs@infoboxtitle}{}
        {%
            \ClassError{fhnwfactsheet}{Did you forget to specify your infobox?}
            {The infobox title is not defined. Make sure your infobox command is properly called.}
        }%
}
%    \end{macrocode}
%\iffalse
%</class>
%\fi
% Chapter code. Taken mostly from \verb|classes.dtx|.
%\iffalse
%<*class> 
%\fi
%    \begin{macrocode}
\renewcommand\chapter{%\if@openright\cleardoublepage\else\clearpage\fi
                    \thispagestyle{plain}%
                    \global\@topnum\z@
                    \@afterindentfalse
                    \secdef\@chapter\@schapter}
\def\@chapter[#1]#2{\ifnum \c@secnumdepth >\m@ne
                         \refstepcounter{chapter}%
                         \typeout{\@chapapp\space\thechapter.}%
                         \addcontentsline{toc}{chapter}{#1}%
                                   %{\protect\numberline{\thechapter}#1}%
                    \else
                      \addcontentsline{toc}{chapter}{#1}%
                    \fi
                    \chaptermark{#1}%
                    \addtocontents{lof}{\protect\addvspace{10\p@}}%
                    \addtocontents{lot}{\protect\addvspace{10\p@}}%
                    \if@twocolumn
                      \@topnewpage[\@makechapterhead{#2}]%
                    \else
                      \@makechapterhead{#2}%
                      \@afterheading
                    \fi}
\def\@makechapterhead#1{%
    \begingroup
        \mbox{\Huge\color{title}#1}\par
        \vspace{2em}
    \endgroup
}
\def\@makeschapterhead#1{%
  \vspace*{50\p@}%
  {\parindent \z@ \raggedright
    \normalfont
    \interlinepenalty\@M
    \Huge\color{title} #1\par\nobreak
    \vskip 40\p@
  }}
\renewcommand*\l@chapter[2]{%
  \ifnum \c@tocdepth >\m@ne
    \addpenalty{-\@highpenalty}%
    \vskip 1.0em \@plus\p@
    \setlength\@tempdima{1.5em}%
    \begingroup
      \parindent \z@ \rightskip \@pnumwidth
      \parfillskip -\@pnumwidth
      \leavevmode 
      \advance\leftskip\@tempdima
      \hskip -\leftskip
      %\@dottedtocline{0}{0em}{0em}{#1}{#2}\par
      #1\nobreak\hfil \nobreak\hb@xt@\@pnumwidth{\hss #2}\par
      \penalty\@highpenalty
    \endgroup
  \fi}
%    \end{macrocode}
%\iffalse
%</class>
%\fi
% Including a fact sheet content file in a master document:
%\iffalse
%<*class> 
%\fi
%    \begin{macrocode}
\newcommand*\includefactsheet[1]{\input{#1}\makefactsheet\newpage}
%    \end{macrocode}
%\iffalse
%</class>
%\fi
% Output routine:
%\iffalse
%<*class> 
%\fi
%    \begin{macrocode}
\newcommand\makefactsheet{
    \begin{factpage}
        \fs@checkinput
        \chapter{\@title}
        % Must be after \chapter
        % Note: We do not want to overwrite the \chapter pagestyle{plain} everywhere.
        \thispagestyle{empty}
        \fs@makesummary
        \vspace{1ex}
        \fs@makegraphics
        \vspace{1ex}
        \fs@makecontent
        \vspace{1ex}
        \tcbset{
            colback=boxbg,
            colframe=title,
            boxrule=0.5pt,
            toptitle=2pt,
            bottomtitle=1pt,
            titlerule=0.5pt,
        }
        \fs@makeinfobox
        \begin{minipage}[b]{0.30\textwidth}
            \ifdef{\@team}{\@team}{}\par
            \ifdef{\@client}{\@client}{}\par
            \ifdef{\@coach}{\@coach}{}\par
            \ifdef{\@coaches}{\@coaches}{}\par
            \ifdef{\@expert}{\@expert}{}\par
            \ifdef{\@experts}{\@experts}{}\par
            \ifdef{\@partner}{\@partner}{}\par
            \ifdef{\@partners}{\@partners}{}\par
        \end{minipage}
    \end{factpage}\raisebox{6ex}{\hspace{3em}\thepage}
    % Undefine for next fact sheet
    \let\fs@graphics\@undefined
    \let\fs@graphicssource\@undefined
    \let\fs@graphicscaption\@undefined
    \let\@team\@undefined
    \let\@client\@undefined
    \let\@coach\@undefined
    \let\@coaches\@undefined
    \let\@expert\@undefined
    \let\@experts\@undefined
    \let\@partner\@undefined
    \let\@partners\@undefined
    \let\fs@infoboxtitle\@undefined
    \let\fs@infoboxcontent\@undefined
}
%    \end{macrocode}
%\iffalse
%</class>
%\fi
%
% \Finale
%^^A >>>
%
%^^A ------------------------------------------------------------------------- %
%^^A TEMPLATE <<<
% \iffalse
%<*template>
\documentclass{fhnwfactsheet}
\usepackage{lipsum}
%\usepackage{FiraSans}

%% ================================================================= SET TITLE %
\title{Phenomenal Title!}
\author{Glorious Author!}

%% ============================================================== END PREAMBLE %
\begin{document}
\begin{factpage}
%% ============================================================= BEGIN CONTENT %

%% ------------------------------------------------------------------ OVERVIEW %
\begin{overview}
    \lipsum[2]
\end{overview}

\begin{content}

    \section{This is a section}
    \lipsum[2]
    \section{This is another section}
    \lipsum[4]
    \section{And another one!}
    \lipsum[5]

\end{content}

\lipsum[1]

\factgraphics{a-stunning-picture.png}{a-very-nice-graphic.png}

%% ------------------------------------------------------------------ STUDENTS %
\begin{students}
    \lipsum[2]
\end{students}

%% ------------------------------------------------------------------- COACHES %
\begin{coaches}
    \lipsum[2]
\end{coaches}

%% ----------------------------------------------------------------- PARTNERS %
\begin{partners}
    \lipsum[2]
\end{partners}

%% ============================================================== END CONTENT %
\end{factpage}
\end{document}
%</template>
% \fi
%^^A >>>
%
%^^A ------------------------------------------------------------------------- %
%^^A MULTI-LANGUAGE SUPPORT <<<
% \iffalse
%<*de>
\expandafter\newif\csname ifenglish\endcsname
\englishfalse
\input{fhnwfactsheet.dtx}
%</de>
% \fi
%^^A >>>
%
%^^A ------------------------------------------------------------------------- %
%^^A CHARACTER TABLE <<<
% \CharacterTable
%  {Upper-case    \A\B\C\D\E\F\G\H\I\J\K\L\M\N\O\P\Q\R\S\T\U\V\W\X\Y\Z
%   Lower-case    \a\b\c\d\e\f\g\h\i\j\k\l\m\n\o\p\q\r\s\t\u\v\w\x\y\z
%   Digits        \0\1\2\3\4\5\6\7\8\9
%   Exclamation   \!     Double quote  \"     Hash (number) \#
%   Dollar        \$     Percent       \%     Ampersand     \&
%   Acute accent  \'     Left paren    \(     Right paren   \)
%   Asterisk      \*     Plus          \+     Comma         \,
%   Minus         \-     Point         \.     Solidus       \/
%   Colon         \:     Semicolon     \;     Less than     \<
%   Equals        \=     Greater than  \>     Question mark \?
%   Commercial at \@     Left bracket  \[     Backslash     \\
%   Right bracket \]     Circumflex    \^     Underscore    \_
%   Grave accent  \`     Left brace    \{     Vertical bar  \|
%   Right brace   \}     Tilde         \~}
%^^A >>>
\endinput
%^^A vim: foldenable foldcolumn=2 foldmethod=marker foldmarker=<<<,>>>

%</de>
% \fi
%^^A >>>
%
%^^A ------------------------------------------------------------------------- %
%^^A CHARACTER TABLE <<<
% \CharacterTable
%  {Upper-case    \A\B\C\D\E\F\G\H\I\J\K\L\M\N\O\P\Q\R\S\T\U\V\W\X\Y\Z
%   Lower-case    \a\b\c\d\e\f\g\h\i\j\k\l\m\n\o\p\q\r\s\t\u\v\w\x\y\z
%   Digits        \0\1\2\3\4\5\6\7\8\9
%   Exclamation   \!     Double quote  \"     Hash (number) \#
%   Dollar        \$     Percent       \%     Ampersand     \&
%   Acute accent  \'     Left paren    \(     Right paren   \)
%   Asterisk      \*     Plus          \+     Comma         \,
%   Minus         \-     Point         \.     Solidus       \/
%   Colon         \:     Semicolon     \;     Less than     \<
%   Equals        \=     Greater than  \>     Question mark \?
%   Commercial at \@     Left bracket  \[     Backslash     \\
%   Right bracket \]     Circumflex    \^     Underscore    \_
%   Grave accent  \`     Left brace    \{     Vertical bar  \|
%   Right brace   \}     Tilde         \~}
%^^A >>>
\endinput
%^^A vim: foldenable foldcolumn=2 foldmethod=marker foldmarker=<<<,>>>

%</de>
% \fi
%^^A >>>
%
%^^A ------------------------------------------------------------------------- %
%^^A CHARACTER TABLE <<<
% \CharacterTable
%  {Upper-case    \A\B\C\D\E\F\G\H\I\J\K\L\M\N\O\P\Q\R\S\T\U\V\W\X\Y\Z
%   Lower-case    \a\b\c\d\e\f\g\h\i\j\k\l\m\n\o\p\q\r\s\t\u\v\w\x\y\z
%   Digits        \0\1\2\3\4\5\6\7\8\9
%   Exclamation   \!     Double quote  \"     Hash (number) \#
%   Dollar        \$     Percent       \%     Ampersand     \&
%   Acute accent  \'     Left paren    \(     Right paren   \)
%   Asterisk      \*     Plus          \+     Comma         \,
%   Minus         \-     Point         \.     Solidus       \/
%   Colon         \:     Semicolon     \;     Less than     \<
%   Equals        \=     Greater than  \>     Question mark \?
%   Commercial at \@     Left bracket  \[     Backslash     \\
%   Right bracket \]     Circumflex    \^     Underscore    \_
%   Grave accent  \`     Left brace    \{     Vertical bar  \|
%   Right brace   \}     Tilde         \~}
%^^A >>>
\endinput
%^^A vim: foldenable foldcolumn=2 foldmethod=marker foldmarker=<<<,>>>

%</de>
% \fi
%^^A >>>
%
%^^A ------------------------------------------------------------------------- %
%^^A CHARACTER TABLE <<<
% \CharacterTable
%  {Upper-case    \A\B\C\D\E\F\G\H\I\J\K\L\M\N\O\P\Q\R\S\T\U\V\W\X\Y\Z
%   Lower-case    \a\b\c\d\e\f\g\h\i\j\k\l\m\n\o\p\q\r\s\t\u\v\w\x\y\z
%   Digits        \0\1\2\3\4\5\6\7\8\9
%   Exclamation   \!     Double quote  \"     Hash (number) \#
%   Dollar        \$     Percent       \%     Ampersand     \&
%   Acute accent  \'     Left paren    \(     Right paren   \)
%   Asterisk      \*     Plus          \+     Comma         \,
%   Minus         \-     Point         \.     Solidus       \/
%   Colon         \:     Semicolon     \;     Less than     \<
%   Equals        \=     Greater than  \>     Question mark \?
%   Commercial at \@     Left bracket  \[     Backslash     \\
%   Right bracket \]     Circumflex    \^     Underscore    \_
%   Grave accent  \`     Left brace    \{     Vertical bar  \|
%   Right brace   \}     Tilde         \~}
%^^A >>>
\endinput
%^^A vim: foldenable foldcolumn=2 foldmethod=marker foldmarker=<<<,>>>
